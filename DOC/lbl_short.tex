\documentclass[11pt]{article}

%%%package{lucbr,graphicx,fancyhdr,longtable}
%%%package[expert,bullet]{lucidabr,graphicx,fancyhdr,longtable}
\usepackage{graphicx,fancyhdr,longtable}
\usepackage{soul}
\newcommand{\kc}{\textsf{kCARTA}\xspace}
\newcommand{\cm}{\hbox{cm}}

% make a single, doubly indented line 
% (mainly used for driver file examples)
\newcommand{\ttab}{\indent\indent}

\input ASL_defs

\DeclareGraphicsExtensions{.pdf,.jpg,.png,.eps}


\setlength{\textheight}{7.5in}
\setlength{\topmargin}{0.25in}
\setlength{\oddsidemargin}{.375in}
\setlength{\evensidemargin}{.375in}
\setlength{\textwidth}{5.75in}

\newlength{\colwidth}
\setlength{\colwidth}{8cm}
\newlength{\colwidthshort}
\setlength{\colwidthshort}{6cm}

\pagestyle{fancy}

% \date{July 5, 1994} % if you want a hardcoded date

\lhead{\textbf{\textsf{DRAFT}}}
\chead{UMBC\_LBL}
\rhead{\textsf{Version 7}}
\lfoot{UMBC}
\cfoot{}
\rfoot{\thepage}

\newcommand{\HRule}{\rule{\linewidth}{1mm}}
\newcommand{\HRulethin}{\rule{\linewidth}{0.5mm}}

\begin{document}
\thispagestyle{empty}
\vspace{2.0in}

\noindent\HRule
\begin{center}
\Huge \textbf{\textsf{UMBC\_LBL} vers 8}: An Algorithm to Compute Line-by-Line 
Spectra
\end{center}
\noindent\HRule

\vspace{0.75in}
\begin{center}
\begin{Large}
Sergio De Souza-Machado, L. Larrabee Strow,\\ David Tobin, Howard Motteler 
and Scott Hannon
\end{Large}
\end{center}

\vspace{0.5in}
\begin{center}
Physics Department\\
University of Maryland Baltimore County\\Baltimore, MD 21250 USA\\
\end{center}

\vspace{0.5in}
\begin{center}
Copyright 2011 \\
University of Maryland Baltimore County \\
All Rights Reserved\\
v6  \today\\
\end{center}

\vfill

\noindent\HRulethin
\begin{flushleft}
\begin{tabbing}
Sergio~De~Souza-Machado: \=    sergio@umbc.edu \\
L.~Larrabee~Strow:   \>        strow@umbc.edu\\
\end{tabbing}
\end{flushleft}

%\begin{flushright}
%\includegraphics[width=1.0in]{umseal.eps}
%\end{flushright}

\newpage
\tableofcontents
\listoftables
\listoffigures

\newpage

\begin{center}
{\bf ABSTRACT}
\end{center}

  We have developed a line-by-line code to compute the spectral lineshapes
  of gases under varying pressure and temperatures. This code can use the
  spectral line parameters of a variety of databases such as HITRAN or 
  GEISA. 
  The code ensures that the same lines are used in all layers of an
  atmosphere, so that the resulting line profiles are smooth. The latest
  models for CO2 line mixing and in water vapor lineshapes are incorporated
  in the code. The code is can be used to compute absorption spectra
  at user specified resolutions, using one of various lineshapes.
  One of our uses for the code is to generate new versions of the 
  spectroscopic database for kCARTA, which is the reference for the AIRS
  forward model.

This document is very much a work in progress.  Some major omissions
include references, significant examples of output, and comparisons of 
output to GENLN2.  These omissions will
be rectified in the future.  Please give us your feedback on both the
code and the documentation!  


\newpage
\section{Introduction}
This documentation describes a line-by-line code that can be used to 
compute optical depths for various gases (that appear in the HITRAN 
database). The code has been written in Matlab, with many of the routines 
written as FORTRAN-MEX files in the interests of speed. The lineshape 
database that is used in the computations, is currently the HITRAN 2008
database. Other databases, such as the GEISA or HITRAN92,96,2K,2004 
databases, can be merged in and used. With this, we can incorporate the 
latest lineshape studies and parameters into our computations. 

The HITRAN homepage is at http://www.cfa.harvard.edu/hitran/. 
\begin{figure}[h]
  \begin{center}\includegraphics[width=6in]{Figures/HITRAN_cover}\end{center}
  \caption{HITRAN home page}
  \label{fig:hitran}
\end{figure}

This code will be used to generate a new database for kCARTA, our radiative
transfer algorithm that is the reference for the AIRS forward model. AIRS is
a high resolution infrared instrument due to be launched by NASA in the 
year 2000. The instrument will be used to make measurements of the gas 
content and temperatures of the Earth's atmosphere, as well as study the 
global climate and weather. 

For all gases except carbon dioxide and water vapor, the lineshapes of the
individual lines are simply added together to give the overall absorption
spectrum. To speed things up, the code uses the binning methods of GENLN2, 
where the lines are divided into near lines, medium-far lines and far 
lines. The general code to compute the lineshape of this majority of gases 
is called $run6.m$ Note that we also have code to incorporate the 
computation of the lineshapes of the cross section gases as well.

For the case of carbon dioxide, we have utilised Dave Tobin's work on 
linemixing and duration of collisions effects in the 4 $\mu$m band, and
applied it to the 15 $\mu$m band as well. The code to compute the lineshape 
of carbon dioxide is called $run6co2.m$

For the case of water vapor, we have written two source codes. $run6water.m$
can compute the spectral lineshapes with or without the basement term removed,
and add on the user specified continuum (CKD 0,2.1,2.3 or 2.4). In addition,
this code also includes Tobin's work on water vapor chi functions. We also
have a specialised continuum code  $run6watercontinuum.m$ that can output one 
of the following (a) the completely computed water continuum (b) the self
continuum coefficients (c) the foreign continuum coefficients or (d) 
combinations of the above three.

We begin the documentation by briefly describing how our code computes an 
optical depths, given the necessary input parameters. This section is 
divided into three; the first can be used for all gases, while the second 
and third apply to water vapor and carbon dioxide respectively.

The next sections describe the $UMBC\_LBL$ Version 6 code implemetation. 
A section on the basic code for most gases, $run8.m$ is then followed by 
section on \\$run8water(continuum).m$ and $run8co2.m$. We have improved upon 
Version 6 by defaulting most parameters to a set of values that are usually 
used; if the user wants to change these settings, he/she can send in the new 
value(s) by using a Matlab structure. This will be the $UMBC\_LBL$ Version 8
implementation. All the basic physics is the same as version 6.

Theis is followed by sections which describe spectral 
lineshapes in detail. First is a section  on general spectral lineshape 
theory, where the meaning of terms such as line widths and line broadening 
is explained. Next are two sections, one on 
computing water vapor spectra and the other computing carbon dixide 
spectra. As mentioned in the previous paragraphs, to accurately compute 
the absorption spectra of these two gases, one needs to use more involved 
lineshape theories than those for other gases. This is particularly 
important as these two gases are actively important in the infrared 
portion of the radiation emitted by the Earth, and so a thorough 
understanding of their spectra would be beneficial to the remote sensing 
community. 
 
Finally, a short set of appendices briefly describe the general code to 
compute partition functions, line widths and line strengths. 

\newpage
\section{General algorithm to compute optical depth}
For each gas cell or atmopheric layer, the code expects a 1x5 array of 
inputs \\
\hl{[index   P   PP    T   q     ] = }
\\
\begin{verbatim}
  [index    press  part.press   temp   col.density ] = 
  [integer  real   real         real   real        ]  = 
  [nounits  atm    atm          K      kmoles/cm2  ]
\end{verbatim}

Assume the user wants the optical depth to be computed for a certain gas, 
using known parameters. These parameters are the total pressure $P$ , 
partial (self) pressure $PS$, temperature $T$ and gas amount (column
density) $U$. The gas amount $U$ is related to self pressure, temperature,
path cell length (or atmospheric layer thickness) $L$ by the equation 
\[ 
U = \frac{CF \times L \times PS}{R \times T}
\]
where R is the molar gas constant (8.31 J/mol/K), $P,PS$ are the total and
partial pressures in atmospheres (only partial pressure is relevant for gas
amount), $T$ is the temperature in Kelvin, and $L$ is the path length in cm.  
$CF$ is a units conversion factor of$1.01325 \times 10^{-4}$ to give us $U$ in
units of $kiloMoles/cm^{2}$. This equation uses the Ideal Gas Law to calculate
molecular density for a unit volume, and then multiplies the density by the 
path length $L$ to convert density to column density (gas amount).
 
Knowing for what gas the spectrum is to be computed, and within which 
frequency interval, the lineshape parameters for that gas need to be read 
in. These  parameters can be obtained from a database such as HITRAN98, 
and should include the following : 

\begin{longtable}{lll}
\hline
\hline
TYPE &  DESCRIPTION & USE  \\
\hline
\hline
gasid   & input HITRAN GasID    & 1=water,2=CO2, etc \\
iso     &  list of isotopes     & 1=most abundant,2=next ... \\
\hline

linct   & number of lines in wavenumber interval & \\
wnum    & wavenumbers of the line centers [$cm^{-1}$] 
        & $\nu_{0}(1),\nu_{0}(2),...$ \\
tsp     & line center shift due to pressure P [cm-1/atm] 
        & $\nu_{0}(j) \rightarrow \nu_{0}(j) + P \times tsp(j)$ \\
stren   & line strengths (See Eqn \ref{eqn:linestren}) 
        & $S(T) \simeq S(296)Z(296)/Z(T) \times \rho$\\
        & [$cm^{-1}/(molecules cm^{-2})$] 
        & Z are partition functions (Eqn \ref{eqn:gamache}) \\ 
        & & $\rho \simeq $ upper/lower state populations \\
\hline
abroad  & air broadened half widths HWHM [$cm^{-1}/atm$] 
        & $brd_{air} = (P - PS) \times abroad$ \\
sbroad  & self broadened half widths HWHM [$cm^{-1}/atm$] 
        & $brd_{self} = (PS) \times sbroad$ \\
abcoef  & temperature dependence of air 
        & $brd = brd_{air} + brd_{self}$\\
      & broadened half width 
      & $brd \rightarrow brd \times (296/T)^{abcoef}$ \\
\hline
 
tprob & transition probabilility [$debyes^{2}$] & \\
els   & lower state energy [$cm^{-1}$] & computing S(T) \\
\hline

usgq & Upper state global quanta index & identifying \\
lsgq & Lower state global quanta index & P,Q,R branches \\
\hline

uslq & Upper State local quanta & local quantum numbers \\
bslq & Lower State local quanta & local quantum numbers \\
\hline

ai  & accuracy  indices & \\
ref & lookup for references & \\
\hline
\hline 
\end{longtable}

Note that the $stren$ parameter is used in units of 
$cm^{-1}/(moleculescm^{-2})$ while the gas amounts $U$ are in units of 
$kiloMoles/cm^{2}$. To get absorption coefficients and optical depths that 
are in the correct units, the program eventually multiplies $stren$ by 
$6.023 \times 10^{26}$, which is the number of molecules per kilomole
(ie Avagadro's Number times 1000). 

Having read in these parameters, the code is almost ready to proceed with 
the computations. Before this, it needs to read in the mass of each of the 
isotopes of the current gas. In addition, it needs to compute the 
partition function $Z$, and thus it needs four parameters $a, b, c, d$ for 
each isotope (refer to Eqn \ref{eqn:gamache}). For the current temperature 
$T$ , $Z(T)$ is then readily computed as : 
\[
Z(T ) = a + bT + cT^{2} + dT^{3}
\] 
The partition function is used when computing the line strength, as that 
term is proportional to the relative populations of the lower and upper 
levels of the transition. 

Using parameter $tsp$, any shift of the line centers due 
to the total pressure is then computed,  giving the adjusted line centers 
\[ 
\nu_{0}(P) = \nu_{0} + P \times tsp(j)
\]

The broadening of each line, due to the self component and the foreign 
component, is then computed. 
\[
brd_{air} = (P - PS) \times abroad
\]
\[
brd_{self} = (PS) \times sbroad
\]
\[
brd = brd_{air} + brd_{self}
\]
\[
brd \rightarrow brd \times (296/T)^{abcoef}
\]

With the above information for each line having been determined, the 
absorption coefficients for the individual lines can now be computed, 
using the required lineshape (Lorentz, Doppler, Voigt, etc). Looking at 
the simplest lineshape equation, the Lorentz lineshape is given by 
\[
 k_{L}(\nu)=\frac{1}{\pi}\left(\frac{\gamma_{L}}
{(\nu-\nu_{0})^{2}+\gamma_{L}^{2}}\right)
\]
where $\gamma_{L}$ is the linewidth and $\nu,\nu_{0}$ are the wavenumber
and line center frequency respectively. This means the units that result
from a lineshape computation is $cm$

The final term required to compute the absorption spectrum is the line
strength associated with each line. In the equation below, $S_{i}(T_{ref})$ 
is the line strength at the HITRAN reference temperature of $T_{ref} = 296K$, 
$E_{i}$ is the lower state energy (given by $els$) and $\nu_{i}$ is the line 
center wavenumber : 
\[
 S_i (T) = S_i (T_{ref})
 \frac{ Z(T_{ref}) }{ Z(T) }
 \frac{ \exp(-hcE_i /kT) }{ \exp(-hcE_i /kT_{ref}) }
 \frac{ [1-\exp(-hc\nu_i /kT)] }{ [1-\exp(-hc\nu_i /kT_{ref})] }
\]
With $S_{i}$ being in units of $cm^{-1}/(molecules cm^{-2})$, when this
mutliplies the lineshape, the final abosrption coeficient units are 
 $1/(molecules cm^{-2}) $
 
By summing over the individual lines, the absorption spectra can be 
computed. However, for purposes of radiative transfer in the atmosphere, 
it is more convenient to think in terms of optical depths (and 
transmittances) instead of absorption spectra. With this in mind, and to 
get the units right, the adjusted line strength is simply multiplied by 
the gas amount times number of molecules per kilomole : 
$U (kiloMoles/cm^{2}) \times 6.023 \times 10^{26} (molecules/kiloMoles)$,
from which the optical depth of the line 
(in dimensionless units) can be computed. By summing over all the 
individual optical depths, the total optical depth can be computed : 
\[
k_{i}^{optical depth}(\nu,\nu_{0}) = S_{i}(T) \times U 
\times 6.023E^{26} \times LineShape(P, PS, T, \nu_{0}(i), \gamma(i), mass(i))
\]
\[ 
k(\nu)^{optical depth} = \sum_{i} k_{i}^{optical depth}(\nu,\nu_{0})
\] 

\begin{figure}[h]
  \begin{center}\includegraphics[width=6in]{Figures/fig0}\end{center}
  \caption[Computing absorption spectra]{Flow diagram to compute $k(\nu)$}
  \label{fig:easy}
\end{figure}

The actual steps described above are summarized in Fig. \ref{fig:easy}. 
In the case of 
water vapor  and carbon dioxide, the spectral lineshape that is used for 
each line can be more complicated than a simple lorentz or voigt shape. 
This will be discussed in more detail later. For now, let us obtain some 
order of magnitude estimates for the optical depths of some of the gases 
in the atmosphere. The gas amounts $U$ come from the AIRS layers. Assume 
the temperature $T$ is 296 K, so all temperature effects are unimportant.
Here we use units $\eta = molecules/cm^{2}$ for gas amounts $U$. Following
are typical line paramter values obtained from the HITRAN database, in
the regions where the optical depths for the individual gases peak in
the atmosphere : 

\begin{longtable}{lllllll} 
\hline
\hline
gas & wavenumber & abroad         & sbroad       & S & U & mix ratio \\
   & $cm^{-1}$  & $cm^{-1}/atm$ & $cm^{-1}/atm$ &$cm^{-1}/\eta$ & $\eta$ &\%\\ 
\hline
\hline
water & 1500 & 0.08 & 0.4  & 2E-19 & 5.6E+21 & 0.77    \\
CO2   & 2400 & 0.08 & 0.1  & 4E-18 & 2.2E+20 & 0.0036  \\
N2    & 2400 & 0.05 & 0.05 & 4E-28 & 4.8E+23 & 78.1    \\
CH4   & 1320 & 0.06 & 0.09 & 1E-19 & 1.0E+18 & 0.00017 \\
\hline
\hline
\end{longtable} 

The amount of water vapor in the atmosphere varies greatly both in
time and space.  The mixing ratio of water near the surface might be
anywhere between 0.01\% and 4\%.  The amount of CO2 and N2 varies by only
a few percent, while CH4 may vary by a factor of two.  Note that due
to the size and range of variation of the water mixing ratio, as well
as the lage difference between air- and self- broadened widths, that
the total line widths can vary by upwards of 15% depending upon the
amount of water in the atmosphere.  That is, the widths of water
lines in a dry atmosphere will be noticeably narrower than the widths
in a wet atmosphere.  This is not true for any of the other important
atmospheric gases; their widths are nearly constant.  This is because
either their mixing ratio is relatively constant, very tiny, or their
air- and self-broadened widths are very similar.

The peak absorption coefficent in the lower atmosphere is approximately
given by $S/(\pi \times \gamma)$ 
where $\gamma$ is the linewidth computed from the self and foreign 
broadening.  Using the values in the previous table, we get the following
estimates for the optical depths in the lowest 200 meters of the atmosphere : 

\begin{longtable}{llllll} 
\hline
\hline
gas & total P & self P & $\gamma$   & peak abs   & peak optical \\
    & $atm$   & $atm$  & [$cm^{-1}$] & [$1/\eta$] & depth \\
\hline
\hline
water & 1.0 & 9.8E-3 & 8.8E-2 & 7.2E-19 & 4000 \\
CO2   & 1.0 & 3.4E-7 & 8.5E-2 & 1.5E-17 & 3000 \\
N2    & 1.0 & 8.4E-1 & 5.5E-2 & 2.3E-27 & 0.001\\
CH4   & 1.0 & 1.8E-6 & 6.4E-2 & 4.9E-19 & 0.5\\
\hline
\hline 
\end{longtable} 

One immediately sees that methane and nitrogen have much smaller peak
optical depths than water or carbon dioxide.  Most of the other gases
in the atmosphere also have maximum optical depths of less than one.
Remember that when the spectra are actually computed, there is a sum over 
the individual lines, so the peak values in the table above are slight 
underestimates. In addition, the above table is only for the lowest 200m
meters of the atmosphere; the total surface-to-space optical depths
are typically ~50 times larger.

Using typical values from the table above (which use parameters for the lowest
layer in the atmosphere), the absorption due to an individual line has 
decayed by 
$ \frac{\gamma}{25^{2}+\gamma^{2}}/\frac{\gamma}{\gamma^{2}}\simeq 10^{-5}$
by the time it is 25$cm^{-1}$ away from line center. For most gases
except H2O and CO2, this $25 cm{-1}$ distance may be used a safe cutoff
limit beyond which the contribution of an individual line to the overall
spectrum can be considered negligible.  The large optical depth of the
strong CO2 and water lines means we must go a couple hundred wavenumbers
into the wing before the optical depth drops off to a negligible level!


To speed up the code while maintaining accuracy, our algorithm uses a variant 
of the GENLN2 method of binning the lines while computing the spectra. To 
compute the absorption 
spectrum in the chosen interval, the code divides up the interval into 
bins of width $fstep \simeq 1 cm^{-1}$.  It then loops through these bins, 
computing the overall spectra in each bin using three stages. To reinforce
the ideas, actual example numbers will be used. Suppose the wavenumber
interval being considered is 1005 - 1030 $cm^{-1}$. There are 25 bins of
width $ffin = 1 cm^{-1}$ in this interval. Suppose we are considering the
4$th$ bin, which is the bin spanning 1008 to 1009 $cm^{-1}$.
\begin{itemize}
\item (a) {\it fine mesh stage} : lines that are $\pm xnear \simeq 1cm^{-1}$
          on either side of the edges of this bin, are used in computing 
          the overall lineshape at this stage, at a very fine resolution of 
          $ffin \simeq 0.0005 cm^{-1}$. The results of this computation 
          are then boxcar averaged to the output resolution $nbox \times 
          ffin \simeq 0.0025 cm^{-1}$ \\
          Thus the lines that are used in this stage have their centers 
          spanning 1007 to 1010 $cm^{-1}$.

\item (b) {\it medium mesh stage} : lines that are an additional 
          $xmed - xnear \simeq (2 - 1) = 1 cm^{-1}$ on either side of the 
          edges of this bin, are now used in computing 
          the overall lineshape, at a medium resolution of 
          $fmed \simeq 0.1 cm^{-1}$. The results of this computation 
          are then splined onto the output resolution and added onto the
          running sum from above.
          Thus the lines that are used in this stage have their centers 
          spanning (1006,1007) and (1010,1011) $cm^{-1}$.

\item (b) {\it coarse mesh stage} : lines that are an additional 
          $xfar - xmed \simeq (25 - 2) = 23 cm^{-1}$ on either side of the 
          edges of the medium meshes, are now used in computing 
          the overall lineshape, at a coarse resolution of 
          $fcor \simeq 0.5 cm^{-1}$. The results of this computation 
          are then splined onto the output resolution and added onto the
          running sum from above.
          Thus the lines that are used in this stage have their centers 
          spanning (983,1006) and (1011,1034) $cm^{-1}$.
\end{itemize}

The speed up in the code is gained by computing contributions at the 
medium and coarse resoutions as much as possible, instead of using the 
fine resolution grid all the time.

As was pointed out above, the strengths of some of the lines sometimes
sometimes make it necessary to use lines that could be upto 200 $cm^{-1}$ 
away (in the coarse part of the computation). Instead of doing this, a 
common practice is to remain with the 25 $cm^{-1}$ width of the coarse 
meshes, but include the effects of lines outside these coarse mesh by adding
on an extra {\it continuum}. This is true for oxygen and nitrogen, and in
particular water vapor (see below). Carbon dioxide also requires this; 
however we choose not to use the continuum in this case, but simply use the
entire 200 $cm^{-1}$ wide coarse meshes.

Carbon dioxide, like water vapor, is radiatively active in the Earth's 
atmosphere. Just like water, it is important to accurately compute the
absorption spectra using the correct lineshape, as optical depths of these
gases vary greately in the atmosphere, allowing one to probe several 
layers of the atmosphere within a small spectral region. A thorough 
understanding of the absorption spectra of these two gases is therefore 
very important for the remote sensing community.

\subsection{Computing water vapor absorption coefficients}
Almost all that has been said above remains valid when computing the 
lineshape of water vapor. However, the lineshape far away from line 
center is sub-Lorentz ($k/k_{lor} \leq 1$), while the lineshape close to 
line center is super Lorentz ($k/k_{lor} \geq 1$).  To account for this 
behavior, the above algorithm has to be slightly modified. This leads to 
the water vapor lineshape algorithm to include the following three 
different modifications, put together : 
\begin{itemize}
\item Instead of using a lorentz lineshape, a {\it local lineshape} is used.
      Upto 25 $cm^{-1}$ away from line cenetr, this is defined as 
      lorentz less the lorentz value \@ $25 cm^{-1}$, and zero everywhere 
      else.
\item To include the super Lorentz behavior close to the line center, 
      the local lineshape is then multiplied by a chi ($\chi$) function
\item To include the sub Lorentz behavior far from the line center, 
      a continuum function is then added on. This is done after the 
      effects of all lines have been used. Another way of thinking of this 
      continuum is to say that the water lines are quite strong, and thus
      the computational algorithm should not restricted to using only lines
      that are at most 25 $cm^{-1}$ wavenumbers away from the spectral 
      region under consideration. But instead of individually using lines 
      that could be upto 200 $cm^{-1}$ away (and modelling their
      sub Lorentz far wing behavior), the far wing effects of these far 
      lines are all lumped into the continuum.
\end{itemize}

To summarize, when computing a water vapor optical depth, the code proceeds
as in the general case described above, except that it uses a {\it local} 
lineshape for each line $j$, multiplied by a $\chi$ function. After all the 
necessary lines have been included, a {\it continuum} absorption coefficient
is also added on : 

\[
k(\nu) = k_{continuum}(\nu) + \sum_{j} k_{local}(\nu,\nu_{j}) \chi(\nu)
\]

which can be rewritten, for the individual lines $j$
\[
k_{local}(\nu,\nu_{j}) = 
\left\{
\begin{array}{cl}
     ( k_{lorentz}(\nu,\nu_{j}) - k_{lorentz}(\nu,25+\nu_{j}) ) 
           \chi(\nu)    & \mbox{if $|\Delta\nu| \leq 25 \mbox{cm}^{-1}$} \\
        0               & \mbox{if $|\Delta\nu| > 25 \mbox{cm}^{-1}$}
\end{array}
\right. 
\]
where appropriate factors of $\nu\tanh\left(\beta\nu/2\right)$ 
multiply the above coefficients. 

The computation for $k_{local}$ proceeds as described in the previous 
section, {\em viz.} using fine, medium and coarse meshes.

\subsection{Computing carbon dioxide absorption coefficients}

Computing the spectral lineshape of carbon dioxide can be quite complicated.
There are many bands within which there are lines that are very closely 
spaced. Collisions have the effect of mixing these lines together,
transferring intensity from the line wings to the line centers. Furthermore,
the  collisions are not instantaneous, but have a finite duration. This 
also makes the lineshape deviate from Lorentz, especially far from the line
centers. Additionally, some of the bands are very strong and have an 
effect on the absorption spectrum, at
quite large distances from their band (line) centers. This third effect can
be accounted for by either using a continuum, or by allowing the inclusion
of effects of lines that are upto 200 $cm^{-1}$ away from the region of 
interest.

We currently use full and dirst order line mixing methods of Strow $et$ $al$;
we are assessing using Hartmann linemixing coeffs.

\newpage
\section{run8}

The Matlab code has four main driver files : $run8.m,$ $run8co2.m$ and
$run8water.m$ and $run8watercontiunuum.m.$ 

$run8.m$ is a general code that will work for all gases; however 
one should run the specialized codes for water and CO2, so as to to utilise 
the above physics in the computed lineshapes. We describe the $run8.m$ 
program parameters and algorithm in detail below; in the next two sections, 
we will discuss the corresponding similarities and differences for the 
water and carbon dioxide codes.

\subsection{input units for run8}
\hl{The input argument list to all codes also includes the name of a profile file 
which specifies the layer number, total pressure, gas partial 
pressures (both in atm), gas temperature (in Kelvin) and gas amount 
(in $kilomoles cm^{-2}$). The profile should be in a 5 column format, 
and should be a text file.}

\subsection{Mex files and HITRAN database}
(this is for run8) \\
If the user wants to change the name of the line database file that is used,
he/she will have to go into the run8* files and change the name of the 
file in the line beginning with the word $fnamePRE$, which is currently 
set to :
\begin{verbatim}
fnamePRE='/salsify/scratch4/h98.by.gas/g';
\end{verbatim}

(this is for run7) \\
The default is to use HITRAN2000; if the user wants to change this, all that
has to be done is supply an additional input parameters, topt.HITTRAN = xxxx
(see below for details!!!)

To speed the code up, a number of loops have been written as fortran MEX 
files. All these files are in subdirectory FORTRANFILES, and assume input 
arrays/matrices that are smaller than certain limits. If the user wants to 
change these limits, he/she will have to edit the file $max.inc$ and 
recompile the Mex files.

\begin{verbatim}
c this is max length of arrays that can be used in the Mex Files  
c this number came out of 
c   200000 = max number of elements in mesh 
c              eg (755-655)/0.0005 = 160000
c        4 = number tacked on to arrays so boxint(y,5) can be done  
      integer MaxLen
      parameter(MaxLen=200010)

c assume max number of any of P,Q,R lines = 300
      integer MaxPQR
      parameter(MaxPQR=300)

c assume max number of any of layers = 100
      integer kMaxLayer
      parameter(kMaxLayer=100)
\end{verbatim}

To compile the Mex files, the user has to type $makemex1$ at the UNIX 
prompt (if only $run8/7.m$ is being used), or type $makemex$ 
(if $run8/7co2.m$ will be used). This compiles all the Mex files, and creates 
symbolic links to these files from the necessary subdirectories.

If the user is going to use $run8/7co2.m$, he/she will also need to go to the 
C02\_COMMON subdirectory, and type $link.sc$ so that symbolic links from the
CO2 subdirectories to the common files are created.

\subsection{Water, nitrogen, oxygen continuum}
Through parameter $CKD$ (see below), the user can toggle the continuum 
calculation on/off for three gases : water, oxygen and nitrogen 
(gasIDs 1,7,22 respectively). 

\begin{itemize}
\item Water : CKD can be set to -1 (no continuum), or 0,21,23,24 for CKD 
              versions 0, 2.1, 2.3, 2.4. Note when the first 3 versions of CKD 
              are included, the computation proceeds by using the code which
              does not require a ``local'' lineshape. However, CKD 2.4 does
              require a ``local'' lineshape.
\item Oxygen : CKD can be set to -1 (no continuum), or +1 (continuum)
\item Nitrogen : CKD can be set to -1 (no continuum), or +1 (continuum)
\item For all other gases, the value of $CKD$ is irrelevant
\end{itemize}

\subsection{run8.m input parameters}

A typical call to run8 would involve sending in the following : 

$[outwave,outarray]=run8(gasID,fmin,fmax,ffin,fmed,fcor,fstep$\\
         $xnear,xmed,xfar,nbox,strfar,strnear,LVG,CKD,profile)$

where the right hand side variables would be 

\begin{longtable}{llll}
\hline
\hline
  TYPE  &   VAR  &         DESCRIPTION  &            TYPICAL VALUE\\
\hline
\hline
integer & gasID  &       HITRAN gas ID      &            3\\
\hline
integer & fmin    &      minimum freq (cm-1) &          605\\
integer & fmax    &      maximum freq (cm-1) &          630\\
\hline
real   &  ffin    &      fine point spacing (cm-1) &    0.0005\\
real   &  fmed    &      medium point spacing (cm-1)&   0.1\\
real   &  fcor    &      coarse point spacing (cm-1)  & 0.5\\
\hline
real   &  fstep   &      wide mesh width size (cm-1) &    1.0\\
real   &  xnear   &      near wing distance(cm-1)    &    1.0\\
real   &  xmed    &      med wing distance(cm-1)     &    2.0\\
real   &  xfar    &      far wing distance(cm-1)     &    25.0\\
\hline
integer & nbox     &     boxcar sum size (odd integer) &  1,5\\
\hline
real   &  strfar   &    min line strength for far wing lines & \\
real   &  strnear  &    min line strength for near wing lines& \\
\hline
char   &  LVG       &    (L)orentz,Voi(G)t,(V)anHuber    &  'V'\\
\hline
integer&  CKD       &    continuum no  = -1 (most gases) &  -1\\
       &            &    yes for water (0,21,23,24)      &  \\
       &            &    yes for N2, O2 (gases 7,22)     &  \\
\hline
matrix & profname   & Nx5 matrix (gasID,pressure,pp,temp,amt)   & \\
       &            & pressure, partial pressure in atm, T in K & \\
       &            & gas amount in kilomolecules/cm2           & \\
\hline
\hline
\end{longtable}

The HITRAN gasIDs and cross sections can be found at 
http://www.cfa.harvard.edu/hitran/
\begin{figure}[h]
  \begin{center}\includegraphics[width=6in]{Figures/HITRAN_gasIDs}\end{center}
  \caption{HITRAN partioal listing of GAS IDs}
  \label{fig:hitran}
\end{figure}

The output arguments from the function call are the output wavevector, 
outwave, and the computed line spectra in outarray. The vector outwave (and 
thus the output array outarray) spans the wavenumber range from $fmin$ to 
$fmax-(ffin \times  nbox)$, at a resolution of $ffin \times nbox$.

\subsection{run7.m input parameters}

The same parameters as above are used. However, many of them are now 
defaulted to preset values, and can be reset by using optional input argument
$topts$, where $topts$ is a structure.

A typical call to run7 would involve sending in the following : 

$[outwave,outarray]=run7(gasID,fmin,fmax,profile,\{topts\})$;

where the required right hand side variables would be 

\begin{longtable}{llll}
\hline
\hline
  TYPE  &   REQUIRED &         DESCRIPTION  &            TYPICAL VALUE\\
\hline
\hline
integer & gasID  &       HITRAN gas ID      &            3\\
\hline
integer & fmin    &      minimum freq (cm-1) &          605\\
integer & fmax    &      maximum freq (cm-1) &          630\\
\hline
matrix & profname   & Nx5 matrix (gasID,pressure,pp,temp,amt)   & \\
       &            & pressure, partial pressure in atm, T in K & \\
       &            & gas amount in kilomolecules/cm2           & \\
\hline
\hline
\end{longtable}

and the optional right hand side arguments would be sent in structure $topts$;
the default values are as shown : 
\begin{longtable}{llll}
\hline
\hline
  TYPE  &   OPTIONAL &         DESCRIPTION  &            DEFAULT VALUE\\
\hline
\hline
real   &  ffin    &      fine point spacing (cm-1)    &   0.0005\\
real   &  fmed    &      medium point spacing (cm-1)  &   0.1\\
real   &  fcor    &      coarse point spacing (cm-1)  &   0.5\\
\hline
real   &  fstep   &      wide mesh width size (cm-1) &    1.0\\
real   &  xnear   &      near wing distance(cm-1)    &    1.0\\
real   &  xmed    &      med wing distance(cm-1)     &    2.0\\
real   &  xfar    &      far wing distance(cm-1)     &    25.0\\
\hline
integer & nbox     &     boxcar sum size (odd integer) &  5\\
\hline
real   &  strfar   &    min line strength for far wing lines  & 0.0\\
real   &  strnear  &    min line strength for near wing lines & 0.0\\
\hline
char   &  LVG       &    lineshape : (L)orentz,Voi(G)t,(V)anHuber  &  'V'\\
string & HITRAN   &   path to HITRAN database  & /asl/data/hitran/h2k.by.gas \\
integer&  CKD       &    continuum no  = -1 (most gases) &  -1\\
       &            &    yes for water (0,21,23,24)      &  \\
       &            &    yes for N2, O2 (gases 7,22)     &  \\
\hline
\hline
\end{longtable}

The output arguments from the function call are the output wavevector, 
outwave, and the computed line spectra in outarray. The vector outwave (and 
thus the output array outarray) spans the wavenumber range from $fmin$ to 
$fmax-(ffin \times  nbox)$, at a resolution of $ffin \times nbox$.

\subsection{Detailed description of the input parameters}

Two of the input parameters are self-describing. The first parameter, 
$gasID$ is an integer value specifying which gas you want to compute the 
line spectra for. This integer value is the same as that used for the 
HITRAN database; for example gasID=3 corresponds to ozone. $LVG$ is a
character parameter that tells the code which lineshape to use for $all$ 
the lines. Of the lineshapes described previously, our code can compute 
one of the following three - \textbf{L}orentz, Voi\textbf{G}t or 
\textbf{V}anVleck-Huber. The 
VanVleck-Huber is computed with a Voigt lineshape, and is the one we 
recommend; to use this lineshape, $LVG$ is set to 'V'.

When the code starts running, it loads in all lines whose centers lie 
between $fmin - xfar$ and $fmax + xfar$, and whose database line strength 
is greater than min(strfar,strnear), as the user assumes these are the 
lines which will have a discernible effect on the overall spectra. Using 
these lines, and their associated parameters, the computations are 
performed on a fine mesh resolution $ffin$ and then boxcar
averaged to an output resolution $ffin \times nbox$. The results of the 
computations are output for a wavector that spans $fmin$ to $fmax-ffin 
\times nbox$. Internally, the computations are essentially performed on a 
fine mesh that spans $fmin - (nbox-1)/2 $ to $fmax-ffin \times nbox + 
(nbox-1)/2$. In this way, the boxcar averaging can be done on the endpoints.

If this direct method were used, depending in the gasID and wavenumber 
region chosen, the code could be agonizingly slow. In the interests of 
speed (and mantaining the accuracy), the code therefore requires some more 
parameters to be sent in. 

With these additional parameters, the output wavevector $fmin$ to 
$fmax-ffin \times nbox$ is divided into equal sized ``wide meshes'' of 
size $fstep$ $cm^{-1}$. Thus there are $N$ wide meshes, where
\begin{equation}
N = \frac{fmax-fmin}{fstep}
\end{equation}

Suppose we are considering the $i$th  widemesh, and we denote the start 
frequency  of this widemesh by $f1$, and the stop frequency by $f2$. These 
two numbers are related to each other and to the other numbers by 
\begin{displaymath}
f1 = fmin + (i-1) \times fstep - ffin \times (nbox-1)/2 
\end{displaymath}
\begin{equation}
f2 = fmin + ii \times fstep - (ffin \times nbox) + ffin \times (nbox-1)/2
\end{equation}
This ``finemesh'' thus spans $(f1,f2)$ at the fine resolution of 
$ffin$ $cm^{-1}$

Associated with this finemesh is a medium resolution mesh, that spans 
$(f3,f4)$ at a coarser resolution $fmed$ where
\begin{displaymath}
f3 = fmin + (i-1) \times fstep\\
\end{displaymath}
\begin{equation}
f4 = fmin + ii \times fstep\\
\end{equation}

In addition there is a coarse resolution mesh, that spans 
$(f3,f4)$ at a coarsest resolution $fcor$ where $f3,f5$ are the same, as are
$f4,f6$ : 
\begin{displaymath}
f5 = fmin + (i-1) \times fstep\\
\end{displaymath}
\begin{equation}
f6 = fmin + ii \times fstep\\
\end{equation}

The spectral region of the output wavevector that corresponds to these 
three meshes is essentially $f3,f4$, adjusted for the last point. In other 
words, the
$i$th output region $fout(i)$ spans $f3$ to $f4-nbox \times ffin$, at a 
resolution of $nbox \times ffin$

For any of the $N$ widemeshes, lines are grouped into three categories, 
depending where they fall within the three categories defined below : \\
(a) near lines are those whose line centers lie in the wavenumber interval
$(f3-xnear,f4+xnear)=(w1,w2)$. All computations using these lines are 
performed on the fine mesh (spanning $f1,f2$) of point spacing ffin, 
and then boxcar averaged to the output wavevector $fout(i)$ \\
(b) medium lines are those whose line centers lie in the wavenumber interval
$(f3-xnear-xmed,f3-xnear) \cup (f4+xnear,f4+xnear+xmed)=(w3,w1) \cup (w2,w4)$. 
All computations using these lines are performed on the medium mesh 
(spanning $f3,f4$) of 
point spacing fmed, and then splined to the output wavevector $fout(i)$\\
(c) far lines are those whose line centers lie in the wavenumber interval
$(f3-xmed,f3-xfar)\cup(f4+xmed,f4+xfar)=(w5,w3)\cup (w4,w6)$. All 
computations using these lines are performed on the coarse mesh 
(spanning $f3,f4$) of 
point spacing fcor, and then splined to the output wavevector $fout(i)$\\
The cartoon in Figure \ref{fig:cartopon_param} summarizes the above 
relationships.

\begin{figure}
  \begin{center}\includegraphics[width=6in]{Figures/fig1}\end{center}
  \caption[Cartoon of Parameter Relations]{}
  \label{fig:cartoon_param}
\end{figure}

With the above description, the following restrictions on the parameters are
now self explanatory : \\
(1) xnear $\le$ xmed $\le$ xfar        \\
(2) xnear $\ge$ fstep              \\
(3) xmed/fmed  xnear/fmin   fstep/fmed   fstep/ffin       are integers \\
(4)fstep/(nbox*ffin)        fcor/ffin                     are integers \\
(5)(fmax-fmin)/fstep        (fmax-fmin)/fcor              are integers \\

A useful rule of thumb is that ffin,fmed,fcor are chosen so that they 
are all equal to $(1/2)/10^{n}$   $n \le 5$, with $n$ chosen as necesary for 
the three parameters. For example, $n=3,1,0$ gives $ffin=0.0005,fmed=0.05,
fcor=0.5 cm^{-1}$.

The above algorithm used is almost the same as that used by GENLN2, except 
that GENLN2 does not currently have the medium resoltion mesh ie the overall 
lineshape is a sum of boxcar averaged fine mesh contribution and a spline
computed coarse mesh contribution.

Other differences found between this LBL code and GENLN2 is that all 
computations here are in real*8, while GENLN2 mixes bewteen real*8 and 
real*4. 
The discrepancies between these two representations is noticeable in 
computations of eg the partition fucntions. In addition, we believe that the
contribution of a line whose center is in the ``far line'' regime, is 
incorrectly splined at the last interval $|x_{center} - x| \sim xfar$

\subsection{Detailed description of the output parameters}
Two parameters are passed out after running the code : a 1d array $outwave$ 
that contains the output wavevector, and a 2d matrix $outarray$ that 
contains the computed lineshapes at the user set atmospheric levels.

\subsection{Outline of the algorithm}

The code starts out by checking to ensure that the input parameters make 
sense and that they are self consistent. For example, parameter LVG must 
be set to
one of the allowable line shapes. In addition, the parameters should all be 
self consistent in that they have to staisfy the restrictions given at the 
end of the previous subsection.

Having ascertained the self consistency of the parameters sent in by the 
user, the program loads in the required mass isotopes for the chosen 
gas. For example ozone has 5 isotopes. 

The program then loads in the user specified profile for the gas.
Having done this, the program then uses $fmin,fmax,ffin,nbox$ to define the 
output wave vector. After this, the gas initializes the $qtipts$ 
coefficients that are used to compute the partition functions. This is 
essentially a GENLN2 subroutine, similar to the program ``tips'' by 
R.R.Gamache.

The program is now ready to load in the gas line parameters from the HITRAN
database. As described above, it loads in all lines whose centers lie 
between $fmin - xfar$ and $fmax + xfar$, and whose database line strength 
is greater than min(strfar,strnear). 

At this point, the program is almost ready to start running in earnest. 
Before doing that, it computes the number of wide meshes $N$ and the number 
of points in each wide mesh that will be mapped to the output wavevector.
If the user loaded in a profile that has more than one layer in it, 
the program calls subroutine $doUnion2$, that
computes the optical depth of each linecenter for the chosen profile 
conditions; if a line is strong enough to be used in any $one$ of the 
levels, it will be used at $all$ levels. This will ensure that the optical 
depth profiles are smooth. An example of this is the case of ozone, where 
lines could ``turn-on'' high in the atmosphere, but have almost no optical 
depth lower in the atmosphere. The importance of this is when the output 
from the code is used to generate the kCARTA database using Singular Value 
Decomposition; the SVD algorithm would work more efficiently with smoothly 
varying data (achieve better compression).

Figure \ref{fig:init_alg} outlines the above initialisation stages of the 
algorithm.

\begin{figure}
  \begin{center}\includegraphics[width=6in]{Figures/fig2}\end{center}
  \caption[Outline of initialization algorithm]{}
  \label{fig:init_alg}
\end{figure}

The program is now ready to loop over the far,medium and near lines. 
For each of the wide meshes, the program first defines the fine, medium and
coarse meshes (frequencies and indices), as described in the previous 
section. It then sorts all the lines it has loaded into three bins; near,
medium and far, also as described in the previous section.

It then enters a loop over layers. For the current layer, the program uses 
the gas profile to determine the gas amount, temperature, total and self 
pressures. For each layer, it first computes the contribution due to the 
near lines, then the medium lines and finally the far lines. The near line 
spectrum is computed
on the fine mesh, and the results are boxcar averaged and added onto the
output array. The medium line spectrum is computed on the medium mesh, 
spline interpolated onto the output wavevector and added on to the output 
array.The far line spectrum is computed on the coarse mesh, spline 
interpolated onto the output wavevector and added on to the output array.

For each of the fine,medium and coarse computations, the code computes
the following line parameters, for each of the lines\\
(a) the partition function, using $qfcn=q(A,B,C,D,G,lines,tempr)$\\
(b) the line center frequency, taking the pressure of the current layer into
    account $freq=lines.ZWNUM+press(jj)*lines.ZTSP$\\
(c) the overall broadening of the line, using the self and foreign 
    broadening
    contributions $brd=broad(p,ps,1.0,forbrd,selfbrd,pwr,tempr,gasID)$\\
(d) the line center line strength, using the necessary layer parameters 
    such as temperature, gas amount and necessary line parameters \\
    $strength=find_stren(qfcn,freq,tempr,energy,s0,GasAmt(jj))$\\
The above computations are essentially GENLN2 routines.

\begin{figure}
  \begin{center}\includegraphics[width=6in]{Figures/fig3}\end{center}
  \caption[Outline of loops over layers and fine,medium,coarse meshes]{}
  \label{fig:loop_alg}
\end{figure}

Figure \ref{fig:loop_alg} outlines the loop stage of the algorithm.

\section{run8water}

run8water.m is a specialised code for H2O, so as to to utilise the above 
physics, namely local lineshape and the CKD continuum effects in the 
computed 
lineshapes. If the user simply wants to do a Lorentz or Voigt computation,
then it would behoove him/her to use $run8.m$ instead of this special code.

\subsection{run8water.m input parameters}

A typical call to run8water would involve sending in the following : 

$[outwave,outarray]=run8water(gasID,fmin,fmax,ffin,fmed,fcor,$\\
              $fstep,xnear,xmed,xfar,nbox,strfar,strnear,LVF,$\\
              $CKD,selfmult,formult,usetoth,local,profname);$

where the right hand side variables are the same as those for run8 
described above; there are 5 new variables on the right side.

\begin{longtable}{llll}
\hline
\hline
  TYPE  &   VAR  &         DESCRIPTION  &            TYPICAL VALUE\\
\hline
\hline
integer & gasID  &       HITRAN gas ID      &            1\\
\hline

integer & fmin    &      minimum freq (cm-1) &          705\\
integer & fmax    &      maximum freq (cm-1) &          730\\
\hline

real   &  ffin    &      fine point spacing (cm-1) &    0.0005\\
real   &  fmed    &      medium point spacing (cm-1)&   0.1\\
real   &  fcor    &      coarse point spacing (cm-1)  & 0.5\\
\hline

real   &  fstep   &      wide mesh width size (cm-1) &    1.0\\
real   &  xnear   &      near wing distance(cm-1)    &    1.0\\
real   &  xmed    &      med wing distance(cm-1)     &    2.0\\
real   &  xfar    &      far wing distance(cm-1)     &    150.0\\
\hline

integer & nbox     &     boxcar sum size (odd integer) &  1,5\\
\hline

real   &  strfar   &    min line strength for far wing lines & \\
real   &  strnear  &    min line strength for near wing lines& \\
\hline

char   &  LVG      &      (L)orentz,Voi(G)t,(V)anHuber  &  'V' \\
integer &  CKD     &       continumm no (-1)            &   -1 \\
        &          &       yes water : (0,21,23,24)    & \\
\hline

real    &  selfmult &       multiplier for self part of contiuum &  0<x<1 \\
        &  formult  &       multiplier for for  part of contiuum  & 0<x<1 \\
\hline

integer & usetoth &        use Toth or HITRAN &           +1 to use Toth \\
        &         &                           &          -1 to use HITRAN \\
\hline

integer & local &         use local lineshape   & +1 to use local*chi defn\\
        &       &                                  &  0 to use local defn \\
        &       &                                  & -1 to use run8 defn\\
\hline
\hline
\end{longtable}

The output arguments from the function call are once again the output 
wavevector,  outwave, and the computed line spectra in outarray. The vector 
outwave (and thus the output array outarray) spans the wavenumber range 
from $fmin$ to $fmax-(ffin \times  nbox)$, at a resolution of 
$ffin \times nbox$.

\subsection{run7water.m input parameters}

The same parameters as above are used. However, many of them are now 
defaulted to preset values, and can be reset by using optional input argument
$topts$, where $topts$ is a structure.

A typical call to run7 would involve sending in the following : 

$[outwave,outarray]=run7water(gasID,fmin,fmax,profile,\{topts\})$;

where the required right hand side variables would be 

\begin{longtable}{llll}
\hline
\hline
  TYPE  &   REQUIRED &         DESCRIPTION  &            TYPICAL VALUE\\
\hline
\hline
integer & gasID  &       HITRAN gas ID      &            1\\
\hline
integer & fmin    &      minimum freq (cm-1) &          605\\
integer & fmax    &      maximum freq (cm-1) &          630\\
\hline
matrix & profname   & Nx5 matrix (gasID,pressure,pp,temp,amt)   & \\
       &            & pressure, partial pressure in atm, T in K & \\
       &            & gas amount in kilomolecules/cm2           & \\
\hline
\hline
\end{longtable}

and the optional right hand side arguments would be sent in structure $topts$;
the default values are as shown (notice that $CKD$ is defaulted to -1 (OFF)
and the lineshape is augmented to ``local'') : 
\begin{longtable}{llll}
\hline
\hline
  TYPE  &   OPTIONAL &         DESCRIPTION  &            DEFAULT VALUE\\
\hline
\hline
real   &  ffin    &      fine point spacing (cm-1)    &   0.0005\\
real   &  fmed    &      medium point spacing (cm-1)  &   0.1\\
real   &  fcor    &      coarse point spacing (cm-1)  &   0.5\\
\hline
real   &  fstep   &      wide mesh width size (cm-1) &    1.0\\
real   &  xnear   &      near wing distance(cm-1)    &    1.0\\
real   &  xmed    &      med wing distance(cm-1)     &    2.0\\
real   &  xfar    &      far wing distance(cm-1)     &    25.0\\
\hline
integer & nbox     &     boxcar sum size (odd integer) &  5\\
\hline
real   &  strfar   &    min line strength for far wing lines  & 0.0\\
real   &  strnear  &    min line strength for near wing lines & 0.0\\
\hline
char   &  LVG       &    lineshape : (L)orentz,Voi(G)t,(V)anHuber  &  'V'\\
string & HITRAN   &   path to HITRAN database  & /asl/data/hitran/h2k.by.gas \\
integer&  CKD       &    continuum no  = -1 (most gases) &  -1\\
       &            &    yes for water (0,21,23,24)      &  \\
       &            &    yes for N2, O2 (gases 7,22)     &  \\
\hline
real    &  selfmult &       multiplier for self contiuum     0< x< 1& 1\\
real    &  formult  &       multiplier for foreign continuum 0< x< 1& 1\\
\hline
integer & local &         modification to LVG lineshape            &  0 \\
        &       &         +1 to use local*chi lineshape defn       &    \\
        &       &          0 to use local lineshape defn           &    \\
        &       &         -1 to use standard LVG lineshape defn    &    \\
\hline
\hline
\end{longtable}

\subsection{Detailed description of the input parameters}

As mentioned above, most of the input parameters are the same as for $run8$
and a description is not repeated here. However, five of the last six 
parameters are new, and so will be explained below.

$CKD$ is a integer parameter that tells the code which continuum to use. 
Note that based on whether or not the ``local'' lineshape was used, the 
appropriate CKD lookup tables are used. For CKD 0,21,23 the code can 
compute the continuum whether or not the local lineshape was used; for 
CKD24, only the local lineshape can be used.

$selfmult$ is a real parameter between 0 and 1, that is used to scale the
``self'' contribution to the continuum.

$formult$ is a real parameter between 0 and 1, that is used to scale the
``foreign'' contribution to the continuum.

$usetoth$ is a integer parameter that tells the code whether or not to use 
the Toth database.

$local$ is a integer parameter that tells the code whether or not to compute
the local lineshape (must be set to ``0'' or ``1'' to use CKD2.4)

\section{run8watercontinuum}

run8watercontinuum.m is a specialised code for H2O, that only computes the
CKD continuum. This code is to be used in conjunction with run8water.m

\subsection{run8watercontinuum.m input parameters}

A typical call to run8watercontinuum would involve sending in the following 
(note that the 'LVG'' parameter has been replaced by ``divide'') : 

$[outwave,outarray]=run8water(gasID,fmin,fmax,ffin,fmed,fcor,$\\
              $fstep,xnear,xmed,xfar,nbox,strfar,strnear,divide,$\\
              $CKD,selfmult,formult,usetoth,local,profname);$

where the right hand side variables are the same as those for run8 
described above; there are 5 new variables on the right side.

\begin{longtable}{llll}
\hline
\hline
  TYPE  &   VAR  &         DESCRIPTION  &            TYPICAL VALUE\\
\hline
\hline
integer & gasID  &       HITRAN gas ID      &            1\\
\hline

integer & fmin    &      minimum freq (cm-1) &          705\\
integer & fmax    &      maximum freq (cm-1) &          730\\
\hline

real   &  ffin    &      fine point spacing (cm-1) &    0.0005\\
real   &  fmed    &      medium point spacing (cm-1)&   0.1\\
real   &  fcor    &      coarse point spacing (cm-1)  & 0.5\\
\hline

real   &  fstep   &      wide mesh width size (cm-1) &    1.0\\
real   &  xnear   &      near wing distance(cm-1)    &    1.0\\
real   &  xmed    &      med wing distance(cm-1)     &    2.0\\
real   &  xfar    &      far wing distance(cm-1)     &    150.0\\
\hline

integer & nbox     &     boxcar sum size (odd integer) &  1,5\\
\hline

real   &  strfar   &    min line strength for far wing lines & \\
real   &  strnear  &    min line strength for near wing lines& \\
\hline
integer &  divide  &      what is overall computation    &   -1 \\
integer &  CKD     &       continumm no (-1)            &   24 \\
        &          &       yes water : (0,21,23,24)     &      \\
\hline

real    &  selfmult &       mult for self part of contiuum & 0< x< 1 \\
        &  formult  &       mult for for  part of contiuum & 0< x< 1 \\
\hline

integer & usetoth &        use Toth or HITRAN &           +1 to use Toth \\
        &         &                           &          -1 to use HITRAN \\
\hline

integer & local &         use local lineshape   & +1 to use local*chi defn\\
        &       &                                  &  0 to use local defn \\
        &       &                                  & -1 to use run8 defn\\
\hline
\hline
\end{longtable}

The output arguments from the function call are once again the output 
wavevector,  outwave, and the computed line spectra in outarray. The vector 
outwave (and thus the output array outarray) spans the wavenumber range 
from $fmin$ to $fmax-(ffin \times  nbox)$, at a resolution of 
$ffin \times nbox$.

\subsection{run7watercontinuum.m input parameters}

The same parameters as above are used. However, many of them are now 
defaulted to preset values, and can be reset by using optional input argument
$topts$, where $topts$ is a structure.

A typical call to run7 would involve sending in the following : 

$[outwave,outarray]=run7watercontinuum(gasID,fmin,fmax,profile,\{topts\})$;

where the required right hand side variables would be 

\begin{longtable}{llll}
\hline
\hline
  TYPE  &   REQUIRED &         DESCRIPTION  &            TYPICAL VALUE\\
\hline
\hline
integer & gasID  &       HITRAN gas ID      &            1\\
\hline
integer & fmin    &      minimum freq (cm-1) &          605\\
integer & fmax    &      maximum freq (cm-1) &          630\\
\hline
matrix & profname   & Nx5 matrix (gasID,pressure,pp,temp,amt)   & \\
       &            & pressure, partial pressure in atm, T in K & \\
       &            & gas amount in kilomolecules/cm2           & \\
\hline
\hline
\end{longtable}

and the optional right hand side arguments would be sent in structure $topts$;
the default values are as shown (as is readily appreciated, most of the run7
input arguments are unnecessary and have been REMOVED; also sice the continuum
is smooth, we do not really need 5 point averaging and so nbox is defaulted to
+1 instead of +5) : 
\begin{longtable}{llll}
\hline
\hline
  TYPE  &   OPTIONAL &         DESCRIPTION  &            DEFAULT VALUE\\
\hline
\hline
real   &  ffin    &      fine point spacing (cm-1)    &   0.0025\\
integer & nbox     &     boxcar sum size (odd integer) &  1\\
\hline
integer&  CKD       &    continuum no  = -1 (most gases) &  -1\\
       &            &    yes for water (0,21,23,24)      &  \\
\hline
integer &  divide  &      what is overall computation    &   -1 \\
real    &  selfmult &       mult for self contiuum      0 < x < 1 & 1\\
real    &  formult  &       mult for foreign continuum  0 < x < 1 & 1\\
\hline
integer & local &         modification to LVG lineshape            &  0 \\
        &       &         +1 to use local*chi lineshape defn       &    \\
        &       &          0 to use local lineshape defn           &    \\
        &       &         -1 to use standard LVG lineshape defn    &    \\
\hline
\hline
\end{longtable}

\subsection{Detailed description of the input parameters}

$ffin\times nbox$ determines the output spacing

$CKD$ is a integer parameter that tells the code which continuum to use. 
Note that based on whether or not the ``local'' lineshape was used, the 
appropriate CKD lookup tables are used. For CKD 0,21,23 the code can 
compute the continuum whether or not the local lineshape was used; for 
CKD24, only the local lineshape can be used.

$selfmult$ is a real parameter between 0 and 1, that is used to scale the
``self'' contribution to the continuum.

$formult$ is a real parameter between 0 and 1, that is used to scale the
``foreign'' contribution to the continuum.

$local$ is a integer parameter that tells the code whether or not to compute
the local lineshape (must be set to ``0'' or ``1'' to use CKD2.4)

$divide$ is an IMPORTANT parameter that enables the user to ouput the correct
total continuum, or along with $selfmult, formult$, just parts of it (such as 
self or foreign or combination). The effects are described as below
\begin{longtable}{lllll}
\hline
\hline
  DIVIDE  &   SELFMULT & FORMULT &  DIVIDES BY & RESULT \\
\hline
\hline
 -1 & xx  & xx  &  1.0 & q v tanh(c2 v/2T) (296/T) $\times$ \\
    &     &     &      & (ps CS + pf CF) \\
 +1 & 1.0 & 0.0 &  q v tanh(c2 v/2T) (296/T) * ps     & CS  \\
 +1 & 0.0 & 1.0 &  q v tanh(c2 v/2T) (296/T) * (p-ps) & CF  \\
 +1 & xx  & xx  &  q v tanh(c2 v/2T) (296/T)          & (ps CS + pf CF) \\
\hline
\hline
\end{longtable}

\section{run8co2}
See lbl.pdf for this!
\newpage

\section{General Spectral Lineshape Theory}\label{chpt:shapes}

This section examines the basic lineshape parameters,
including line centers, shifts, widths, and strengths.  The standard 
lineshapes for natural and Doppler broadening are then presented, followed 
by a review of collisional lineshapes. Most of this information 
is from Dave Tobin's PhD dissertation \cite{tob:96}.

\subsection{Molecular Absorption and Beer's Law}\label{sec:beerslaw}

Molecular absorption occurs when a molecule absorbs light and 
simultaneously makes a transition to a higher level of internal 
energy.  The absorption of incident photons decreases the outgoing 
radiation, and a spectral line is produced.  The shape or frequency 
dependence of this absorption is often called the lineshape.  Because 
of various factors, the absorption occurs not only at the resonant 
frequency of the transition (determined by the difference between the 
upper and lower energy levels), but over a spread of frequencies.  
These broadening factors lead to a finite width of the spectral 
line.  While the resonant frequency and the intensity of the 
absorption are determined primarily by the structure of the molecule, the 
lineshape is determined by the molecules' environment.  

The frequency dependence of the absorption coefficient, $k(\nu)$, 
determines the shape of a spectral line.  Beer's law relates the 
absorption of radiation through a gaseous medium linearly to the incident 
radiation (see Figure \ref{fig:beers_law}):
\begin{equation}
   -dI = k(\nu)I_{0}P dl
\label{eqn:blaw1}
\end{equation}
with the absorption coefficient, $k(\nu)$ being the constant of
proportionality.  $-dI$ is the decrease in radiation flux over a path 
length of $dl$ through a gas of constant and uniform pressure $P$.  
Integrating this equation over a homogeneous path length of $L$ yields 
the integrated form of Beer's law:
\begin{equation}
 T(\nu)=\frac{I_f(\nu)}{I_0(\nu)}=\exp\left(-k(\nu)P L\right).
\label{eqn:blaw2}
\end{equation}
$T(\nu)$ is is the transmission at frequency $\nu$. $I_{0}(\nu)$ and
$I_f(\nu)$ are the initial and final radiation intensities.  Thus, the 
absorption coefficient is related to the observed transmission by 
\begin{equation}
 k(\nu)=-\frac{1}{P L}\ln\left(T(\nu)\right).
\end{equation}

It should be noted that deviations from the linear form of Beer's law are
only observed at extremely high photon densities.  Under atmospheric
conditions, however, the linear dependence of the extinction on the amount 
of absorbing material and incident radiation is valid. 

\begin{figure}
 \begin{center}
  \includegraphics[width=6in]{Figures/beerslaw}\end{center}
  \caption[An illustration of Beer's law.]{A gas cell of pressure $P$ and 
    length $L$ with incident radiation,
    $I_{0}$, from the left.  The amount of absorption at frequency $\nu$ is
    determined by the magnitude of the absorption coefficient, $k(\nu)$,
    the gas pressure, and the path length according to Beer's law.}
  \label{fig:beers_law}
\end{figure}

\subsection{Line Parameters}\label{sec:lines}

The lineshape of a single (non-interacting) transition is commonly 
characterized by several parameters including the line {\it center} 
($\nu_0$), 
line {\it strength} ($S$), and line {\it width} ($\gamma$).  These are
illustrated in Figure \ref{fig:spectralline}.
\begin{figure}
  \begin{center}
  \includegraphics[width=6in]{Figures/spectralline}
  \end{center}
  \caption[A spectral line depicting the line center and line width.]
	{A spectral line depicting the line center, $\nu_0$, and half-width,
	$\gamma$.  The line strength, $S$, is the absorption coefficient,
	$k(\nu)$, integrated over all wavenumbers $\nu$.}
  \label{fig:spectralline}
\end{figure}

Several spectral line databases are available which provide a compilation
of the line positions, strengths, and widths as well as several other
important parameters such as the lower state energy, pressure induced line
center shifts, isotopic abundances, rotational and vibrational quantum
indexing, width-temperature exponents, and transition probabilities.
HITRAN\cite{rot:87,rot:92} 
(the high resolution transmission molecular absorption database) 
is one such database used in this work which is maintained by the Phillips 
Laboratory Geophysics Directorate .  It currently lists the
parameters of over 700,000 rotation and vibration-rotation spectral lines
for 31 molecules of atmospheric importance from 0--23,000 cm$^{-1}$.
This database represents the most accurate compilation of line parameters.
However, due to its size, it is only updated every two to four years and 
thus recent
state-of-the-art measurements and calculations are not always in the
database and must be obtained elsewhere.

\subsubsection{Line Centers}

The line {\it center}, or position, of a spectral line is determined by the
molecular structure, just as the allowable vibrational-rotational energy
levels of the molecule are determined by its structure.  Planck's relation: 
\begin{equation}
\nu_{0}=\frac{\Delta E}{h}
\end{equation}
relates the transition frequency, $\nu_{0}$ (cm$^{-1}$), to the 
change in internal energy, $\Delta E$, where $h$ is Planck's constant.  
The line centers are thus determined by the structure and allowed energy
levels of the molecule and by transition selection rules. 

Since the line centers do not depend critically 
upon the interactions with other molecules, or upon the population of 
various states, they do not vary significantly with temperature or 
pressure. One common exception to this are the very small shifts in line 
center with increasing pressure.  Just as molecular collisions can 
disturb optical transitions leading to increased line widths 
(discussed later), these disturbances can also lead to an apparent change 
in the resonant frequency of the molecule's wave-train.  Computationally, 
the shifted position is given by
\begin{equation}
\nu_{0}(P)=\nu_{0}+P\cdot\delta_{\nu}
\end{equation}
where P is the total pressure and $\delta_{\nu}$ is the 
pressure induced frequency shift and is determined either 
theoretically or experimentally.  A more theoretical explanation of line
shifts which are due to distant collisions was first given by
Lenz and more recently by Breene\cite{bre:56}.

\subsubsection{Line Strengths}

The line {\it strength} or line intensity is a direct measure of the 
ability of a molecule to absorb photons corresponding to a given transition.
It depends upon both the properties of the single molecule and the relative
number of molecules in the upper and lower states.  The strength, $S$, is 
defined as
\begin{equation}
S=\int k(\nu_{0},\nu)d\nu
\label{eqn:strength1}
\end{equation}
where the integral is over all $\nu$.  Experimentally, $S$ can be
determined using Equation \ref{eqn:strength1} if $k(\nu_0,\nu)$ is
measured.  Alternatively, if the functional form of $k(\nu_0,\nu)$ is
known,  regression techniques can be used to determine $S$.  

By far the strongest interaction between matter and an incident field of 
electromagnetic radiation involves the molecule's electric dipole moment.  
The intensity of a dipole transition is proportional to the square of the 
matrix element of the dipole moment operator $M$:
\begin{equation}
R_{i,j}=\int \Psi_i^{*}M\Psi_j dV
\end{equation}
where $dV$ is a volume element in configuration space and the integral 
is over all space.  $\Psi_i$ and $\Psi_j$ are the wavefunctions 
of the lower and upper levels of the transition.  The 
wavefunctions are orthogonal and therefore, if $M$ is unchanged during the
transition, $R=0$.  Consequently, for a dipole transition to occur, the 
electric dipole moment must change between the initial and final energy 
levels 
of a transition.  Otherwise, the molecule is not linked to the 
incident radiation and no absorption occurs.  Weaker transitions can occur,
however, for quadrupole transitions even if there is no change in the 
dipole moment,  although these are not considered in this work.

The line strength is also proportional to the relative populations of 
the upper and lower transition levels.  In thermodynamic equilibrium 
the probability of a molecule being in a specific energy level is given by
\begin{equation}
\frac{N_i}{N}=g_i \exp(-\frac{hc}{kT}E_i)/Z(T)
\label{eqn:boltz}
\end{equation}
where $h$ is Planck's constant, $c$ is the speed of light, $k$ is 
Boltzmann's constant, $T$ is the temperature, $N$ is the total number of
molecules, $N_i$ is the number of molecules in energy level $E_i$, $g_i$ is 
the statistical weight of the level, and $Z(T)$ is the partition function 
given by:
\begin{equation}
Z(T)=\sum_i g_i \exp(-\frac{hc}{kT}E_i)
\label{eqn:partfunc}
\end{equation}
In this work, the partition functions are computed using Gamache's
\cite{gam:90} convenient parameterization: 
\begin{equation}
Z(T) = a + b T + c T^2 + d T^3
\label{eqn:gamache}
\end{equation}
where $a$, $b$, $c$, and $d$ have been tabulated for most molecules found
in the lower atmosphere.  Combining Equations \ref{eqn:boltz} and 
\ref{eqn:partfunc}, the relative 
population of the upper and lower energy levels is given by:
\begin{equation}
 \frac{N_j - N_i}{N}=
  \frac{g_j \exp(-\frac{hc}{kT}E_j) - g_i\exp(-\frac{hc}{kT}E_i)}{Z(T)}
\end{equation}
The line strength is then expressed as 
\begin{equation}
S_{i,j}=\sigma_{i,j}\frac{N_j - N_i}{N}
\end{equation}
where $\sigma_{i,j}$ are integrated absorption cross sections and are 
given by
\begin{equation}
\sigma_{i,j}=\frac{8\pi^3}{3h}\nu_{i,j}|R_{i,j}|^2.
\end{equation}
Combining these results yields the full expression for the line strength:
\begin{equation}
S_{i,j}=\frac{8\pi^3}{3h}\nu_{i,j}|R_{i,j}|^2 
 \frac{g_i \exp(-\frac{hc}{kT}E_i)}{Z(T)}[1 -\exp(-\frac{hc}{kT}\nu_{i,j})]
\end{equation}
Using line strengths determined either theoretically or experimentally at
some reference temperature $T_{ref}$, the strength can be converted to other
temperatures using
\begin{equation}
 S_i (T) = S_i (T_{ref})
 \frac{ Z(T_{ref}) }{ Z(T) }
 \frac{ \exp(-hcE_i /kT) }{ \exp(-hcE_i /kT_{ref}) }
 \frac{ [1-\exp(-hc\nu_i /kT)] }{ [1-\exp(-hc\nu_i /kT_{ref})] }
\label{eqn:linestren}
\end{equation}

\subsubsection{Line Widths}

The line {\it width}, or halfwidth, is defined as half the frequency 
interval 
between $\nu_{0}$ and the frequency at which $k(\nu)$ has fallen to one
half of its maximum value.  Values of line widths in the Earth's atmosphere
can range from 0.0002 cm$^{-1}$ for
conditions where the molecules are isolated to 0.5 cm$^{-1}$ for conditions
of extreme pressure broadening.  Under pressure
broadening conditions, the resulting lineshape near line center is
well approximated as a Lorentzian with a nominal line width of
\begin{equation}
\gamma = \frac{2r^2}{m}\frac{P}{R_m T}
         \left(\frac{3kT}{m}\right)^{\frac{1}{2}},
\end{equation}
which has been derived from classical Kinetic theory using the 
Equipartition theorem and the Ideal gas law.  $r$ s the effective radius 
of the molecule,  $m$ is
the molecule's mass, $P$ is the total pressure, $R_m$ is the gas constant,
and $T$ is the temperature.  A typical time between collisions for
an atmospheric gas at room temperature and pressure is $\sim 10^{-10}s$,
which leads to a Lorentz width of $\sim$0.05 cm$^{-1}$.  
If the line width, $\gamma_0$, is determined at a given 
pressure, temperature combination ($P_0, T_0$), the line width at other 
conditions is
\begin{equation}
\gamma = \gamma_0\left(\frac{T_0}{T}\right)^{\frac{1}{2}}
         \left(\frac{P}{P_0}\right)
\end{equation}
Thus, the line width increases linearly with pressure and decreases with
temperature.   Although this kinetic theory does not result in accurate
values of $\gamma_0$, the pressure dependence is observed in most cases.
More commonly, $\gamma_0$ is determined either experimentally or
calculated with more realistic theories when accurate measurements are not
available.  Furthermore, the temperature exponent, $\frac{1}{2}$, is
generally replaced with a parameter $n$, which is also determined
experimentally. The accuracy of $n$ was
investigated by Lui Zheng and Strow \cite{lui:89} for both
CO$_{2}$$\leftrightarrow$ CO$_{2}$ and CO$_{2}$ $\leftrightarrow$ N$_{2}$
collisions: $n \simeq 0.69$ for CO$_{2}$ $\leftrightarrow$ CO$_{2}$
collisions and $n \simeq 0.75$ for CO$_{2}$ $\leftrightarrow$ N$_{2}$
collisions.  In general, $n$ can vary with transition for the same
molecule.  For example, accepted values of $n$ for H$_2$O 
range from 0.5 to 1.  When $n$ is unknown, default values of 0.64 and 0.68
are generally used.

For mixtures of gases, the total line width is the sum of the individual
partial widths:
\begin{equation}
\gamma_{TOT}=\sum_i \gamma_{0,i}P_i
\end{equation}

From quantum Fourier transform theory calculations, the line width for the
$f\leftarrow i$ transition is calculated using \cite{gam:83}
\begin{equation}
\gamma_i = \left(\frac{n v}{2\pi c}\right)\sum_{J_2}\rho(J_2)\sigma_{if,J_2}
\end{equation}
where $n$ is the perturber density, $c$ is the speed of light, $v$ is the
mean relative thermal velocity ($v=\sqrt{8k_B T/\pi \mu}$), $\mu$ is the
reduced mass of the perturber/absorber system, and $\rho(J_2)$ is the
density of the perturber state $J_2$.  $\sigma_{if,J_2}$ are the absorption
cross sections and are dependent upon which type of interactions are
dominant.  For example, for H$_2$O-N$_2$ collisions, the strongest
interaction is dipole-quadrupole, yielding 
\begin{equation}
\sigma_{if, J_2}^{DQ}=\pi b_0^2 \left(1+s_{if,J_2}(b_0)\right)
\end{equation}
where $b_0$ is an impact parameter related to the minimum distance between 
absorber and perturber during the interaction and $s_{if,J_2}$ is related
to the dipole moment of H$_2$O, the quadrupole moment of N$_2$, and the
impact parameter.  For self-broadened H$_2$O, the main interaction is
dipole-dipole and similar calculations can be performed.  This quantum
treatment of line widths represents a large improvement over simple kinetic
theory calculations.  Such calculations, however, are most often scaled to 
agree with experimental results to obtain the highest degree of accuracy 
and are included in spectral line databases whenever accurate measurements 
are not available or possible.

Experimental studies of line widths can become surprisingly complicated for
several reasons.  One common complication is due to the overlapping and
blending of adjacent spectral lines.  Others include excessive experimental
noise, badly-characterized instrument functions, incorrect ``background'' 
absorptions, and lack of characterization of the optical path.  Some of 
these concerns have been reviewed by Gamache {\it et.al.}\cite{gam:94}.
Furthermore, for some gases such as water vapor, experimental results from 
different investigators for the same spectral line lie well outside quoted 
uncertainties.  Larger systematic and analysis errors, not inaccurate
experimental spectra, are most likely responsible for these disagreements.
The case for N$_2$-broadened water vapor line widths is investigated in
detail in section \ref{chpt:h2o}.

\subsection{Lineshape Theories}\label{sec:shapetheory}

The frequency dependence of the absorption coefficient is determined by the 
molecule's physical state and its environment.  Broadening factors can be 
divided into three general classes.  They are (1) natural broadening,
(2) Doppler broadening, and (3) collision broadening.  While natural and 
Doppler broadening can be described with relatively simple theoretical
models, providing an accurate generalized collision broadening theory is a 
very challenging problem.  Each of these are addressed below.

Thew HITRAN line parameters can be found at 
http://www.cfa.harvard.edu/hitran/. 
\begin{figure}[h]
  \begin{center}\includegraphics[width=6in]{Figures/HITRAN_format}\end{center}
  \caption{HITRAN format}
  \label{fig:hitran}
\end{figure}

\subsubsection{Natural Broadening}\label{sec:natural}

The {\it natural} lineshape is best described by considering a stationary,
isolated molecule. If such a molecule is allowed to absorb radiation,
undisturbed by any other form, it will eventually make a transition back 
to a lower level of internal energy.  Consequently, the molecule has a 
limited lifetime at any given energy level.  The resulting lineshape is 
given by
\begin{equation}
 k_{nat}(\nu)=\frac{S}{\pi}\left(\frac{\gamma_{nat}}
{(\nu-\nu_{0})^{2}+\gamma_{nat}^{2}}\right)
\end{equation}
where $\gamma_{nat}=1/\tau_{nat}$ is the ``natural'' line width.  Due to
the relatively long lifetimes of these undisturbed molecules, 
$\gamma_{nat}$ is very small, with values on the order of
$10^{-5}$ cm$^{-1}$.  For this reason, natural lineshapes are not 
observable 
under atmospheric conditions or with spectrometers of average resolution.

\subsubsection{Doppler Broadening}\label{sec:doppler}

The inhomogeneous {\it Doppler} lineshape is applicable to conditions 
encountered in 
the upper troposphere and stratosphere.  In these cases, the temperature is 
assumed to be high enough to produce molecular motion, but the pressure is 
low enough so that the molecules experience no collisions; or at least are 
not subject to {\it strong} collisions which terminate the dipole moment 
oscillation.
At pressures of about 5 Torr or less, the Doppler lineshape 
is predominant, with a typical line width of 0.001 cm$^{-1}$ at 296 K. 
The molecular motion produces an apparent shift in the observed frequencies
and such broadening is called Doppler broadening.

The shifted Doppler frequency, $\nu '$, for a molecule moving with a
speed $v_{m}$ along the direction of observation, relative to the observer,
is given by
\begin{equation}
 \nu '=\nu_{0}\frac{\sqrt{1-(v_{m}/c)^{2}}}{1+v_{m}/c}
\end{equation}
where $\nu_{0}$ is the un-shifted frequency. For $v_{m} \ll c$, $\nu '$
can be approximated with a binomial expansion as
\begin{equation}
 \nu '=\nu_{0}\left(1-\frac{v_{m}}{c}\right)
\end{equation}
Therefore, for each $v_{m}$, there exists a corresponding shifted frequency.
Given a Maxwell distribution of velocities within the gas, the density of
molecules with velocity $v_{m}$ is given by
\begin{equation}
 dn=N\left(\frac{m}{2\pi kT}\right)^{\frac{1}{2}}\exp\left(-\frac{m}{2kT}
v_{m}^{2}\right)dv_{m}
\end{equation}
$N$ is the total number of molecules, $m$ is the molecular mass, $T$ is the
temperature, and $k$ is Boltzmann's constant.  The corresponding absorption
coefficient, $k_{D}(\nu)$, like the Boltzmann distribution, has a
Gaussian form:
\begin{equation}
 k_{D}(\nu)=\frac{S}{\gamma_{D}}\sqrt{\frac{\ln 2}{\pi}} \; e^{-\ln 2
  \left(\frac{\nu -\nu_{0}}{\gamma_{D}}\right)^{2}}
\end{equation}
$\gamma_{D}$, the line width of the Doppler lineshape, is given by
$\nu_{0}\sqrt{\frac{2kT \ln 2}{mc^{2}}}$.    Notice how quickly the
Doppler lineshape goes to zero far from the line center due to the negative
exponential.  
%It is interesting to note that in the relativistic limit, a
%transverse Doppler shift also occurs, yielding a shifted frequency of
%$\nu_{0}\sqrt{1-\frac{v_{m}^{2}}{c^{2}}}$.  

\subsubsection{Collision Broadening}\label{sec:collision}

At pressures greater than $\sim$ 5 Torr, the collisions between molecules
must be addressed.  Collisions are the most important phenomenon to
contribute to broadening at these higher pressures.  In 1906 Lorentz showed
that line broadening takes place when absorbing molecules or atoms collide.
If one assumes that a collision takes place during the time in which
radiation is being absorbed, the coherency of the wave train is
interrupted.  This interruption results in a broadening of the spectral
line.  Quantum mechanically, pressure broadening is caused by the
broadening of the molecules' energy levels by fields produced by the
colliding molecules.  This is a complex
subject, and exact solutions for the absorption coefficient are found only
under certain approximations.  The exact treatment of this problem
requires the knowledge of the time-dependent quantum mechanical
wavefunction of an ensemble of colliding molecules.  In general, this has 
not been achieved to date and therefore the problem is often approached by 
developing empirical or semi-empirical models which simulate the system.
In the following sections, a model leading to the standard Lorentz
lineshape is presented, followed by descriptions of more complex techniques
of dealing with collisional broadening.

\subsubsection{Lorentz Lineshape}\label{sec:lorentz}

In the simplest treatment, the collisional lineshape is that of a
Lorentzian.  At high pressures, collisions occur often and it is unlikely
that a molecule is allowed to oscillate undisturbed for its entire
natural lifetime.  Instead, the molecule is usually perturbed by many 
collisions.  This model makes several assumptions which lead to a simple
solution for the absorption coefficient.  The molecule's dipole moment is
assumed to be oscillating with frequency $\nu_{0}$.  When a collision
occurs at time $t$, the oscillation terminates instantaneously.  No
natural damping is included because the time between collisions,
$t$, is much less than the natural lifetime, $\tau_{nat}$.  In other
words, $\exp(-t/\tau_{nat})=1$ for all times considered.

It is important to understand the assumptions which have been made for this
model.  One of the assumptions is called the {\it impact approximation}, 
which assumes that the time between collisions is much greater than the 
duration of a collision, $\tau_{dur}$, and therefore, the behavior of
the dipole moment during the collision is negligible.  In this case,
$\tau_{dur}$ is taken to be zero, corresponding to an instantaneous 
phase shift in the dipole moment.  These types of collisions are also
sometimes called {\it adiabatic} in that the system has no time to react
to the collisions.  The opposite of the impact approximation is called the
{\it quasi-static} approximation, in which the collision durations are
essentially assumed to be much larger than the time between collisions.
This point is addressed when statistical lineshapes are discussed.
Another assumption made here is that
of {\it strong collisions}.  A strong collision is taken to be an 
interaction which terminates the oscillation, leaving no memory regarding 
its orientation or other properties before the collision.
On the other extreme, {\it weak} collisions are those which have little or
no effect in disturbing the molecule.  In this case, collisional effects 
are only felt as a damping effect after a large number of weaker impacts.
Collisions are also assumed to involve only two molecules, and such 
collisions are referred to as {\it binary collisions}.  One final
assumption is that the molecules follow classical straight line
trajectories between collisions.  So in the Lorentz model, which is often 
called the billiard-ball model, colliding molecules can be thought of as 
quickly moving hard spheres which do not interact with one another 
until they actually touch each other.  

Fourier analysis of this model wavetrain leads to a spectral distribution of
the form
\begin{equation}
\mid {F}\{\mu(t)\} \mid^{2}= \frac{\sin^{2}
  [2\pi(\nu-\nu_{0})t/2]}{[2\pi(\nu-\nu_{0})]^{2}}
\end{equation}
This expression must be averaged over all possible values of $t$.
From the kinetic theory of gases, the distance traveled between collisions, 
$l$, by a molecule of average velocity $v_{m}$ has a Poisson distribution:
\begin{equation}
 p(l)dl=\frac{dl}{l_{m}}e^{-l/l_{m}}
\end{equation}
where $l_{m}$ is the mean free path.  Using $dt=\frac{dl}{v_{m}}$, the
distribution for the time between collisions is
\begin{equation}
 p(t)dt=\frac{dt}{\tau_{col}}e^{-t/\tau_{col}}
\end{equation}
where $\tau_{col}$ is the mean time between collisions.  Using this
distribution, the absorption coefficient, $k_{L}(\nu)$ is found to be
\begin{equation}
 k_{L}(\nu)=\frac{S}{\pi}\left(\frac{\gamma_{L}}
{(\nu-\nu_{0})^{2}+\gamma_{L}^{2}}\right)
\end{equation}
where $\gamma_{L}=1/\tau_{col}$ is the Lorentz line width.  Within this
billiard-ball model, $\tau_{col}$ is calculated as $l_{m}/v_{m}$ and has 
an average value of about $1.5 \times 10^{-10}$ seconds\cite{bre:56}.
This corresponds to a Lorentz line width of approximately 0.02 cm$^{-1}$, 
which is much larger than a typical Doppler width.  Thus, whenever 
collisions are present, they provide the primary form of broadening.  

This absorption coefficient is called the {\it Lorentz} 
lineshape.  It has the same form as the natural lineshape; the only
difference being the value of the line widths.  It is useful to compare 
the Doppler and Lorentz lineshapes.  The Doppler model assumes a Boltzmann 
velocity distribution, which goes smoothly to zero at large velocities.  Its
corresponding spectral distribution, therefore, also decays quickly in the 
far-wing (far from line center).  This is not the case for the Lorentz
model, which assumes instantaneous behavior during collisions.  The effect
of this unphysical temporal behavior is the placement of extremely high
frequency components in the lineshape's spectral distribution.
Consequently, $k_{L}(\nu)$ is too large in the far-wing, and the Lorentz
model predicts too much absorption in this region.

Despite the apparent shortcomings of the model used for the Lorentzian line
shape, it is very accurate for many applications.  The Lorentz lineshape is
accurate as long as two conditions are satisfied: (1) the spectral
region of interest is not too far removed from the line center where the 
impact approximation results in the prediction of too much absorption, and 
(2) there exists no significant overlapping of adjacent 
spectral lines.  The latter of these
two conditions arises because the Lorentz theory assumes no transfer of
intensity from one spectral line to another (often called ``line mixing'').
Experimental deviations from the Lorentz lineshape within $\sim$2--4
$\gamma_0$ of line center of isolated lines have not been confirmed for
systems of atmospheric interest.
%Other collisional broadening theories which take these effects into
%account have been developed and will now be reviewed.

\subsubsection{Van Vleck - Weisskopf Lineshape}\label{sec:vvw}

When describing the procedures used to calculate
the Lorentz and natural lineshapes, the assumption that transitions occur
at relatively high frequencies (i.e. infrared) was made.  When
computing the Lorentz lineshape, the Fourier transform of the dipole moment 
actually yields two terms -- one centered about $\nu_{0}$ and
the other about $-\nu_{0}$. The lineshape should be written as 
\cite{van:45}
\begin{equation}
k(\nu)=\frac{S}{\pi}\left(\frac{\gamma}{(\nu-\nu_0)^{2}+\gamma^{2}}
+ \frac{\gamma}{(\nu+\nu_0)^{2}+\gamma^{2}}\right).
\label{eqn:vvw1}
\end{equation}
This shape is more often used in the microwave
region of the spectrum, where the second term of the sum is not negligible.
For molecules active in the infrared region, however,  $\nu_{0}$ is large
enough such that $(\nu-\nu_{0}) \ll (\nu+\nu_{0})$ and the resulting
lineshape can most often be safely approximated as Lorentzian.  An
exception is in ``window'' regions (far from any line centers).
Another modification to
the Lorentz model involves the behavior of the molecule directly after a 
collision.  In the Lorentz model, we essentially assumed the wave-function 
experienced {\it random} phase shifts during collisions and immediately
began oscillating again at its resonant frequency.  The wave-function,
however, does not experience a random reorientation, but should be
distributed according to the Boltzmann distribution of the field when the
collision occurs\cite{van:77}.  Following this approach leads to a slight 
modification of Equation \ref{eqn:vvw1}:
\begin{equation}
k(\nu)=\frac{S}{\pi}\left(\frac{\nu}{\nu_0}\right)^2 
\left(\frac{\gamma}{(\nu-\nu_0)^{2}+\gamma^{2}}
+ \frac{\gamma}{(\nu+\nu_0)^{2}+\gamma^{2}}\right)
\label{eqn:vvw2}
\end{equation}
which is commonly called the {\it Van Vleck-Weisskopf} lineshape 
\cite{van:45}.
Equation \ref{eqn:vvw2} is an improvement over Equation \ref{eqn:vvw1} in
that (\ref{eqn:vvw1}) predicts no absorption in the limit of zero resonant
frequency, which is not observed experimentally.  Equation \ref{eqn:vvw2}
is also more acceptable in that it agrees with 
Debye's\cite{van:77} relaxation theory in the same limit.

\subsubsection{Van Vleck - Huber Lineshape }\label{sec:vvh}

Another similar lineshape which was developed to satisfy the principle of
{\it detailed balance} (discussed below) is\cite{van:77}
\begin{equation}
k(\nu)=\frac{S}{\pi}\left(\frac{\nu}{\nu_0}\right)
\frac{\mbox{tanh}(hc\nu/2kT)}{\mbox{tanh}(hc\nu_0/2kT)}
\left(\frac{\gamma}{(\nu-\nu_0)^{2}+\gamma^{2}}
+ \frac{\gamma}{(\nu+\nu_0)^{2}+\gamma^{2}}\right)
\label{eqn:vvh}
\end{equation}
which is called the {\it Van Vleck-Huber} lineshape.

\subsubsection{Voigt Lineshape }\label{sec:voigt}

Before going on to explain more elaborate models,
the {\it Voigt lineshape} should be introduced.  It does not
introduce any new physical insight into broadening phenomenon, but is very
useful computationally.  The Voigt lineshape is the convolution of the
Doppler and Lorentz lineshapes.  For this reason, it assumes Doppler
characteristics at low pressure and Lorentz characteristics at higher
pressures.  Therefore, one single expression for the lineshape can be used
throughout a wide range of pressures.  The Voigt lineshape, $k_{V}(\nu)$
is given by
\begin{equation}
 k_{V}(\nu)=\frac{k_{0}y}{\pi}\int_{-\infty}^{\infty} \frac{e^{-t^{2}}}
{y^{2} + (x-t)^{2}}dt
\end{equation}
with
\begin{equation}
 k_{0}=\frac{S}{\gamma_{D}}, \; \; \;
y=\frac{\gamma_{L}}{\gamma_{D}}\sqrt{\ln 2}, \; \; \;
x=\left(\frac{\nu-\nu_{0}}{\gamma_{D}}\right)\sqrt{\ln 2}
\end{equation}
The Voigt lineshape does assume there is no correlation between collision
cross sections and the relative speed of the colloids.  Again, for
atmospheric systems this approximation appears to be quite accurate.

The VanVleck-Huber lineshape can be computed using the Voigt lineshape 
instead of the Lorentz lineshape.

\subsubsection{General Techniques for Calculating Collisional
Lineshapes }\label{sec:gentechs}

The lineshape models have been presented informally in order to provide 
general physical insight.  However,
for more advanced approaches, it is useful to understand the more formal
techniques in which absorption coefficients are calculated.  This is needed
to help understand the deviations from Lorentz lineshapes. For more on 
this, one is refered to Dave Tobin's \cite{tob:96} thesis for a 
discussion, as well as for more references. In particular, his 
dissertation describes line mixing for carbon dioxide, as well as 
intermolecular forces and potentials used in lineshape calculations for 
water vapor.

%%%%%%%%%%%%%%%%%%%%%%%%%%%%%%%%%%%%%%%%%%%%%%%%%%%
\section{Water vapor lineshape}\label{sec:local}
When computing water wapor spectral lineshapes, the effects of the lines
far away from the current region can be included in two ways : by directly 
individually adding on the far wings of each line to the current region, or
just using the lines in the current region, plus a lump sum ``continuum'' 
contribution. As the far wings lineshapes might not be lorentz, and also to 
account for the possibility that the near wing lineshapes might also not
be lorentz, a local lineshape definition is preferentially used, along with 
a continuum contribution. 

The local lineshape is defined as the lorentz lineshape out to 
$\pm 25 cm^{-1}$
minus the lorentz value at 25$cm^{-1}$ away from line center. With this 
definition, the possibility that the lineshape near linecenter is itself not
lorentz, can be modelled by including the effect into the continuum that has
to be added on.

Non-Lorentz H$_2$O lineshapes also have a
significant impact {\it within} the strong pure rotational and vibrational
bands.  This {\it in-band} continuum is
particularly important for satellite infrared remote sensing of atmospheric
H$_2$O profiles.  

For well isolated pressure-broadened water vapor lines in the infrared, the
Lorentz lineshape is very accurate near line center.  However, if one uses
a Lorentz lineshape, this generally overestimate the observed 
absorption in the far-wing atmospheric window regions and underestimates the
absorption within the rotational and vibrational bands. This means that 
the actual water vapor lineshape is extremely sub-Lorentzian in the 
far-wing and at least somewhat super-Lorentzian in the intermediate
and near-wing.  Most experimental studies have focussed on the window 
regions and so the far-wing lineshape has been studied more than 
the near-wing (roughly within 5 to 25 cm$^{-1}$ of line center).

Deviations of H$_2$O spectral lineshapes from Lorentz have been studied
extensively for the atmospheric windows at 4 and 10 $\mu$m.  In general,
these deviations are observed to vary slowly with wavenumber and the 
anomalous absorption has become known as the water vapor continuum.  

Several characteristics were found to be common to all window region 
continuum studies.  In general, the continuum absorption 
\cite[for example]{bur:81,gra:90}: (1) does not change rapidly
with wavenumber, (2) decreases rapidly with increasing temperature for
pure water vapor, (3) is greater for self-broadening than for foreign
broadening, (4) is more significant in regions of weak absorption
than in regions of strong absorption, and (5) displays the pressure
dependencies associated with gaseous absorption.

It is accepted that the deviations from the impact theory calculations in 
window regions are due to the non-Lorentz behavior of the 
far-wings of pure rotational and vibration-rotation water vapor 
absorption lines.  

In the following sections, a review of previous studies of this continuum 
absorption is presented.  These can be separated into two generally 
different approaches.  The first, which is most often adopted in 
experimental
studies, is to express the observed deviations from Lorentzian behavior
through the use of {\em continuum coefficients}.  With this method, the
cumulative effects of all lines are characterized in a convenient form.
The second approach, which provides more information about the shape of 
individual spectral lines, is used in most theoretical studies.
$\chi$-functions are often the end result of this approach.

\subsection{A Definition of the Continuum}

The definition proposed by Clough is widely used in atmospheric 
spectroscopy and radiative transfer, particularly in line-by-line codes 
such as FASCODE\cite{clo:81}, GENLN2\cite{edw:87}, LINEPAK\cite{gor:94}, 
and LBLRTM.  The ``local''
absorption for a single transition is defined as a Lorentz lineshape out to
$\pm$25cm$^{-1}$ from the line center, minus the Lorentz value at 25cm
$^{-1}$. For several lines, the local absorption is expressed 
as\cite{clo:89}
\begin{equation}
k_{local}(\nu)=\nu\tanh\left(\beta\nu/2\right) \rho_{ref}
\frac{T_{ref}}{T}P_{H_{2}O}L
\sum_i\frac{S_i}{\pi}\left\{\begin{array}{cl}
\frac{\gamma_i}{\Delta\nu^{2}+\gamma_i^2}-\frac{\gamma_i}{25^2+\gamma_i^2}
                        & \mbox{if $|\Delta\nu| \leq 25 \mbox{cm}^{-1}$} \\
        0               & \mbox{if $|\Delta\nu| > 25 \mbox{cm}^{-1}$}
\end{array}\right.
\label{eqn:local_def}
\end{equation}
where $T_{ref}=273.15K$, $\rho_{ref}$ is the absorber number density per
atmosphere at $T_{ref}$, $\beta=hc/kT$, and $\Delta\nu=\nu-\nu_i$.
All of the continuum measurements presented in this work are consistent
with Equation \ref{eqn:local_def}.  
This is actually a slight modification of Clough's
definition\footnote{Equations 6 through 8 of Reference
\protect\cite{clo:89} do not reflect the local lineshape definition used
in FASCODE.  They actually lead to a $\chi$ dependent local lineshape, 
which is not used in the line-by-line codes.}, which
also includes the negative resonance terms ($\frac{\gamma_i}{(\nu+\nu_i)^2
+\gamma_i^2}$).  In the infrared region (actually for $\nu > 25$
cm$^{-1}$), the two definitions are equal.  {\em The continuum is then 
simply defined to be any observed absorption not attributable to the local
absorption}.  The continuum therefore includes 
far-wing absorption (beyond 25cm$^{-1}$ from line center), absorption 
due to any near-wing (within 25cm$^{-1}$) non-Lorentz behavior, and 
the Lorentzian value at 25cm$^{-1}$ within $\pm$25cm$^{-1}$ of line 
center (this is often called the ``basement'' term).  This is illustrated 
in Figure \ref{fig:localdef} for a single absorption line.
The ``basement'' term is a relatively minor part of the continuum and is 
introduced to ensure a smooth continuum for computational reasons.  

\begin{figure}
 \begin{center}
 \includegraphics[width=6in]{Figures/H2O/local_def}
 \end{center}
 \caption[The local lineshape definition.]{The local lineshape definition
 used in this work.  The far-wing (beyond 25 cm$^{-1}$), near-wing (within
 25 cm$^{-1}$), and basement components of the continuum are labeled.}
 \label{fig:localdef}
\end{figure}

With this definition of the local absorption, the continuum is always a 
positive quantity.  The basement and far-wing
components are certainly always positive.  The near-wing component, which
represents the difference between the actual lineshape and Lorentz within
25 cm$^{-1}$, is also positive because water vapor has a super-Lorentzian
lineshape in this region. In fact, with this continuum definition, a 
non-zero continuum exists even for the Lorentz lineshape.
A calculation of the total absorption coefficient and the continuum 
absorption (total minus local) using the Lorentz lineshape for the 
0-4000 cm$^{-1}$ region is shown
in Figure \ref{fig:lor_con}.  All of the high frequency components of the
absorption are contained in $k_{local}$ and the continuum is therefore a 
smoothly varying function, which can be stored in a look-up table for ease 
of computation.

\begin{figure}
\begin{center}
\includegraphics[width=6in]{Figures/H2O/lor_con}\end{center}
\caption[Lorentz calculations for 0-4000 cm$^{-1}$.]{Absorption coefficient
	calculations using the Lorentz lineshape for 0-4000 cm$^{-1}$ at
	$\sim$2 cm$^{-1}$ resolution.  The total
	(solid curve) and continuum (dashed curve, as defined by Equation 
	\protect\ref{eqn:local_def}) absorption coefficients are shown.  The
calculations include the effects of all lines between 0 and 5000 cm$^{-1}$.
Conditions are: 1 torr H$_2$O, 760 torr N$_2$, 12 m path length, 296K.}
\label{fig:lor_con}
\end{figure}

In the line-wings, the displacement $\Delta\nu$ is much greater than the
halfwidth $\gamma$ and the Lorentz terms can be approximated as 
$\frac{\gamma_i}{\Delta\nu^2}$.  This leads to a quadratic pressure
dependence in the absorption coefficient on $P_{H_2O}$ for self-broadened 
water vapor and a linear dependence on both $P_{H_2O}$ and the broadening 
pressure, $P_f$, for foreign broadened water vapor.  Since the continuum 
is mainly due to line wings, the total continuum absorption coefficient 
for all lines is formulated as \cite{bur:81,clo:89}
\begin{equation}
k_{con}(\nu)=\nu\tanh\left(\beta\nu/2\right) \rho_{ref}
\frac{T_{ref}}{T}P_{H_{2}O}L
\frac{296}{T}\left(P_{H_{2}O} C_s^0(\nu,T) + P_{f} C_f^0(\nu,T)\right)
\label{eqn:kcon}
\end{equation}
where $C_s^0$ and $C_f^0$ are the self- and foreign-broadened 
continuum coefficients at 296K and 1 atmosphere.  To express experimental
and theoretical results, the quantities $C_f^0$ and $C_s^0$ are often used.

Because absorption in the windows is very weak, all spectra gathering 
techniques require very long path lengths.  The laboratory studies have 
focussed primarily on self- and nitrogen broadening at room temperature or 
above, while most atmospheric measurements have naturally looked at 
air-broadening at room temperature or below.  Most of the results from 
these measurements are in accord with
those of Burch {\it et.al.}, which are discussed in the larger lbl.pdf
\newpage


\bibliographystyle{unsrt}
%\bibliography{/salsify/packages/Tex/atmspec}
\bibliography{/home/sergio/PAPERS/BIB/atmspec2002}

\end{document}

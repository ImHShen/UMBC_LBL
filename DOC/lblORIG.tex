\documentclass[11pt]{article}

\usepackage{lucbr,graphicx,fancyhdr,longtable}

\newcommand{\kc}{\textsf{kCARTA}\xspace}
\newcommand{\cm}{\hbox{cm}}

% make a single, doubly indented line 
% (mainly used for driver file examples)
\newcommand{\ttab}{\indent\indent}

\input ASL_defs

\setlength{\textheight}{7.5in}
\setlength{\topmargin}{0.25in}
\setlength{\oddsidemargin}{.375in}
\setlength{\evensidemargin}{.375in}
\setlength{\textwidth}{5.75in}

\newlength{\colwidth}
\setlength{\colwidth}{8cm}
\newlength{\colwidthshort}
\setlength{\colwidthshort}{6cm}

\pagestyle{fancy}

% \date{July 5, 1994} % if you want a hardcoded date

\lhead{\textbf{\textsf{DRAFT}}}
\chead{UMBC\_LBL}
\rhead{\textsf{Version 6}}
\lfoot{UMBC}
\cfoot{}
\rfoot{\thepage}

\newcommand{\HRule}{\rule{\linewidth}{1mm}}
\newcommand{\HRulethin}{\rule{\linewidth}{0.5mm}}

\begin{document}
\thispagestyle{empty}
\vspace{2.0in}

\noindent\HRule
\begin{center}
\Huge \textbf{\textsf{UMBC\_LBL}}: An Algorithm to Compute Line-by-Line 
Spectra
\end{center}
\noindent\HRule

\vspace{0.75in}
\begin{center}
\begin{Large}
Sergio De Souza-Machado, L. Larrabee Strow,\\ David Tobin, Howard Motteler 
and Scott Hannon
\end{Large}
\end{center}

\vspace{0.5in}
\begin{center}
Physics Department\\
University of Maryland Baltimore County\\Baltimore, MD 21250 USA\\
\end{center}

\vspace{0.5in}
\begin{center}
Copyright 1999 \\
University of Maryland Baltimore County \\
All Rights Reserved\\
v6  \today\\
\end{center}

\vfill

\noindent\HRulethin
\begin{flushleft}
\begin{tabbing}
Sergio~De~Souza-Machado: \=    sergio@umbc.edu \\
L.~Larrabee~Strow:   \>        strow@umbc.edu\\
\end{tabbing}
\end{flushleft}

%\begin{flushright}
%\includegraphics[width=1.0in]{umseal.eps}
%\end{flushright}

\newpage
\tableofcontents
\listoftables
\listoffigures

\newpage

\begin{center}
{\bf ABSTRACT}
\end{center}

  We have developed a line-by-line code to compute the spectral lineshapes
  of gases under varying pressure and temperatures. This code can use the
  spectral line parameters of a variety of databases such as HITRAN or 
  GEISA. 
  The code ensures that the same lines are used in all layers of an
  atmosphere, so that the resulting line profiles are smooth. The latest
  models for CO2 line mixing and in water vapor lineshapes are incorporated
  in the code. The code is can be used to compute absorption spectra
  at user specified resolutions, using one of various lineshapes.
  One of our uses for the code is to generate new versions of the 
  spectroscopic database for kCARTA, which is the reference for the AIRS
  forward model.

This document is very much a work in progress.  Some major omissions
include references, significant examples of output, and comparisons of 
output to GENLN2.  These omissions will
be rectified in the future.  Please give us your feedback on both the
code and the documentation!  


\newpage

\section{Introduction}
This documentation describes a line-by-line code that can be used to 
compute optical depths for various gases (that appear in the HITRAN 
database). The code has been written in Matlab, with many of the routines 
written as FORTRAN-MEX files in the interests of speed. The lineshape 
database that is used in the computations, is currently the HITRAN98 
database. Other databases, such as the GEISA or HITRAN92 databases, can be 
merged in and used. With this, we can incorporate the latest lineshape 
studies and parameters into our computations. 

This code will be used to generate a new database for kCARTA, our radiative
transfer algorithm that is the reference for the AIRS forward model. AIRS is
a high resolution infrared instrument due to be launched by NASA in the 
year 2000. The instrument will be used to make measurements of the gas 
content and temperatures of the Earth's atmosphere, as well as study the 
global climate and weather. 

For all gases except carbon dioxide and water vapor, the lineshapes of the
individual lines are simply added together to give the overall absorption
spectrum. To speed things up, the code uses the binning methods of GENLN2, 
where the lines are divided into near lines, medium-far lines and far 
lines. The general code to compute the lineshape of this majority of gases 
is called $run6.m$ Note that we also have code to incorporate the 
computation of the lineshapes of the cross section gases as well.

For the case of carbon dioxide, we have utilised Dave Tobin's work on 
linemixing and duration of collisions effects in the 4 $\mu$m band, and
applied it to the 15 $\mu$m band as well. The code to compute the lineshape 
of carbon dioxide is called $run6co2.m$

For the case of water vapor, we have written $run6water.m$. This code can 
compute the spectral lineshapes with or without the basement term removed,
and add on the user specified continuum (CKD 0,2.1,2.3 or 2.4). In addition,
this code also includes Tobin's work on water vapor chi functions.

We begin the documentation by briefly describing how our code computes an 
optical depths, given the necessary input parameters. This section is 
divided into three; the first can be used for all gases, while the second 
and third apply to water vapor and carbon dioxide respectively.

The next sections describe the $UMBC\_LBL$ Version 6 code implemetation. 
A section on the basic code for most gases, $run6.m$ is then followed by 
section on $run6water.m$ and $run6co2.m$.

Theis is followed by sections which decscribe spectral 
lineshapes in detail. First is a section  on general spectral lineshape 
theory, where the meaning of terms such as line widths and line broadening 
is explained. Next are two sections, one on 
computing water vapor spectra and the other computing carbon dixide 
spectra. As mentioned in the previous paragraphs, to accurately compute 
the absorption spectra of these two gases, one needs to use more involved 
lineshape theories than those for other gases. This is particularly 
important as these two gases are actively important in the infrared 
portion of the radiation emitted by the Earth, and so a thorough 
understanding of their spectra would be beneficial to the remote sensing 
community. 
 
Finally, a short set of appendices briefly describe the general code to 
compute partition functions, line widths and line strengths. 

\section{General algorithm to compute optical depth}
Assume the user wants the optical depth to be computed for a certain gas, 
using known parameters. These parameters are the total pressure $P$ , 
self pressure $PS$, temperature $T$ and gas amount $Q$. The gas amount $Q$ 
is related to self pressure, temperature, path cell length 
(or atmospheric layer thickness) $L$ by the equation 
\[ 
Q = \frac{L \times 101325 \times PS}{10^{9} \times R \times T}
\]
where R is the molar gas constant (8.31 J/mol/K), $PS$ is the partial 
pressure in atmospheres and $T$ is the temperature in Kelvin. Thus 
$\frac{101325 \times PP}{R \times T}$ gives the molecular density using 
the Ideal Gas Law (with pressure in atmospheres changed to $Nm^{-2}$; 
dividing by $10^{6}$ changes the units to moles per cubic cm, and dividing 
by a further $10^{3}$ changes it to kiloMoles per cubic cm. After 
multiplying this density by the path cell length  (or layer thickness) $L$ 
in cm, the final units are in the GENLN2 units of $kiloMoles/cm^{2}$
 
Knowing for what gas the spectrum is to be computed, and within which 
frequency interval, the lineshape parameters for that gas need to be read 
in. These  parameters can be obtained from a database such as HITRAN98, 
and should include the following : 

\begin{longtable}{lll}
TYPE &  DESCRIPTION & USE  \\
\hline

gasid   & input HITRAN GasID    & 1=water,2=CO2, etc \\
iso     &  list of isotopes     & 1=most abundant,2=next ... \\
\hline

linct   & number of lines in wavenumber interval & \\
wnum    & wavenumbers of the line centers [$cm^{-1}$] 
        & $\nu_{0}(1),\nu_{0}(2),...$ \\
tsp     & line center shift due to pressure P [cm-1/atm] 
        & $\nu_{0}(j) \rightarrow \nu_{0}(j) + P \times tsp(j)$ \\
stren   & line strengths (See Eqn \ref{eqn:linestren}) 
        & $S(T) \simeq S(296)Z(296)/Z(T) \times \rho$\\
        & [$cm^{-1}/(molecules cm^{-2})$] 
        & Z are partition functions (Eqn \ref{eqn:gamache}) \\ 
        & & $\rho \simeq $ upper/lower state populations \\
\hline
abroad  & air broadened half widths HWHM [$cm^{-1}/atm$] 
        & $brd_{air} = (P - PS) \times abroad$ \\
sbroad  & self broadened half widths HWHM [$cm^{-1}/atm$] 
        & $brd_{self} = (PS) \times sbroad$ \\
abcoef  & temperature dependence of air 
        & $brd = brd_{air} + brd_{self}$\\
      & broadened half width 
      & $brd \rightarrow brd \times (296/T)^{abcoef}$ \\
\hline
 
tprob & transition probabilility [$debyes^{2}$] & \\
els   & lower state energy [$cm^{-1}$] & computing S(T) \\
\hline

usgq & Upper state global quanta index & identifying \\
lsgq & Lower state global quanta index & P,Q,R branches \\
\hline

uslq & Upper State local quanta & local quantum numbers \\
bslq & Lower State local quanta & local quantum numbers \\
\hline

ai  & accuracy  indices & \\
ref & lookup for references & \\
\hline
 
\end{longtable}

Note that the $stren$ parameter is used in units of 
$cm^{-1}/(moleculescm^{-2})$ while the gas amounts $Q$ are in units of 
$kiloMolescm-2$. To get absorption coefficients and optical depths that 
are in the correct units, the program eventually multiplies $stren$ by 
$6.023 \times 10^{26}$, which is the number of molecules per kilomole. 

Having read in these parameters, the code is almost ready to proceed with 
the computations. Before this, it needs to read in the mass of each of the 
isotopes of the current gas. In addition, it needs to compute the 
partition function $Z$, and thus it needs four parameters $a, b, c, d$ for 
each isotope (refer to Eqn \ref{eqn:gamache}). For the current temperature 
$T$ , $Z(T)$ is then readily computed as : 
\[
Z(T ) = a + bT + cT^{2} + dT^{3}
\] 
The partition function is used when computing the line strength, as that 
term is proportional to the relative populations of the lower and upper 
levels of the transition. 

Using parameter $tsp$, any shift of the line centers due 
to the total pressure is then computed,  giving the adjusted line centers 
\[ 
\nu_{0}(P) = \nu_{0} + P \times tsp(j)
\]

The broadening of each line, due to the self component and the foreign 
component, is then computed. Note that for the atmosphere, since nitrogen 
and oxygen are the dominant gases, the self component would conceivably be 
important for these two. However, due to their structure, they are very 
weakly active in the infrared, and so for all practical purposes, there is 
hardly any change. Similarly, almost all the other gases in the atmosphere 
ae mixed in very weakly, and so the contribution to the line width, due 
to the self component, is very small. Thus if there is a slight 
perturbation to the self pressure of almost any gas in the atmosphere, 
there would hardly be any change in the broadening. Water vapor is a 
special case, as sometimes an especially dry or wet profile is 
encountered, leading to noticeable changes in the broadening. 

\[
brd_{air} = (P - PS) \times abroad
\]
\[
brd_{self} = (PS) \times sbroad
\]
\[
brd = brd_{air} + brd_{self}
\]
\[
brd \rightarrow brd \times (296/T)^{abcoef}
\]

Finally, the overall strength of each line, adjusted for the temperature 
is computed. In the equation below, $S_{i}(T_{ref})$ is the line strength 
at the HITRAN reference temeperature of $T_{ref} = 296K$, $E_{i}$ is the 
lower state energy (given by $els$) and $\nu_{i}$ is the line center 
wavenumber : 
\[
 S_i (T) = S_i (T_{ref})
 \frac{ Z(T_{ref}) }{ Z(T) }
 \frac{ \exp(-hcE_i /kT) }{ \exp(-hcE_i /kT_{ref}) }
 \frac{ [1-\exp(-hc\nu_i /kT)] }{ [1-\exp(-hc\nu_i /kT_{ref})] }
\]

With the above information for each line having been determined, the 
absorption coefficients for the individual lines can now be computed, 
using the required lineshape (Lorentz, Doppler, Voigt, etc). Looking at 
the simplest lineshape equation, the Lorentz lineshape is given by 
\[
 k_{L}(\nu)=\frac{S}{\pi}\left(\frac{\gamma_{L}}
{(\nu-\nu_{0})^{2}+\gamma_{L}^{2}}\right)
\]

where $\gamma_{L}$ is the linewidth and $\nu,\nu_{0}$ are the wavenumber
and line center frequency respectively. This means the units that result
from such computations are $1/(moleculescm^{-2}) $
 
By summing over the individual lines, the absorption spectra can be 
computed. However, for purposes of radiative transfer in the atmosphere, 
it is more convenient to think in terms of optical depths (and 
transmittances) instead of absorption spectra. With this in mind, and to 
get the units right, the adjusted line strength is simply multiplied by 
the gas amount times number of molecules per kilomole : 
$Q$ $(kiloMolescm^{-2})$ $ \times 6.023 \times 10^{26}$ 
$molecules/kilomole$, from which the optical depth of the line 
(in dimensionless units) can be computed. By summing over all the 
individual optical depths, the total optical depth can be computed : 
\[
k_{i}(\nu,\nu_{0} = S_{i}(T) \times Q \times  6.023E^{26} \times
LineShape(P, PS, T, \nu_{0}(i), \gamma(i), mass(i))
\]
\[ 
k(\nu) = \sum_{i} k_{i}(\nu,\nu_{0})
\] 

\begin{figure}[h]
  \begin{center}\includegraphics[width=1.5in]{Figures/fig0.eps}\end{center}
  \caption[Computing absorption spectra]{Flow diagram to compute $k(\nu)$}
  \label{fig:easy}
\end{figure}

The actual steps described above are summarized in Fig. \ref{fig:easy}. 
In the case of 
water vapor  and carbon dioxide, the spectral lineshape that is used for 
each line can be more complicated than a simple lorentz or voigt shape. 
This will be discussed in more detail later. For now, let us obtain some 
order of magnitude estimates for the optical depths of some of the gases 
in the atmosphere. The gas amounts $Q$ come from the AIRS layers. Assume 
the temperature $T$ is 296 K, so all temperature effects are unimportant. 
Let units $\eta = molecules/cm^{2}$. Following are typical values obtained 
from the HITRAN database, in the regions where the optical depths for the 
individual gases peak in the atmosphere : 

\begin{longtable}{llllll} 
gas & wavenumber & abroad         & sbroad       & S & Q  \\
    & $cm^{-1}$  & $cm^{-1}/atm$ & $cm^{-1}/atm$ &$cm^{-1}/\eta$ & $\eta$ \\ 
\hline
water & 1500 & 0.08 & 0.4  & 2E-19 & 9.4E-06 \\
CO2   & 2400 & 0.08 & 0.1  & 4E-18 & 3.4E-07 \\
N2    & 2400 & 0.05 & 0.05 & 4E-28 & 8.0E-04 \\
CH4   & 1400 & 0.06 & 0.09 & 1E-19 & 1.7E-09 \\
\hline
\end{longtable} 
 
There are a few points immediately obvious from this table. First, the 
self broadening of water is quite large compared to that of the other 
gases. Thus if there is a small perturbation in water self pressure, there 
will be an appreciable perturbation in the width of the water lines. 

Second, carbon dioxide also has a larger self width than the other two gases
in the table, but since its relative abundance in the atmosphere is so much
less than water, there will hardly be any perturbation to the width of
its lines, if its self pressure is perturbed. 

Thirdly, one can get a feel 
for the optical depths of the gas lines by simply multiplying $S \times Q
\times 6E26$ -- one immediately sees that methane and nitrogen have much 
smaller peak optical depths than water or carbon dioxide. Incidentally, 
this allows one to quickly set the lowest individual line strength to be 
used in the spectral computation. For example, the lower AIRS layers are 
about  200 m thick, with gas amounts of about $10^{-5}$ 
$kiloMolescm^{-2}$; using a minimum line strength of $10^{-28}$ 
$1/(moleculescm^{-2})$ with a line width of about $10^{-2} cm^{-1}$ 
implies that an isolated line will have an optical depth of about
$2 \times 10^{-5}$ 

Putting in the above values into the equations, and 
using $peak(optical depth) \simeq 6.0E26 \times S/(\pi \times \gamma)$ 
where $\gamma$ is the linewidth computed from the self and foreign 
broadening, we get the following estimates for the optical depths : 

\begin{longtable}{llllll} 
gas & total P & self P & $\gamma$   & peak abs   & peak optical \\
    & $atm$   & $atm$  & [$cm^{-1}$] & [$1/\eta$] & depth \\
\hline
water & 1.07 & 9.8E-3 & 8.87E-2 & 7.2E-19 & 4100 \\
CO2   & 1.07 & 3.4E-7 & 8.56E-2 & 1.5E-17 & 3000 \\
N2    & 1.07 & 8.4E-1 & 5.58E-2 & 2.3E-27 & 0.0011\\
CH4   & 1.07 & 1.8E-6 & 6.42E-2 & 4.9E-19 & 0.5\\
\hline 
\end{longtable} 
 
Remember that when the spectra are actually computed, there is a sum over 
the individual lines. So the peak values are in the table above are slight 
underestimates. In addition, for CO2, line mixing theory means that 
intensity is transferred away from line wings onto the line centers, and 
so the peak optical depths could be a little higher. 

To speed up the code while maintaining accuracy, our algorithm uses the 
GENLN2 method of binning the lines while computing the
spectra. Since the definition and use of the water vapor local lineshape 
and continuum uses spectral intervals of 25$cm^{-1}$, a natural choice for
the spectral interval within which an absorption spectrum/optical depth is 
computed, is 25$cm^{-1}$. 

Using typical values from the table above, the
absorption due to an individual line has decayed by 
$ \frac{\gamma}{25^{2}+\gamma^{2}}/\frac{\gamma}{\gamma^{2}}\simeq 10^{-5}$
by the time it is 25$cm^{-1}$ away from line center. This means that one can
usually use this as a cutoff distance for when a line affects the 
contribution to the overall spectrum (note that this rule does not apply for
all lines; for example, the strong 667 Q branch lines of CO2 affect the
spectrum at distances upto 200$cm^{-1}$ away). 
Similarly, the contribution has decayed by $\simeq 10^{-3}$ when it is 
2$cm^{-1}$ away, and by $\simeq 7 \times 10^{-3}$ when it is 
1$cm^{-1}$ away.

To compute the absorption 
spectrum in the chosen interval, the code divides up the interval into 
bins of width $fstep \simeq 1 cm^{-1}$.  It then loops through these bins, 
computing the overall spectra in each bin using three stages). To reinforce
the ideas, actual example numbers will be used. Suppose the wavenumber
interval being considered is 1005 - 1030 $cm^{-1}$. There are 25 bins of
width $ffin = 1 cm^{-1}$ in this interval. Suppose we are considering the
4$th$ bin, which is the bin spanning 1008 to 1009 $cm^{-1}$.
\begin{itemize}
\item (a) {\it fine mesh stage} : lines that are $\pm xnear \simeq 1cm^{-1}$
          on either side of the edges of this bin, are used in computing 
          the overall lineshape at this stage, at a very fine resolution of 
          $ffin \simeq 0.0005 cm^{-1}$. The results of this computation 
          are then boxcar averaged to the output resolution $nbox \times 
          ffin \simeq 0.0025 cm^{-1}$ \\
          Thus the lines that are used in this stage have their centers 
          spanning 1007 to 1010 $cm^{-1}$.

\item (b) {\it medium mesh stage} : lines that are an additional 
          $xmed - xnear \simeq (2 - 1) = 1 cm^{-1}$ on either side of the 
          edges of this bin, are now used in computing 
          the overall lineshape, at a medium resolution of 
          $fmed \simeq 0.1 cm^{-1}$. The results of this computation 
          are then splined onto the output resolution and added onto the
          running sum from above.
          Thus the lines that are used in this stage have their centers 
          spanning (1006,1007) and (1010,1011) $cm^{-1}$.

\item (b) {\it coarse mesh stage} : lines that are an additional 
          $xfar - xmed \simeq (25 - 2) = 23 cm^{-1}$ on either side of the 
          edges of the medium meshes, are now used in computing 
          the overall lineshape, at a coarse resolution of 
          $fcor \simeq 0.5 cm^{-1}$. The results of this computation 
          are then splined onto the output resolution and added onto the
          running sum from above.
          Thus the lines that are used in this stage have their centers 
          spanning (983,1006) and (1011,1034) $cm^{-1}$.
\end{itemize}

The speed up in the code is gained by computing contributions at the 
medium and coarse resoutions as much as possible, instead of using the 
fine resolution grid all the time.

As was pointed out above, the strengths of some of the lines sometimes
sometimes make it necessary to use lines that could be upto 200 $cm^{-1}$ 
away (in the coarse part of the computation). Instead of doing this, a 
common practice is to remain with the 25 $cm^{-1}$ width of the coarse 
meshes, but include the effects of lines outside these coarse mesh by adding
on an extra {\it continuum}. This is true for oxygen and nitrogen, and in
particular water vapor (see below). Carbon dioxide also requires this; 
however we choose not to use the continuum in this case, but simply use the
entire 200 $cm^{-1}$ wide coarse meshes.

Carbon dioxide, like water vapor, is radiatively active in the Earth's 
atmosphere. Just like water, it is important to accurately compute the
absorption spectra using the correct lineshape, as optical depths of these
gases vary greately in the atmosphere, allowing one to probe several 
layers of the atmosphere within a small spectral region. A thorough 
understanding of the absorption spectra of these two gases is therefore 
very important for the remote sensing community.

\subsection{Computing water vapor absorption coefficients}
Almost all that has been said above remains valid when computing the 
lineshape of water vapor. However, the lineshape far away from line 
center is sub-Lorentz ($k/k_{lor} \leq 1$), while the lineshape close to 
line center is super Lorentz ($k/k_{lor} \geq 1$).  To account for this 
behavior, the above algorithm has to be slightly modified. This leads to 
the water vapor lineshape algorithm to include the following three 
different modifications, put together : 
\begin{itemize}
\item Instead of using a lorentz lineshape, a {\it local lineshape} is used.
      Upto 25 $cm^{-1}$ away from line cenetr, this is defined as 
      lorentz less the lorentz value \@ $25 cm^{-1}$, and zero everywhere 
      else.
\item To include the super Lorentz behavior close to the line center, 
      the local lineshape is then multiplied by a chi ($\chi$) function
\item To include the sub Lorentz behavior far from the line center, 
      a continuum function is then added on. This is done after the 
      effects of all lines have been used. Another way of thinking of this 
      continuum is to say that the water lines are quite strong, and thus
      the computational algorithm should not restricted to using only lines
      that are at most 25 $cm^{-1}$ wavenumbers away from the spectral 
      region under consideration. But instead of individually using lines 
      that could be upto 200 $cm^{-1}$ away (and modelling their
      sub Lorentz far wing behavior), the far wing effects of these far 
      lines are all lumped into the continuum.
\end{itemize}

To summarize, when computing a water vapor optical depth, the code proceeds
as in the general case described above, except that it uses a {\it local} 
lineshape for each line $j$, multiplied by a $\chi$ function. After all the 
necessary lines have been included, a {\it continuum} absorption coefficient
is also added on : 

\[
k(\nu) = k_{continuum}(\nu) + \sum_{j} k_{local}(\nu,\nu_{j}) \chi(\nu)
\]

which can be rewritten, for the individual lines $j$
\[
k_{local}(\nu,\nu_{j}) = 
\left\{
\begin{array}{cl}
     ( k_{lorentz}(\nu,\nu_{j}) - k_{lorentz}(\nu,25+\nu_{j}) ) 
           \chi(\nu)    & \mbox{if $|\Delta\nu| \leq 25 \mbox{cm}^{-1}$} \\
        0               & \mbox{if $|\Delta\nu| > 25 \mbox{cm}^{-1}$}
\end{array}
\right. 
\]
where appropriate factors of $\nu\tanh\left(\beta\nu/2\right)$ 
multiply the above coefficients. 

The computation for $k_{local}$ proceeds as described in the previous 
section, {\em viz.} using fine, medium and coarse meshes.

\subsection{Computing carbon dioxide absorption coefficients}

Computing the spectral lineshape of carbon dioxide can be quite complicated.
There are many bands within which there are lines that are very closely 
spaced. Collisions have the effect of mixing these lines together,
transferring intensity from the line wings to the line centers. Furthermore,
the  collisions are not instantaneous, but have a finite duration. This 
also makes the lineshape deviate from Lorentz, especially far from the line
centers. Additionally, some of the bands are very strong and have an 
effect on the absorption spectrum, at
quite large distances from their band (line) centers. This third effect can
be accounted for by either using a continuum, or by allowing the inclusion
of effects of lines that are upto 200 $cm^{-1}$ away from the region of 
interest.

\subsubsection{Line Mixing : probability of mixing (parameter $\beta$)}

Deviations from the Lorentz lineshape in regions of overlapping spectral
lines have been observed in many cases. In particular, large deviations are
found in infrared Q-branches, where the spectral lines are very closely 
spaced.  Theories that treat the collisions between molecules as 
instantaneous will be accurate only in spectral regions close to the 
line centers.  

Within an ensemble of colliding molecules, the Hamiltonian of a single 
molecule is not a conservative system.  The total Hamiltonian 
of the molecule is given by $H(t)=H_{0}(t)+ H_{1}(t)$.  In this
expression, $H_{0}(t)$ is the Hamiltonian of the molecule without any
interaction with its perturbers.  $H_{0}(t)$ therefore has real eigenvalues
and its eigenvectors are the stationary states of the molecule.  
$H_{1}(t)$ is the Hamiltonian representing the interaction of the molecule
with its perturbers, such as the effects of inelastic collisions.  
In the context of line mixing of a band, the eigenvalues of $H_{0}(t)$ give
the energies (line centers) of the individual transitions within a band.

For a given band, if the off-diagonal elements of $H_{1}$ are zero, the 
lineshape that results is a sum of individual Lorentzians : the model obeys 
the impact approximation.  The role of the off-diagonal elements of the 
interaction potential potential is the interaction between spectral 
transitions. If these off-diagonal elements of  $H_{1}$ are non-zero, 
intensity can be transferred from one line to another, 
with the ``amount'' of line-mixing determined by the magnitude of the 
corresponding off-diagonal element of $H_{1}$.  These elements are
proprtional to the {\bf probability $\beta$} that spectral intensity is 
transferred from one line to another. If $\beta = 0$, there is no line 
mixing and the lineshape that results is Lorentzian; if $\beta \ne 0,$ 
there is line mixing, and the lineshape deviates from a sum of Lorentzians.

Calculating an absorption coefficient for many transitions over a
large spectral range is computationally most efficiently performed using 
\[
k_{mix}(\nu)=\frac{N}{\pi} {\bf IM} \left({\bf d}\cdot{\bf G}({
\bf
  \nu})^{-1} \cdot{\bf \rho}\cdot{\bf d}\right)
\]
where ${\bf G}={\bf \nu}-{\bf H}$ and 
${\bf H}={\bf \nu_{0}}+\imath P{\bf W}$, where $\nu_{0}$ are the 
eigenvalues of $H_{0}$ and $H_{1} = \imath P{\bf W}$ where $P$ is the 
pressure and $\imath {\bf W}$ is the interaction matrix. 
 
${\bf H}$ is diagonalized with a complex matrix ${\bf A}$ to get the 
diagonal matrix ${\bf L}={\bf A^{-1}}\cdot{\bf H}\cdot{\bf A}$.
${\bf G}$ is also diagonalized by ${\bf A}$ and $k_{mix}(\nu)$ is written
as
\[
k_{mix}(\nu)=\frac{N}{\pi} {\bf IM} \left(\sum_{i}\frac{({\bf d
\cdot
    A})_{i} ({\bf A^{-1}\cdot\rho\cdot d})_{i}}{\nu - l_{i}}\right)
\]
where $l_{i}$ are the diagonal elements of ${\bf L}$.  

Using time-independent perturbation theory, Rosenkranz found the first-order 
approximation for $k_{mix}(\nu)$ to be 
\[
k_{1st}(\nu)=\frac{N}{\pi}\sum_{j} S_{j}
\left(\frac{P \gamma_{j}+ 
(\nu-\nu_{j})PY_{j}}{(\nu-\nu_{j})^{2}+(P\gamma_{j})^{2}}\right)
\; \; \mbox{with} \; \; 
Y_j=2\sum_{k\neq j}\frac{d_k}{d_j}\frac{W_{kj}}{\nu_j-\nu_k}
\]
where 
%$N$ is the absorber density, $P$ is the pressure, $S_j$, $\nu_j$, and
%$\gamma_j$ are the line strength, center and half-width, and 
$\mbox{Y}_j$ are first-order mixing coefficients.  For a single transition, 
this lineshape is the sum of a Lorentzian and an asymmetric term.  Far from
the line centers, the asymmetric terms become proportional to $\nu^{-1}$.  
In order for $k_{1st}(\nu)$ to go to zero in these regions, the sum of the
coefficients must vanish.  That is, detailed balance must be obeyed.  In 
this context, Strow and Reuter\cite{str:88*1} showed that detailed balance 
is obeyed if
\[
 \sum_{j} S_{j} Y_{j}=0.
\]
They used this result to show that, in the far-wing limit, the 
ratio of mixing and Lorentz absorption coefficients is a
constant\cite{str:88*1}
\[
 \frac{k_{1st}}{k_{L}}= 1 + \frac{\sum_{j}S_{j}Y_{j}\nu_{j}}
{\sum_{j}S_{j}\gamma_{j}}
\]
This is a useful result because it allows the mixing lineshape to be
calculated by simply multiplying the Lorentz lineshape by a constant in
the far wing.

The only point left to be addressed is the determination of the 
off-diagonal, or mixing, terms of ${\bf W}$. This will be described in 
more detail below.

\subsubsection{Determining the Interaction Matrix ${\bf W}$}

Empirical scaling laws based on energy changes caused by inelastic 
collisions are often used to model the interactions.  The PEG law models the
energetically upward state-to-state inelastic collisional rates as a 
function of the rotational energy difference, $\Delta E_{j'j}$.  An upward 
rate going from the state $j$ to state $j'$ is modeled as 
\[
 K_{j'j}=a_{1}\left(\frac{\mid\Delta E_{j'j}\mid}{B_{0}}\right)^{-a_{2}}
\exp\left(\frac{-a_{3}\mid\Delta E_{j'j}\mid}{kT}\right)
\]
where $B_{0}$ is the rotational constant and $a_{1}$, $a_{2}$, and $a_{3}$
are adjustable parameters which are discussed below. 

Detailed balance is obeyed if
\[
K_{jj'}(2j'+1)e^{-\frac{E_{j'}}{kT}}=K_{j'j}(2j+1)e^{-\frac{E_{j}}{kT}}
\]
This relation essentially ensures that energy is conserved and gives the
downward rates, $K_{jj'}$:
\[
K_{jj'}=K_{j'j} \frac{2j+1}{2j'+1} e^{\frac{\Delta E}{kT}}
\]

$a_{1}$, $a_{2}$, and $a_{3}$ above are determined 
by equating the width of a spectral line to the sum of all of the rates
which limit the lifetime of that transition via a least-square fit to the
known linewidths, as any rates which shorten the molecule's lifetime in a 
specific energy state broadens the spectral line. These rates include all 
of those which occur in either the lower or upper state of the transition. 
Since vibrational energies are much greater than rotational energies, only
collisions between states within the same vibrational level are considered.
Therefore, using line widths extracted from experimental data, $a_{1}$,
$a_{2}$, and $a_{3}$ are determined by requiring
\[
\gamma_{j} \equiv {\bf W_{jj}} = \sum_{j'\neq j}
\mbox{All} \; K_{j'j} \; \mbox{which limit the transition lifetime}.
\]
{\em The details of the exact expression used in calculations
depend on the type of transition which is occurring,} such as Q branch 
mixing for a $\Sigma-\Pi$ transition.

The off-diagonal elements of the matrix ${\bf W}$ are then taken to be 
proportional to the corresponding collisional rates of the matrix 
${\bf K}$.  {\it These are the mixing terms.} 
For two rotational levels which are energetically close, the collisional 
rate between them is relatively large and they experience mixing.  On the 
other hand, if two levels are energetically far from each other, the 
corresponding ${\bf K}$ rates are negligible and no mixing occurs.

Summarizing the calculational procedures, the relaxation rates, $K_{jj'}$, 
are first determined by adjusting $a_{1}$, $a_{2}$,
and $a_{3}$ so that the sum of all relaxation rates which limit the
lifetime of a transition equals the known line width.  
The off-diagonal elements of ${\bf W}$ are then taken to be proportional to 
the corresponding off-diagonal elements of ${\bf K}$.  The details of this 
step depend on the symmetry of the band.

The diagonal elements of ${\bf W}$ are equated to the line widths.
The absorption coefficients are then calculated using 
full or first order mixing.  

\subsubsection{Duration of collisions : Birnbaum chi function ($\tau$ 
parameter)}
The Lorentz model ignores the effects of finite durations of collisions, and
so has a larger intensity far from line center than that actually measured. 
To account for the finite duration of collisions, Birnbaum developed a 
theory which resulted the lineshape being described by a chi function 
mutiplying the Lorentz lineshape (see Eqn. \ref{eqn:birnbaum} 
\[
k_{B}(\nu)=k_{L}(\nu)\chi_{B}(\nu)=k_{L}(\nu) A_{m} z K_{1}(z)\exp\left(
\tau\gamma+\tau_{0}\Delta\nu\right)
\]
with
\[
z=\sqrt{(\gamma^{2}+\Delta\nu^{2})(\tau_{0}^{2}+\tau^{2})}
\; \; \mbox{and} \; \;
\Delta\nu=\nu-\nu_{j}
\]
where $K_{1}(z)$ is a modified Bessel function of the second kind, 
$\tau_{0}=\frac{0.72}{T}$ \footnote{$\tau_{0}$ and $\tau$ have been
converted to units of $\mbox{cm}$ by multiplication by $2\pi\mbox{c}$.}, 
and A$_{m}$ is an adjustible constant (set to 1.0 in the code).  $T$ is the 
temperature, $\gamma$ the linewidth and and $\tau$ is the duration of 
collision parameter.

This ``corrective'' factor, $\chi_{B}(\nu,\gamma,\tau_{0}(T),\tau)$, 
removes much of the far-wing absorption of the impact approximation for the 
Lorentz line shape. For the parameters used in the code (Earth atmosphere),
$\chi_{B}$ is only very weakly dependent on the linewidth $\gamma$.  
When computing the Birnbaum function, our code uses a look up table to do 
an interpolation in temperature/duration-of-collision.

\subsubsection{Combined Line-Mixing and Duration-of-Collision Lineshape}
Both line-mixing and the duration-of-collision effect have shown to reduce
the amount of absorption in the far-wing limit.  Since the line-mixing
theory is valid only under the impact approximation, the combined effects
of line-mixing and duration-of-collision are approximated as if each
effect were independent.  In order to include the effects of line-mixing
over the entire frequency range, we have 
\[
k(\nu)=\sum_i k_{1st}(\nu_i,\nu)\chi_{B}(\nu_i,\nu)
\]
The full line-mixing lineshape $k_{mix}(\nu)$ can also be implemented by 
using 
\[
k(\nu)=\frac{k_{mix}(\nu)}{\sum_i k_{Lor}(\nu_i,\nu)}
\sum_i k_{Lor}(\nu_i,\nu)\chi_B(\nu_i,\nu)
\]
if the first-order approximation is too inaccurate (which is seldom true 
for atmospheric applications). Note that this lineshape is parameterised
by only two parameters : $\beta$ which is the probability that a collision
will transfer intensity from one transition to another, and $\tau$ which
is the duration of collsions parameter.

\subsection{Cousin versus Birnbaum chi functions}
To model the CO2 spectrum in the 4 $\mu$m region, Cousin developed a set of
empirical chi functions, with many parameters to account for varying
temperatures, broadeners and so on. Multiplying the Lorentz lineshape
with the suitable Cousin chi function can model the experimental spectrum,
epsecially far away from band center and for low optical depths. The Cousin
chi functions combine the physics of line mixing and finite duration of 
collisions. However, these Cousin functions are also used in the 15 $\mu$m
region of CO2 as well. As mentioned above, the $\beta$ factor in this region
is 0.5, about half that in the 4 $\mu$m region. Thus it is incorrect to use
the Cousin functions in the 15 $\mu$m region. There will be too much 
intensity transferred to line centers, which means the absorption in 
between lines is too weak, leading to lower optical depths and higher 
transmissions. As remote sensing algorithms frequently use the information
``in between'' lines to retrieve atmospheric properties, this could 
certainly be a source of errors in the retrievals.

If the user asks for ``birnbaum'' chi function to be used, our code 
``blends'' in the Cousin and Birnbaum chi functions. Close to the 
lines (typically within 25 $cm^{-1}$ for the strong bands, and 
within 1-2 $cm^{-1}$ for the weaker bands), the Birnbaum chi function is 
used. Outside of this region, the Cousin function is turned on gradually, 
till it is fully on a distance 1 $cm^{-1}$ away from this edge. This 
birnbaum chi function can be used whether or not line mixing is turned 
on or not.

If the user asks for ``cousin'' chi function to be used, our code just 
turns on the Cousin chi function only. Because the Cousin function includes
the effects of line mixing, this feature can only be used if line mixing is 
turned off.

\subsection{Computing the line-mixing lineshape}

We have identified many of the strong P,Q,R bands in the 4 and 15 $\mu$m 
regions, where line mixing computations are performed. For all these bands, 
the code used to compute the line mixing lineshape is almost the same and
is therefore common. As mentioned earlier, the main difference comes in the
setting up of the interaction matrices ${\bf W}$, whiere the details of the
symmetry peculiar to the band under consideration have to be explicitly
implemeted.

In this section, we outline the steps in the computation of the $\Sigma-\Pi$
{\bf Q} branch lineshape (eg {\bf Q 667}). Almost all the steps will be 
applicable to he other bands.

Given the gas amount $Q$ in $kilomoles/cm^{2}$, the self pressure and 
temperature, the code first finds the path length $L$. 

\subsubsection{Loading in line parameters, storing $\beta, \tau$}
It then calls subroutine $loader$ which performs a number of 
initializations. For instance, it computes the line mixing strength
\[
\beta = \frac{\beta_{self}p_{self} + \beta_{for}p_{for}}{p_{total}}
\]
where $p_{total} = p_{self} + p_{for}$. Similarly it computes the duration 
of collision parameter (usually not needed for {\bf Q} branches, as the 
lines are all very close together)
\[
\tau = \frac{\tau_{self}p_{self} + \tau_{for}p_{for}}{p_{total}}
\]
The code then initialises other constants, and loads in the relevant
HITRAN parameters for the band in question, such as line strengths and 
widths. It then adjusts the line widths for the pressures and temperatures
in question, as well as initialising the partition fiunctions and computing
the temperature adjusted line strengths. For the given band, the rotational
quantum indices and associated parameters (widths, strengths etc) are all
ordered in terms of increasing $J$, using routine $orderer$.

\subsubsection{Computing lower state rotational energies $E_{lower}$}
Routine $efitter$ is then called, to compute the rotational energies of any 
missing levels. When interaction matrix ${\bf W}$ is built up, a 
complete knowledge of lower and upper state rotational energies is required,
so that state to state collision rates can be computed. However, this
may require the code to know the rotational energies of some missing levels.
To circumvent this, a three parameter least squares fit of the lower
state energies $elower$ versus rotational number $j$ is performed : 
\[
E_{lower} = B \times j(j+1) - D \times (j(j+1))^{2} + E_{vib}
\]
For the ${\bf Q 667}$ branch, the even $j$'s from 2 to 102 are stored on the
HITRAN tape, as are the associated lower state energies $E_{lower}$
The least squares fit thus gives an estimate for parameters $B,D,E_{vib}$.
Having obtained these parameters, using the above equation the code can 
easilty compute the energies of missing levels if necessary -- in this 
case, the odd $j$'s and $j=0$. The final step is to save in memory the
rotational energies $B \times j(j+1) - D \times (j(j+1))^{2}$ for the 
$j=0,1,2,3,...$

\subsubsection{Computing relaxation matrix ${\bf K}$  using PEG law}
Now comes the set of routines which actually use the above information to 
set up the interaction matrix ${\bf W}$. As mentioned earlier, the physics
of the band in question is used in setting up this matrix. Having computed
the rotational energies of the lower states as described above, a set of 
routines now fit the foreign broadening widths $\gamma_{for}(j)$ to the 
three parameters $a_{1}^{for},a_{2}^{for},a_{3}^{for}$ of the Power 
Exponential Gap (PEG) equations, and then similarly the self broadening 
widths $\gamma_{self}(j)$ to the corresponding three  parameters 
$a_{1}^{self},a_{2}^{self},a_{3}^{self}$. The procedure is described below.

As mentioned earlier, the PEG law models the energetically upward 
state-to-state inelastic collisional rates as a function of the rotational 
energy difference, $\Delta E_{j'j} = E_{j'} - E{j}$.  First set up matrix 
${\bf \Delta E}$ which governs the energy transitions from lower state 
$j$ to upper state $j'$ :
\[
{\bf \Delta E} = {E_{j'j}} = 
\left(
\begin{array}{cccc} 
         0    & E_{j2}-E_{j1} & E_{j3}-E_{j1} & E_{j4}-E_{j1} \\
E_{j1}-E_{j2} & 0             & E_{j3}-E_{j2} & E_{j4}-E_{j2} \\
E_{j1}-E_{j3} &E_{j2}-E_{j3}  & 0             & E_{j4}-E_{j3} \\
E_{j1}-E_{j4} &E_{j2}-E_{j4}  & E_{j3}-E_{j4} & 0 
\end{array}\\
\right)
\]
Note that if one takes the lower triangular diagonal elements of the above
matrix, one has the lower to upper state transitions $j \rightarrow j'$, 
while the upper triangular diagonal elements of the above
matrix yield the upper to lower state transitions $j' \rightarrow j$. 

Using matrix ${\bf \Delta E}$, the matrix containing the upward relaxation 
rates going from lower state $j$ to upper state $j'$ is then
\[
 K_{j'j}=a_{1}\left(\frac{\mid\Delta E_{j'j}\mid}{B_{0}}\right)^{-a_{2}}
\exp\left(\frac{-a_{3}\mid\Delta E_{j'j}\mid}{kT}\right)
\]
where $B_{0}$ is the rotational constant and $a_{1}$, $a_{2}$, and $a_{3}$
are adjustable parameters. 

Detailed balance is obeyed when the upward transition rate is equal to the
downward transition rate, giving the downward rates in another matrix :
\[
K_{jj'}=K_{j'j} \frac{2j+1}{2j'+1} e^{\frac{\Delta E}{kT}}
\]

Having constructed matrix $K_{j'j}$, {\it we can now construct the 
relaxation rate matrix ${\bf K}$} to fit for parameters $a_{l}, l=1,2,3$. 
This is most conveniently done using the lower and upper tridiagonals 
of ${\bf \Delta E}$ and the detailed balance relations : 
\[
K_{j'j} \rightarrow K_{j'j} + K_{jj'} 
\]
\[
K_{j'j} \rightarrow K_{j'j} + 
               K_{j'j} \frac{2j+1}{2j'+1} e^{\frac{\Delta E(j'j)}{kT}}
\]
This can be eloquently combined in Matlab as 
\begin{verbatim}
K=tril(K,-1)+triu(K,1).*exp(btz*energy_diff/temperature).*J'./J;
\end{verbatim}
The upper diagonals have the downward relaxation rates, while the upper 
relaxation rates are in the lower diagonals. This matrix now has all the 
elements that can interact with one another, as well as extra elements 
which may not be needed (remember $efitter$ precomputed the rotational 
energies of any missing $j$ levels, in case they are required in what 
follows).

Recall that the relaxation rates ${\bf K_{jj'}}$ are determined
by adjusting $a_{1},a_{2},a_{3}$, so that the sum of the relaxation rates
that limit the lifetime of a transition equal the linewidth. 
Upto this point, the discussion above has been quite general and applicable
to all the bands. Now however, the particular details of the band become 
important. For the $\bf{Q 667}$ branch, the line widths are given by
\[
   W_{jj}\equiv\gamma_{j}=-\frac{1}{2}\left(
   \sum_{j'\neq j}^{even}K_{j'j}^{\Sigma(e\leftarrow e)}\right)-
   \frac{1}{2}\left(
   \sum_{j'\neq j}^{even}\beta K_{j'j}^{\Pi(f\leftarrow f)}+
   \sum_{j'\neq j}^{odd}(1-\beta)K_{j'j}^{\Pi(e\leftarrow f)}\right)
\]
where the $\beta$ probability factor, introduced by Strow and
Edwards \cite{edw:91} differentiates between the collisional relaxation
rates which connect states of similar and opposite rotational
symmetry. As mentioned earlier, $\beta$ is a measure of the probability that
a collision transfer intensity from one transition to another. For each of 
the bands for which line mixing is done, we have experimentally 
determined $\beta$. 
For the ${\bf Q 667}$ band, the superscript on $K_{j'j}$ denotes the 
symmetry of the levels involved.  In the $\Pi$ vibrational level, the 
relaxation rates have
been divided into two groups.  Those which connect rotational levels of
similar symmetry are multiplied by $\beta$ and those which connect levels
of opposite symmetry are multiplied by $(1-\beta)$.  The $\beta$ factor is
not needed in the $\Sigma$ state simply because the odd rotational levels
do not exist there. For the ${\bf Q 667}$ $\Sigma-\Pi$ band, a cartoon of 
the allowed transitions, and the lines
to which they can mix, is shown in Fig. \ref{fig:sigpi_krates}.

\begin{figure}[h]
\begin{center}
\includegraphics[width=3.5in]{Figures/CO2/sigpi_krates.eps}
\end{center}
  \caption[Rotational relaxation rates, $K_{j'j}$, for several Q-branch
        lines of a $\Sigma\leftarrow\Pi$ transition.]
        {Rotational relaxation rates, $K_{j'j}$, for several Q-branch
        lines of a $\Sigma\leftarrow\Pi$ transition.  In our line-mixing
        model, $\Pi$ rates connecting rotational levels of similar parity
        are multiplied by $\beta$ and rates connecting levels of opposite 
        parity are multiplied by ($1-\beta$).} 
  \label{fig:sigpi_krates}
\end{figure}

The user is referred to Tobin's thesis \cite{tob:96} for implementation 
details of the other bands. 

\subsubsection{Computing the interaction matrix ${\bf W}$}
By using a least squares fitting procedure, parameters $a_{1},a_{2},a_{3}$
can be adjusted so that the sum of the relaxation rates
on the RHS of the previous equation are almost equal to the 
actual line width $W_{jj}\equiv\gamma_{Qj}$. Using these fitted parameter 
values, the off diagonal elements of $K_{j'j}$ are recomputed using the 
PEG scaling law
\[
 K_{j'j}=a_{1}\left(\frac{\mid\Delta E_{j'j}\mid}{B_{0}}\right)^{-a_{2}}
\exp\left(\frac{-a_{3}\mid\Delta E_{j'j}\mid}{kT}\right)
\]
while the diagonal elements are equated to the linewidths $K_{jj} = 
\gamma_{j}$. This is done separately for the self and foreign broadened 
widths. For the pressures under consideration, the complete relaxation 
matrix  is then a weighted sum of the two relaxation matrices
\[
{\bf K} = \frac{{\bf K_{for}}p_{for} + {\bf K_{self}}p_{self}}
                {p_{for}+p_{self}}
\]

The diagonal elements of interaction Hamilitonian are the linewidths of 
${\bf K}$. If there is no line mixing, the mechanics of the algorithm
will just give a Lorentzian lineshape.

The off diagonal elements of the perturbation or interaction matrix are 
then simply proportional to the off diagonal elements of the relaxation 
matrix; if there is no line mixing ($\beta = 0$), these elements are zero.

Combining the above two Hamiltonians, the interaction matrix 
${\bf W} \simeq H_{1}$ is
\[
{\bf W} = diag({\bf K}) + \beta \times offdiag({\bf K})
\]

\subsubsection{Computing the transition population amplitudes}
In order to compute the full line mixinfg lineshape, a knowledge of the
transition population amplitudes is necessary. These amplitudes are computed
in $trans\_pop$

The transition amplitude is computed using 
\[
d = \sqrt{\frac{S}{\rho}}
\]
where $d$ is the transition amplitude (which is the square root of the
transition probability), $S$ is the line strength and $\rho$ is the 
population.

For the isotope in questions, the partition function coeffeicients are
read in $a,b,c,d$ (obtained from qtips.f). This enables the computation of 
the partition functions both at the reference and actual temperatures,
$Z(296), Z(T)$. For a line of rotational quantum number $j$, centered 
at $\nu_{0}(j)$ with lower rotational energy $E_{lower}(j)$ and strength 
$S_{j}$, the reference temperature transition amplitude $d$ is
\[
\alpha(296) = \frac{8 \pi^{3} \nu_{0}(j) (1-e^{-k\nu_{0}(j)/296})}
                    {0.3 \times 6.626176e^{-34} \times 2.99792458e^{10}}
\]
\[
\beta(296) = (2j+1) \times e^{-k E_{lower}(j)/296} 
                     \times 1e^{-36}/Z(296)
\]
\[
d = \sqrt {\frac{S_{j}}
                    {\alpha(296) \times \beta(296) \times 10^{-7}}}
\]

The population $\rho$ at the desired temperature can similarly be computed 
as
\[
\alpha(T) = \frac{8 \pi^{3} \nu_{0}(j) (1-e^{-k\nu_{0}(j)/T})}
                    {0.3 \times 6.626176e^{-34} \times 2.99792458e^{10}}
\]
\[
\beta(T) = (2j+1) \times e^{-k E_{lower}(j)/T} 
                     \times 1e^{-36}/Z(T)
\]
\[
\rho = \alpha(T) \times \beta(T) \times 10^{-7}
\]

\subsubsection{Computing the full mixing lineshape}
As mentioned earlier, calculating an absorption coefficient for many 
transitions over a large spectral range is performed using 
\[
k_{mix}(\nu)=\frac{N}{\pi} {\bf IM} \left({\bf d}\cdot{\bf G}({
\bf
  \nu})^{-1} \cdot{\bf \rho}\cdot{\bf d}\right)
\]
where ${\bf G}={\bf \nu}-{\bf H}$ and 
${\bf H}={\bf \nu_{0}}+\imath {\bf W}$, where $\nu_{0}$ are the 
eigenvalues of $H_{0}$ and $H_{1} = \imath {\bf W}$ where the interaction
matrix $\imath {\bf W}$ already includes the pressure effects.
 
${\bf H}$ is diagonalized with a complex matrix ${\bf A}$ to get the 
diagonal matrix ${\bf L}={\bf A^{-1}}\cdot{\bf H}\cdot{\bf A}$.
${\bf G}$ is also diagonalized by ${\bf A}$ and $k_{mix}(\nu)$ is written
as
\[
k_{mix}(\nu)=\frac{N}{\pi} {\bf IM} \left(\sum_{i}\frac{({\bf d
\cdot
    A})_{i} ({\bf A^{-1}\cdot\rho\cdot d})_{i}}{\nu - l_{i}}\right)
\]
where $l_{i}$ are the diagonal elements of ${\bf L}$.  

The complete Hamiltonian is thus written as
\[
H = H_{0} + H_{1} = diag(v_{0}) + \imath W
\]
from which the eigenvalues ${l_{j}}$ and eigenvectors ${\bf A_{j}}$ of $H$ 
are found. Having done this, and having already computed the transition
amplitudes $d_{j}$ (dipole moment matrix emlements) and density $\rho_{j}$
(matrix elemtns that represent population difference between lower and 
upper levels), it is now straightforward to loop over the individual lines
and compute the total lineshape. Note that if necessary, a birnbaum $\chi$
function would have been mutiplied into each contribution : 
\[
k_{mix}^{j}(\nu)=\frac{N}{\pi} {\bf IM} \left( \frac{({\bf d
\cdot
    A})_{j} ({\bf A^{-1}\cdot\rho\cdot d})_{j}}{\nu - l_{j}}\right)
    \times \chi_{birnbaum}(\gamma_{j},T,p,ps)
\]
\[
k_{mix}(\nu) = \sum_{j} k_{mix}^{j}(\nu)
\]
where $N = \frac{ps}{p_{ref}} \frac{T_{ref}}{T} \frac{L\mu}{\pi}$, $L$ 
being the path length and $\mu$ the molecular density. Note that we have 
chosen to directly include the effects of the Birnbaum chi function as 
shown, instead of using an additional factor of
\[
k_{mix}(\nu) = \left( \sum_{j} k_{mix}^{j}(\nu) \right)
               \left( \frac{\sum_{j} k_{lor} \chi_{birnbaum}}
                           {\sum_{j} k_{lor}} \right)
\]

Recall that if there is no linemixing, the lineshape should just be a sum
of Lorentzians. Thus if $H_{1}$ has no off diagonal elements ie its elements
consists entirely of the linewisths on the diagonal, then the eigenvalues of
$H$ would be the line centers $v_{0}(j)$, and the spectrum would end up 
being a sum of Lorentzians, with line center $v_{0}(j)$ and line width
$\gamma_{j}$. If there $is$ linemixing, then the linecenters would still
be close to $v_{0}(j)$, but the lineshape would deviate from a sum of 
lorentzians.

\subsubsection{Computing the first order mixing lineshape}
The first-order approximation for $k_{mix}(\nu)$ is
\[
k_{1st}(\nu)=\frac{N}{\pi}\sum_{j} S_{j}
\left(\frac{ \gamma_{j}+ 
(\nu-\nu_{j})Y_{j}}{(\nu-\nu_{j})^{2}+(\gamma_{j})^{2}}\right)
\; \; \mbox{with} \; \; 
Y_j=2\sum_{k\neq j}\frac{d_k}{d_j}\frac{W_{kj}}{\nu_j-\nu_k}
\]
where, as described above, $N = \frac{ps}{p_{ref}} \frac{T_{ref}}{T} 
\frac{L\mu}{\pi}$, $S_{j}$ is the line strength, and the mixing coefficients
$Y_{j}$ are obtained from the interaction matrix ${\bf W}$ and transition
amplitudes $d_{j}$.

\section{run6}

The Matlab code has three main driver files : $run6.m,$ $run6co2.m$ and 
$run6water.m.$ run6.m is a general code that will work for all gases; 
however 
one should run the specialized code for water and CO2, so as to to utilise 
the above physics in the computed lineshapes. We describe the $run6.m$ 
program parameters and algorithm in detail below; in the next two sections, 
we will discuss the corresponding similarities and differences for the 
water and carbon dioxide codes.

The input argument list to the code contains the name of a profile file 
which specifies the layer number, total pressure, gas partial 
pressures (both in atm), gas temperature (in Kelvin) and gas amount 
(in $kilomolecules cm^{-2}$). The profile should be in a 5 column format, 
and should be a text file.

\subsection{Mex files and HITRAN database}
(this is for run5) \\
If the user wants to change the name of the line database file that is used,
he/she will have to go into $hittomat2.$ and change the name of the file in
the line beginning with the word $INFILE$ :
\begin{verbatim}
INFILE='Hitlin/hitlin98.bin';
\end{verbatim}

(this is for run6) \\
If the user wants to change the name of the line database file that is used,
he/she will have to go into the run6* files and change the name of the 
file in the line beginning with the word $fnamePRE$, which is currently 
set to :
\begin{verbatim}
fnamePRE='/salsify/scratch4/h98.by.gas/g';
\end{verbatim}

To speed the code up, a number of loops have been written as fortran MEX 
files.
All these files are in subdirectory FORTRANFILES, and assume input 
arrays/matrices that are smaller than certain limits. If the user wants to 
change these limits, he/she will have to edit the file $max.inc$ and 
recompile the Mex files.

\begin{verbatim}
c this is max length of arrays that can be used in the Mex Files  
c this number came out of 
c   200000 = max number of elements in mesh 
c              eg (755-655)/0.0005 = 160000
c        4 = number tacked on to arrays so boxint(y,5) can be done  
      integer MaxLen
      parameter(MaxLen=200010)

c assume max number of any of P,Q,R lines = 300
      integer MaxPQR
      parameter(MaxPQR=300)

c assume max number of any of layers = 100
      integer kMaxLayer
      parameter(kMaxLayer=100)
\end{verbatim}

To compile the Mex files, the user has to type $makemex1$ at the UNIX 
prompt 
(if only $run6.m$ is being used), or type $makemex$ (if $run6co2.m$ will be 
used). This compiles all the Mex files, and creates symbolic links to these
files from the necessary subdirectories.

If the user is going to use $run6co2.m$, he/she will also need to go to the 
C02\_COMMON subdirectory, and type $link.sc$ so that symbolic links from the
CO2 subdirectories to the common files are created.

\subsection{Water, nitrogen, oxygen continuum}
Through parameter $CKD$ (see below), the user can toggle the continuum 
calculation on/off for three gases : water, oxygen and nitrogen 
(gasIDs 1,7,22 respectively). 

\begin{itemize}
\item Water : CKD can be set to -1 (no continuum), or 0,21,23 for CKD 
              versions 0, 2.1, 2.3. Note when these versions of CKD 
              are included, the computation proceeds by using the code which
              does not require a ``local'' lineshape.
\item Oxygen : CKD can be set to -1 (no continuum), or +1 (continuum)
\item Nitrogen : CKD can be set to -1 (no continuum), or +1 (continuum)
\item For all other gases, the value of $CKD$ is irrelevant
\end{itemize}

\subsection{run6.m input parameters}

A typical call to run6 would involve sending in the following : 

$[outwave,outarray]=run6(gasID,fmin,fmax,ffin,fmed,fcor,$\\
         $fstep,xnear,xmed,xfar,nbox,strfar,strnear,LVG,CKD,profile)$

where the right hand side variables would be 

\begin{longtable}{llll}
  TYPE  &   VAR  &         DESCRIPTION  &            TYPICAL VALUE\\
\hline
integer & gasID  &       HITRAN gas ID      &            3\\
\hline
integer & fmin    &      minimum freq (cm-1) &          605\\
integer & fmax    &      maximum freq (cm-1) &          630\\
\hline
real   &  ffin    &      fine point spacing (cm-1) &    0.0005\\
real   &  fmed    &      medium point spacing (cm-1)&   0.1\\
real   &  fcor    &      coarse point spacing (cm-1)  & 0.5\\
\hline
real   &  fstep   &      wide mesh width size (cm-1) &    1.0\\
real   &  xnear   &      near wing distance(cm-1)    &    1.0\\
real   &  xmed    &      med wing distance(cm-1)     &    2.0\\
real   &  xfar    &      far wing distance(cm-1)     &    25.0\\
\hline
integer & nbox     &     boxcar sum size (odd integer) &  1,5\\
\hline
real   &  strfar   &    min line strength for far wing lines & \\
real   &  strnear  &    min line strength for near wing lines& \\
\hline
char   &  LVG       &    (L)orentz,Voi(G)t,(V)anHuber  &  'V'\\
\hline
\end{longtable}

The output arguments from the function call are the output wavevector, 
outwave, and the computed line spectra in outarray. The vector outwave (and 
thus the output array outarray) spans the wavenumber range from $fmin$ to 
$fmax-(ffin \times  nbox)$, at a resolution of $ffin \times nbox$.

\subsection{Detailed description of the input parameters}

Two of the input parameters are self-describing. The first parameter, 
$gasID$ is an integer value specifying which gas you want to compute the 
line spectra for. This integer value is the same as that used for the 
HITRAN database; for example gasID=3 corresponds to ozone. $LVG$ is a
character parameter that tells the code which lineshape to use for $all$ 
the lines. Of the lineshapes described previously, our code can compute 
one of the following three - \textbf{L}orentz, Voi\textbf{G}t or 
\textbf{V}anVleck-Huber. The 
VanVleck-Huber is computed with a Voigt lineshape, and is the one we 
recommend; to use this lineshape, $LVG$ is set to 'V'.

When the code starts running, it loads in all lines whose centers lie 
between $fmin - xfar$ and $fmax + xfar$, and whose database line strength 
is greater than min(strfar,strnear), as the user assumes these are the 
lines which will have a discernible effect on the overall spectra. Using 
these lines, and their associated parameters, the computations are 
performed on a fine mesh resolution $ffin$ and then boxcar
averaged to an output resolution $ffin \times nbox$. The results of the 
computations are output for a wavector that spans $fmin$ to $fmax-ffin 
\times nbox$. Internally, the computations are essentially performed on a 
fine mesh that spans $fmin - (nbox-1)/2 $ to $fmax-ffin \times nbox + 
(nbox-1)/2$. In this way, the boxcar averaging can be done on the endpoints.

If this direct method were used, depending in the gasID and wavenumber 
region chosen, the code could be agonizingly slow. In the interests of 
speed (and mantaining the accuracy), the code therefore requires some more 
parameters to be sent in. 

With these additional parameters, the output wavevector $fmin$ to 
$fmax-ffin \times nbox$ is divided into equal sized ``wide meshes'' of 
size $fstep$ $cm^{-1}$. Thus there are $N$ wide meshes, where
\begin{equation}
N = \frac{fmax-fmin}{fstep}
\end{equation}

Suppose we are considering the $i$th  widemesh, and we denote the start 
frequency  of this widemesh by $f1$, and the stop frequency by $f2$. These 
two numbers are related to each other and to the other numbers by 
\begin{displaymath}
f1 = fmin + (i-1) \times fstep - ffin \times (nbox-1)/2 
\end{displaymath}
\begin{equation}
f2 = fmin + ii \times fstep - (ffin \times nbox) + ffin \times (nbox-1)/2
\end{equation}
This ``finemesh'' thus spans $(f1,f2)$ at the fine resolution of 
$ffin$ $cm^{-1}$

Associated with this finemesh is a medium resolution mesh, that spans 
$(f3,f4)$ at a coarser resolution $fmed$ where
\begin{displaymath}
f3 = fmin + (i-1) \times fstep\\
\end{displaymath}
\begin{equation}
f4 = fmin + ii \times fstep\\
\end{equation}

In addition there is a coarse resolution mesh, that spans 
$(f3,f4)$ at a coarsest resolution $fcor$ where $f3,f5$ are the same, as are
$f4,f6$ : 
\begin{displaymath}
f5 = fmin + (i-1) \times fstep\\
\end{displaymath}
\begin{equation}
f6 = fmin + ii \times fstep\\
\end{equation}

The spectral region of the output wavevector that corresponds to these 
three meshes is essentially $f3,f4$, adjusted for the last point. In other 
words, the
$i$th output region $fout(i)$ spans $f3$ to $f4-nbox \times ffin$, at a 
resolution of $nbox \times ffin$

For any of the $N$ widemeshes, lines are grouped into three categories, 
depending where they fall within the three categories defined below : \\
(a) near lines are those whose line centers lie in the wavenumber interval
$(f3-xnear,f4+xnear)=(w1,w2)$. All computations using these lines are 
performed on the fine mesh (spanning $f1,f2$) of point spacing ffin, 
and then boxcar averaged to the output wavevector $fout(i)$ \\
(b) medium lines are those whose line centers lie in the wavenumber interval
$(f3-xnear-xmed,f3-xnear) \cup (f4+xnear,f4+xnear+xmed)=(w3,w1) \cup (w2,w4)$. 
All computations using these lines are performed on the medium mesh 
(spanning $f3,f4$) of 
point spacing fmed, and then splined to the output wavevector $fout(i)$\\
(c) far lines are those whose line centers lie in the wavenumber interval
$(f3-xmed,f3-xfar)\cup(f4+xmed,f4+xfar)=(w5,w3)\cup (w4,w6)$. All 
computations using these lines are performed on the coarse mesh 
(spanning $f3,f4$) of 
point spacing fcor, and then splined to the output wavevector $fout(i)$\\
The cartoon in Figure \ref{fig:cartopon_param} summarizes the above 
relationships.

\begin{figure}[h]
  \begin{center}\includegraphics[width=6in]{Figures/fig1.eps}\end{center}
  \caption[Cartoon of Parameter Relations]{}
  \label{fig:cartoon_param}
\end{figure}

With the above description, the following restrictions on the parameters are
now self explanatory : \\
(1) xnear $\le$ xmed $\le$ xfar        \\
(2) xnear $\ge$ fstep              \\
(3) xmed/fmed  xnear/fmin   fstep/fmed   fstep/ffin       are integers \\
(4)fstep/(nbox*ffin)        fcor/ffin                     are integers \\
(5)(fmax-fmin)/fstep        (fmax-fmin)/fcor              are integers \\

A useful rule of thumb is that ffin,fmed,fcor are chosen so that they 
are all equal to $(1/2)/10^{n}$   $n \le 5$, with $n$ chosen as necesary for 
the three parameters. For example, $n=3,1,0$ gives $ffin=0.0005,fmed=0.05,
fcor=0.5 cm^{-1}$.

The above algorithm used is almost the same as that used by GENLN2, except 
that GENLN2 does not currently have the medium resoltion mesh ie the overall 
lineshape is a sum of boxcar averaged fine mesh contribution and a spline
computed coarse mesh contribution.

Other differences found between this LBL code and GENLN2 is that all 
computations here are in real*8, while GENLN2 mixes bewteen real*8 and 
real*4. 
The discrepancies between these two representations is noticeable in 
computations of eg the partition fucntions. In addition, we believe that the
contribution of a line whose center is in the ``far line'' regime, is 
incorrectly splined at the last interval $|x_{center} - x| \sim xfar$

\subsection{Detailed description of the output parameters}
Two parameters are passed out after running the code : a 1d array $outwave$ 
that contains the output wavevector, and a 2d matrix $outarray$ that 
contains the computed lineshapes at the user set atmospheric levels.

\subsection{Outline of the algorithm}

The code starts out by checking to ensure that the input parameters make 
sense and that they are self consistent. For example, parameter LVG must 
be set to
one of the allowable line shapes. In addition, the parameters should all be 
self consistent in that they have to staisfy the restrictions given at the 
end of the previous subsection.

Having ascertained the self consistency of the parameters sent in by the 
user, the program loads in the required mass isotopes for the chosen 
gas. For example ozone has 5 isotopes. 

The program then loads in the user specified profile for the gas.
Having done this, the program then uses $fmin,fmax,ffin,nbox$ to define the 
output wave vector. After this, the gas initializes the $qtipts$ 
coefficients that are used to compute the partition functions. This is 
essentially a GENLN2 subroutine, similar to the program ``tips'' by 
R.R.Gamache.

The program is now ready to load in the gas line parameters from the HITRAN
database. As described above, it loads in all lines whose centers lie 
between $fmin - xfar$ and $fmax + xfar$, and whose database line strength 
is greater than min(strfar,strnear). 

At this point, the program is almost ready to start running in earnest. 
Before doing that, it computes the number of wide meshes $N$ and the number 
of points in each wide mesh that will be mapped to the output wavevector.
If the user loaded in a profile that has more than one layer in it, 
the program calls subroutine $doUnion2$, that
computes the optical depth of each linecenter for the chosen profile 
conditions; if a line is strong enough to be used in any $one$ of the 
levels, it will be used at $all$ levels. This will ensure that the optical 
depth profiles are smooth. An example of this is the case of ozone, where 
lines could ``turn-on'' high in the atmosphere, but have almost no optical 
depth lower in the atmosphere. The importance of this is when the output 
from the code is used to generate the kCARTA database using Singular Value 
Decomposition; the SVD algorithm would work more efficiently with smoothly 
varying data (achieve better compression).

Figure \ref{fig:init_alg} outlines the above initialisation stages of the 
algorithm.

\begin{figure}[h]
  \begin{center}\includegraphics[width=1.5in]{Figures/fig2.eps}\end{center}
  \caption[Outline of initialization algorithm]{}
  \label{fig:init_alg}
\end{figure}

The program is now ready to loop over the far,medium and near lines. 
For each of the wide meshes, the program first defines the fine, medium and
coarse meshes (frequencies and indices), as described in the previous 
section. It then sorts all the lines it has loaded into three bins; near,
medium and far, also as described in the previous section.

It then enters a loop over layers. For the current layer, the program uses 
the gas profile to determine the gas amount, temperature, total and self 
pressures. For each layer, it first computes the contribution due to the 
near lines, then the medium lines and finally the far lines. The near line 
spectrum is computed
on the fine mesh, and the results are boxcar averaged and added onto the
output array. The medium line spectrum is computed on the medium mesh, 
spline interpolated onto the output wavevector and added on to the output 
array.The far line spectrum is computed on the coarse mesh, spline 
interpolated onto the output wavevector and added on to the output array.

For each of the fine,medium and coarse computations, the code computes
the following line parameters, for each of the lines\\
(a) the partition function, using $qfcn=q(A,B,C,D,G,lines,tempr)$\\
(b) the line center frequency, taking the pressure of the current layer into
    account $freq=lines.ZWNUM+press(jj)*lines.ZTSP$\\
(c) the overall broadening of the line, using the self and foreign 
    broadening
    contributions $brd=broad(p,ps,1.0,forbrd,selfbrd,pwr,tempr,gasID)$\\
(d) the line center line strength, using the necessary layer parameters 
    such as temperature, gas amount and necessary line parameters \\
    $strength=find_stren(qfcn,freq,tempr,energy,s0,GasAmt(jj))$\\
The above computations are essentially GENLN2 routines.

\begin{figure}[h]
  \begin{center}\includegraphics[width=2.0in]{Figures/fig3.eps}\end{center}
  \caption[Outline of loops over layers and fine,medium,coarse meshes]{}
  \label{fig:loop_alg}
\end{figure}

Figure \ref{fig:loop_alg} outlines the loop stage of the algorithm.

\section{run6water}

run6water.m is a specialised code for H2O, so as to to utilise the above 
physics, namely local lineshape and the CKD continuum effects in the 
computed 
lineshapes. If the user simply wants to do a Lorentz or Voigt computation,
then it would behoove him/her to use $run6.m$ instead of this special code.

\subsection{run6water.m input parameters}

A typical call to run6water would involve sending in the following : 

$[outwave,outarray]=run6water(gasID,fmin,fmax,ffin,fmed,fcor,$\\
              $fstep,xnear,xmed,xfar,nbox,strfar,strnear,LVF,$\\
              $CKD,selfmult,formult,usetoth,local,profname);$

where the right hand side variables are the same as those for run6 
described above; there are 5 new variables on the right side.

\begin{longtable}{llll}
  TYPE  &   VAR  &         DESCRIPTION  &            TYPICAL VALUE\\
\hline
integer & gasID  &       HITRAN gas ID      &            2\\
\hline

integer & fmin    &      minimum freq (cm-1) &          705\\
integer & fmax    &      maximum freq (cm-1) &          730\\
\hline

real   &  ffin    &      fine point spacing (cm-1) &    0.0005\\
real   &  fmed    &      medium point spacing (cm-1)&   0.1\\
real   &  fcor    &      coarse point spacing (cm-1)  & 0.5\\
\hline

real   &  fstep   &      wide mesh width size (cm-1) &    1.0\\
real   &  xnear   &      near wing distance(cm-1)    &    1.0\\
real   &  xmed    &      med wing distance(cm-1)     &    2.0\\
real   &  xfar    &      far wing distance(cm-1)     &    150.0\\
\hline

integer & nbox     &     boxcar sum size (odd integer) &  1,5\\
\hline

real   &  strfar   &    min line strength for far wing lines & \\
real   &  strnear  &    min line strength for near wing lines& \\
\hline

char   &  LVG      &      (L)orentz,Voi(G)t,(V)anHuber  &  'V' \\
integer &  CKD     &       continumm no (-1)            &   -1 \\
        &          &       yes water : (0,21,23,24)    & \\
\hline

real    &  selfmult &       multiplier for self part of contiuum &  0<x<1 \\
        &  formult  &       multiplier for for  part of contiuum  & 0<x<1 \\
\hline

integer & usetoth &        use Toth or HITRAN &           +1 to use Toth \\
        &         &                           &          -1 to use HITRAN \\
\hline

integer & local &         use local lineshape   & +1 to use local*chi defn\\
        &       &                                  &  0 to use local defn \\
        &       &                                  & -1 to use run6 defn\\
\hline
\end{longtable}

The output arguments from the function call are once again the output 
wavevector,  outwave, and the computed line spectra in outarray. The vector 
outwave (and thus the output array outarray) spans the wavenumber range 
from $fmin$ to $fmax-(ffin \times  nbox)$, at a resolution of 
$ffin \times nbox$.

\subsection{Detailed description of the input parameters}

As mentioned above, most of the input parameters are the same as for $run6$
and a description is not repeated here. However, five of the last six 
parameters are new, and so will be explained below.

$CKD$ is a integer parameter that tells the code which continuum to use. 
Note that based on whether or not the ``local'' lineshape was used, the 
appropriate CKD lookup tables are used. For CKD 0,21,23 the code can 
compute the continuum whether or not the local lineshape was used; for 
CKD24, only the local lineshape can be used.

$selfmult$ is a real parameter between 0 and 1, that is used to scale the
``self'' contribution to the continuum.

$formult$ is a real parameter between 0 and 1, that is used to scale the
``foreign'' contribution to the continuum.

$usetoth$ is a integer parameter that tells the code whether or not to use 
the Toth database.

$local$ is a integer parameter that tells the code whether or not to compute
the local lineshape (must be set to ``1'' to use CKD2.4)

Eventually there will be an additional parameter, so as to use Dave Tobin's
chi functions.
\section{run6co2}

run6co2.m is a specialised code for CO2, so as to to utilise the above 
physics, namely line mixing and duration of collision effects in the 
computed lineshapes. If the user simply wants to do a Lorentz or Voigt 
computation, then it would behoove him/her to use $run6.m$ instead of this 
special code.

\subsection{run6co2.m input parameters}

A typical call to run6co2 would involve sending in the following : 

$[outwave,outarray]=run6co2(gasID,fmin,fmax,ffin,fmed,fcor,$\\
              $fstep,xnear,xmed,xfar,nbox,strfar,strnear,LVF,IO,birn,$
              $profile)$

where the right hand side variables are the same as those for run6 
described above; there are three new variables on the right side.

\begin{longtable}{llll}
  TYPE  &   VAR  &         DESCRIPTION  &            TYPICAL VALUE\\
\hline
integer & gasID  &       HITRAN gas ID      &            2\\
\hline
integer & fmin    &      minimum freq (cm-1) &          705\\
integer & fmax    &      maximum freq (cm-1) &          730\\
\hline
real   &  ffin    &      fine point spacing (cm-1) &    0.0005\\
real   &  fmed    &      medium point spacing (cm-1)&   0.1\\
real   &  fcor    &      coarse point spacing (cm-1)  & 0.5\\
\hline
real   &  fstep   &      wide mesh width size (cm-1) &    1.0\\
real   &  xnear   &      near wing distance(cm-1)    &    1.0\\
real   &  xmed    &      med wing distance(cm-1)     &    2.0\\
real   &  xfar    &      far wing distance(cm-1)     &    150.0\\
\hline
integer & nbox     &     boxcar sum size (odd integer) &  1,5\\
\hline
real   &  strfar   &    min line strength for far wing lines & \\
real   &  strnear  &    min line strength for near wing lines& \\
\hline
char   &  LVF       &    (L)orentz,(V)anHuber,(F)ullMixing &  'F'\\
char   &  IO        &    '0' for no mixing, '1' for mixing &  '1'\\
char   &  birn      &    (n)o chi , (c)ousin, (b)irnbaum   &  'b'\\
\hline
\end{longtable}

The output arguments from the function call are once again the output 
wavevector, 
outwave, and the computed line spectra in outarray. The vector outwave (and 
thus the output array outarray) spans the wavenumber range from $fmin$ to 
$fmax-(ffin \times  nbox)$, at a resolution of $ffin \times nbox$.

\subsection{Detailed description of the input parameters}

As mentioned above, most of the input parameters are the same as for $run6$
and a description is not repeated here. Howvere, the last three parameters
are either new or slightly different than before, and so will be expounded
upon below.

$LVF$ is a character parameter that tells the code which lineshape to use 
for the lines. Of the lineshapes described previously, this code can 
compute one of the following three - \textbf{L}orentz, 
\textbf{V}anVleck-Huber or 
\textbf{F}ull mixing. The VanVleck-Huber is computed with a Voigt 
lineshape, while the full mixing is essentially a Lorentz computation and so
should only be used at higher pressures.

$IO$ is a character parameter that works if $LVF$ is not 'F' or 'f'. It 
tells the code whether or not to do $no$ line mixing (IO='0') or to do 
first order line mixing (IO='1').

$birn$ is a character parameter that determines the chi function to be used.
There are three choices : (n)o, (c)ousin and (b)irnbaum. If $birn$='C','c'
then the cousin lineshape cannot be used with mixing turned on; hence the 
user cannot have $LVF$='F','f' or $IO$='1', with cousin chi functions on.

\subsection{Outline of the algorithm}
The initialisation part of the code is the same as that of $run6$. This code
is designed to compute the line mixing in various P,Q,R bands using the 
latest
knowledge. As such, after the code loads in the line parameters from the 
HITRAN database, it then goes through a list of bands/branches that it has
specific code for, and removes $all$ the lines that fall within this list,
saving the parameters in assorted *.mat files. All the lines that remain are
considered background lines.

Figure \ref{fig:init_algCO2} outlines the above initialisation stages of 
the algorithm for run6co2.

\begin{figure}[h]
  \begin{center}\includegraphics[width=1.25in]{Figures/fig4.eps}\end{center}
  \caption[Outline of initialization algorithm for CO2]{}
  \label{fig:init_algCO2}
\end{figure}

The computation of the spectrum then proceeds in two stages. The first 
stage 
involves using only the background lines, and the spectrum associated with 
these lines is computed in the exact same fashion as described above for 
$run6$. The slight difference is that only one of two lineshapes is used :
if the user specified $LVF$ = 'L','l', then a Lorentz/vanhuber line shape 
is used, else a voigt/vanhuber line shape is used.

The second stage is entirely peculiar to run6co2. For the bands that lie 
within the HITRAN lines read in, the program proceeds as follows. First the 
code creates a ``high resolution'' wavevector that spans 
$f_{high}=(fmin,fmax)$, at 
the fine (highest) resolution. The code is then ready to loop. The
outermost loop is over the bands/branches, while the inner loop is over 
atmosphere layers. For each layer, the program stores current layer profile 
data : total and self pressure, gas amount and temperature. Depending on 
the total pressure of the layer, it then follows one of the following four 
choices : \\
(a) if $LVF$  = 'F','f' and total pressure is above 0.158 atm, do full line
    mixing computation, as described below. This is because pressures are
    high enough that a Lorentz line shape is deemed good enough OR\\
(b) if $LVF$  = 'F','f' and total pressure is below 0.021 atm, do first 
    order
    mixing computation, as described below. This is because pressures are
    low enough that a Doppler line shape is required OR\\
(c) if $LVF$  = 'F','f' and total pressure is between 0.021 and 0.158 atm, 
    do a weighted average of first order and full mixing computation, as 
    described below. This is because pressures are high enough that first 
    order approximations are not good enough, while pressures are low 
    enough 
    that a Doppler line shape is required OR \\
(d) if $LVF$ = 'L','V', do the user input lorentz/voigt with linemixing 
    turned on or off (depending on the setting of $IO$).

Figure \ref{fig:PQRloop_CO2} outlines the second stage of the algorithm for 
run6co2.

\begin{figure}[h]
  \begin{center}\includegraphics[width=3.5in]{Figures/fig5.eps}
  \end{center}
  \caption[Outline of PQR band loop details for CO2]{}
  \label{fig:PQRloop_CO2}
\end{figure}

For each of the bands, and within each of the above 4 choices, the program 
proceeds as follows. First the HITRAN line parameters for the particular 
band/branch combination are read in from the hit*.mat saved earlier. The 
code then computes parameters such as foreign and self broadening, line 
strength and so on. This is done in $loader.m$ 

Now the code makes some important speed-affecting decisions based
on the current band/branch, and the input high resolution wavevector.
Suppose the band lines span the wavevector range $(A,B)$. A preset
lookup table has a set of pairs of numbers $(a,b)$ outside of which the 
effects of linemixing can be simply computed from the lorentz line shape 
i.e. $k = k_{lorentz} \times MixRatio$. For the current profile 
conditions, this 
ratio is computed on the fly using the sum rules derived by Strow et al. 
If this ratio is computed to be negative, it is reset to a very small number
(practically zero). The input wavevector $f_{high}$ then can be divided 
into three regions : one that has wavenumbers 
that are less than $a$, another that has wavenumbers that are greater than 
$b$ and a third region whose wavenumbers lie in the interval $(a,b)$.

To compute the  ratio, the code has to go ahead and compute the 
matrices/mixing coefficients associated with the current band/branch 
combination. The matlab routines are typically $efitter,$ $wfunco2er,$ 
$wfun1co2,$ $wgradco2,$ $trans_pop$

The code now loops over the three wavenumber regions defined above :\\
(a) for the first two regions, the code simply does a lorentz computation, 
    and multiplies it by the required ratio. Note that in order to speed 
    things up, the computation is done at the $output$ wavevector 
    resolution, so that no boxcar integrations, or splines, have to be 
    done. If a birnbaum or cousin factor is required, it is computed for 
    each line and multiplied in. Thus the overall lineshape in this 
    region is
    \begin{equation}
 k(\nu) = ratio \times \sum_{i=1}^{i=F} k_{lorentz}(\nu,\nu_{i},brd_{i}) 
 \times \chi(\nu)
  \end{equation}
   where the sum is over all the lines in the band, $\nu_{i},brd_{i}$ 
   being the line centers and broadening respectively, and $\chi$ is 
   either the birnbaum factor or 1.00, and $\nu$ is the portion of the 
   $output$ wavevector. The rationale behind the output wavevector being 
   used for the computation is that the first two wavenumber regions are 
   so far away from the lines in the band that the optical depth $k$ is 
   both very small and varying very slowly.\\
  For the third region, the code does an entire mixing computation : \\

(b) If a full mixing computation is required, the code sets up the required 
    matrix and finds the eigenvalues. The spectra is then built up, at 
    the $fine$ resolution, using the eigenvalues as described earlier on 
    in this document. If a birnbaum factor is required, it
    is computed as follows. For each line in the band, a lorentz line 
    shape is computed, as is the corresponding birnbaum factor. A vector 
    containing the sum of the products of the lorentz, birnbaum factors, 
    as well as another vector containing the sum of the lorentz lines, is 
    kept. The overall full mixing lineshape is then multiplied by the 
    ratio of thses two sums. Thus the overall lineshape in this region is
\begin{equation}
k(\nu) = k_{full}(\nu,brd,strengths) \times \chi(\nu)
\end{equation}
where $\chi$ is either 1.00 or computed as follows
\begin{equation}
\chi(\nu) = 
\frac{\sum_{i=1}^{i=F} k_{lorentz}(\nu,\nu_{i},brd_{i}) \times birn(\nu)}
     {\sum_{i=1}^{i=F} k_{lorentz}(\nu,\nu_{i},brd_{i})}
\end{equation}
    Having done this the code then boxcar averages the result to the output
    resolution, and adds on the result to the required part of $outarray$

(c) If a first order mixing computation is required, the code sets up the 
    required first order coeffs. The spectra is then built up, at the $fine$
    resolution, using the mix coeffs. If a birnbaum factor is required, it
    is computed for each line, and then multiplied in. 
    Thus for $LVF$ = 'L', the overall lineshape in this region is
\begin{equation}
k(\nu) = \sum_{i=1}^{i=F} (k_{lorentz} + Y_{i} \times f(\nu - \nu_{i}))
\times \chi(\nu)
\end{equation}
    while for $LVF$ = 'V', the overall lineshape in this region is
\begin{equation}
k(\nu) = \sum_{i=1}^{i=F} (k_{voigt} + Y_{i} \times f(\nu - \nu_{i}))
\times \chi(\nu)
\end{equation}
    where the sum is over all the lines in the band, $\nu_{i},brd_{i}$ 
    being the line centers and broadening respectively, $\chi$ is the 
    birnbaum factor or 1.00, and $\nu$ is the portion of the $high 
    resolution$ wavevector. $Y_{i}$ are the first order mixing 
    coefficients, which multiply a term that essentially depends on the 
    distance away from the line center, $\nu - \nu_{i}$\\
    Having done this the code then boxcar averages the result to the output
    resolution, and adds on the result to the required part of $outarray$

(d) If a cousin computation is required, the code is also broken up into the
    three regions above. However, the code does $not$ have to do any of the
    mixing computations. The spectra is then built up, at the $fine$
    resolution, for region 3, or at the lower output resolution, for 
    regions 1 and 2. Thus for $birn$ = 'C', $LVF$ = 'L', the overall 
    lineshape is
\begin{equation}
k(\nu) = \sum_{i=1}^{i=F} k_{lorentz} \times \chi(\nu)
\end{equation}
    while for for $birn$ = 'C', $LVF$ = 'L', the overall lineshape is
\begin{equation}
k(\nu) = \sum_{i=1}^{i=F} k_{voigt} \times \chi(\nu)
\end{equation}
    where the sum is over all the lines in the band, $\chi$ is the cousin 
    factor, and $\nu$ is the relevant portion of the $high resolution$ or 
    $output resolution$ wavevector. 
    Having done this the code then boxcar either averages the result to 
    the output resolution, and adds on the result to the required part of 
    $outarray$, or directly adds on the result to the required part of 
    $outarray$, as appropriate.

Both the birnbaum and cousin computations are done as lookup tables.

Figure \ref{fig:details_CO2} outlines the above stages of the core workings 
of the PQR band/branches algorithm for run6co2, assuming the user wants to 
do either a line mixing or a cousin computation.

\begin{figure}[h]
  \begin{center}\includegraphics[width=5.0in]{Figures/fig6.eps}\end{center}
  \caption[Core PQR band details for CO2]{}
  \label{fig:PQRloop_CO2}
\end{figure}

\subsection{Band details}
Each of the bands used for the special linemixing computations in $runCO2$ 
are associated with a unique identifier, which is approximately equal to the
band center. This identifier enables the code to pull out the lines from 
the HITRAN database that have similar quantum numbers. For example, $720$ 
would pull out all HITRAN lines having the lower vibrational quanta 
index = 2, upper vibrational quanta index = 5,  isotope = 1 (main isotope).
This is the Q720 sigpi branch.

The bands are divided into subsets associated with the lower and upper level
angular momenta : sigmapi, deltapi, sigmasigma, pipi, deltadelta. To see 
which HITRAN quantum numbers/isotopes are associated with each subset, 
the user can look at file $makeDAVEhitlin.m$. Only the strongest bands have
been included in $run6co2$.

Since some of the bands in the subsets are associated with isotopes, the 
code used to compure the required mixing matrices for some bands within 
the same isotope, may be quite different from that for the other bands in 
the isotope. In particular, computing the $a1,a2,a3$  coefficients for the 
power-energy scaling gap can vary significantly.

Below, copied from $makeDAVEhitlin$, is a summary of the bands currently 
used for line mixing: 
\begin{verbatim}
%find all PQR lines for isotope 1 (or 2 or 3) : these are PQR_sigpi
if (band==618)
  v_l=2;v_u=3;
elseif (band==648)
  v_l=1;v_u=2;isotope=2;
elseif (band==662)
  v_l=1;v_u=2;isotope=3;
elseif (band == 667)
  v_l=1;  v_u=2; 
elseif (band == 720)
  v_l=2;  v_u=5; 
elseif (band==791)
  v_l=3;v_u=8;
elseif (band==2080)
  v_l=1;v_u=8;

%find all PQR lines for isotope 1 : these are PQR_deltpi
elseif (band==668)
  v_l=2;v_u=4;
elseif (band==740)
  v_l=4;v_u=8;
elseif (band==2093)
  v_l=2;v_u=14;

%find all PQR lines for isotope 1 : these are PQR_sigsig
elseif (band==2350)
  v_l=1;v_u=9;
elseif (band==2351)
  v_l=1;v_u=9; isotope=2;
elseif (band==2352)
  v_l=1;v_u=9; isotope=3;
elseif (band==2353)
  v_l=3;v_u=23; 
elseif (band==2354)
  v_l=5;v_u=25; 

%find all PQR lines for isotope 1 : these are PQR_pipi
elseif (band==2320) 
  v_l=2;v_u=16; 
elseif (band==2321) 
  v_l=2;v_u=16; isotope=2;
elseif (band==2322) 
  v_l=2;v_u=16; isotope=3;

%find all PQR lines for isotope 1 : these are PQR_deltdelt
elseif (band==2310) 
  v_l=4;v_u=24;
elseif (band==2311) 
  v_l=4;v_u=24; isotope=2;

\end{verbatim}

\subsection{PQR\_sigpi}
For the Q branches, the code is quite straightforward, except for the 662 
band. Since this is the 3rd isotope (O16 C12 O18) the symmetry is broken 
and now we can have many more allowed lines. 

For the PR branches, only the 720 branch has the line mixing matrices being
computed ... the rest of the bands use $k/klor$ = 0.5

\subsection{PQR\_deltpi}
For the Q branches, the code is quite straightforward. For the PR branches,
all the bands use $k/klor$ = 0.5

\subsection{PR\_sigsig}
No Q branch mixing done. The strongest lines in the 4 $\mu m$ region are 
from this subset. For the PR branches, full line mixing done. Band 2352 is 
special because it is a symmetry breaking isotope.

\subsection{PR\_pipi}
No Q branch mixing done. For the PR branches, full line mixing done. Band 
2322 is special because it is a symmetry breaking isotope. In addition, 
band 2321 needed a special $efitter$ initiliazation when fitting for 
missing energy levels.

\subsection{PR\_deltdelt}
No Q branch mixing done. For the PR branches, full line mixing done. 

\newpage

\section{General Spectral Lineshape Theory}\label{chpt:shapes}

This section examines the basic lineshape parameters,
including line centers, shifts, widths, and strengths.  The standard 
lineshapes for natural and Doppler broadening are then presented, followed 
by a review of collisional lineshapes. Most of this information 
is from Dave Tobin's PhD dissertation \cite{tob:96}.

\subsection{Molecular Absorption and Beer's Law}\label{sec:beerslaw}

Molecular absorption occurs when a molecule absorbs light and 
simultaneously makes a transition to a higher level of internal 
energy.  The absorption of incident photons decreases the outgoing 
radiation, and a spectral line is produced.  The shape or frequency 
dependence of this absorption is often called the lineshape.  Because 
of various factors, the absorption occurs not only at the resonant 
frequency of the transition (determined by the difference between the 
upper and lower energy levels), but over a spread of frequencies.  
These broadening factors lead to a finite width of the spectral 
line.  While the resonant frequency and the intensity of the 
absorption are determined primarily by the structure of the molecule, the 
lineshape is determined by the molecules' environment.  

The frequency dependence of the absorption coefficient, $k(\nu)$, 
determines the shape of a spectral line.  Beer's law relates the 
absorption of radiation through a gaseous medium linearly to the incident 
radiation (see Figure \ref{fig:beers_law}):
\begin{equation}
   -dI = k(\nu)I_{0}P dl
\label{eqn:blaw1}
\end{equation}
with the absorption coefficient, $k(\nu)$ being the constant of
proportionality.  $-dI$ is the decrease in radiation flux over a path 
length of $dl$ through a gas of constant and uniform pressure $P$.  
Integrating this equation over a homogeneous path length of $L$ yields 
the integrated form of Beer's law:
\begin{equation}
 T(\nu)=\frac{I_f(\nu)}{I_0(\nu)}=\exp\left(-k(\nu)P L\right).
\label{eqn:blaw2}
\end{equation}
$T(\nu)$ is is the transmission at frequency $\nu$. $I_{0}(\nu)$ and
$I_f(\nu)$ are the initial and final radiation intensities.  Thus, the 
absorption coefficient is related to the observed transmission by 
\begin{equation}
 k(\nu)=-\frac{1}{P L}\ln\left(T(\nu)\right).
\end{equation}

It should be noted that deviations from the linear form of Beer's law are
only observed at extremely high photon densities.  Under atmospheric
conditions, however, the linear dependence of the extinction on the amount 
of absorbing material and incident radiation is valid. 

\begin{figure}[h]
 \begin{center}
  \includegraphics[width=3.5in]{Figures/beerslaw.eps}\end{center}
  \caption[An illustration of Beer's law.]{A gas cell of pressure $P$ and 
    length $L$ with incident radiation,
    $I_{0}$, from the left.  The amount of absorption at frequency $\nu$ is
    determined by the magnitude of the absorption coefficient, $k(\nu)$,
    the gas pressure, and the path length according to Beer's law.}
  \label{fig:beers_law}
\end{figure}

\subsection{Line Parameters}\label{sec:lines}

The lineshape of a single (non-interacting) transition is commonly 
characterized by several parameters including the line {\it center} 
($\nu_0$), 
line {\it strength} ($S$), and line {\it width} ($\gamma$).  These are
illustrated in Figure \ref{fig:spectralline}.
\begin{figure}[h]
  \begin{center}
  \includegraphics[width=4.0in]{Figures/spectralline.eps}
  \end{center}
  \caption[A spectral line depicting the line center and line width.]
	{A spectral line depicting the line center, $\nu_0$, and half-width,
	$\gamma$.  The line strength, $S$, is the absorption coefficient,
	$k(\nu)$, integrated over all wavenumbers $\nu$.}
  \label{fig:spectralline}
\end{figure}

Several spectral line databases are available which provide a compilation
of the line positions, strengths, and widths as well as several other
important parameters such as the lower state energy, pressure induced line
center shifts, isotopic abundances, rotational and vibrational quantum
indexing, width-temperature exponents, and transition probabilities.
HITRAN\cite{rot:87,rot:92} 
(the high resolution transmission molecular absorption database) 
is one such database used in this work which is maintained by the Phillips 
Laboratory Geophysics Directorate .  It currently lists the
parameters of over 700,000 rotation and vibration-rotation spectral lines
for 31 molecules of atmospheric importance from 0--23,000 cm$^{-1}$.
This database represents the most accurate compilation of line parameters.
However, due to its size, it is only updated every two to four years and 
thus recent
state-of-the-art measurements and calculations are not always in the
database and must be obtained elsewhere.

\subsubsection{Line Centers}

The line {\it center}, or position, of a spectral line is determined by the
molecular structure, just as the allowable vibrational-rotational energy
levels of the molecule are determined by its structure.  Planck's relation: 
\begin{equation}
\nu_{0}=\frac{\Delta E}{h}
\end{equation}
relates the transition frequency, $\nu_{0}$ (cm$^{-1}$), to the 
change in internal energy, $\Delta E$, where $h$ is Planck's constant.  
The line centers are thus determined by the structure and allowed energy
levels of the molecule and by transition selection rules. 

Since the line centers do not depend critically 
upon the interactions with other molecules, or upon the population of 
various states, they do not vary significantly with temperature or 
pressure. One common exception to this are the very small shifts in line 
center with increasing pressure.  Just as molecular collisions can 
disturb optical transitions leading to increased line widths 
(discussed later), these disturbances can also lead to an apparent change 
in the resonant frequency of the molecule's wave-train.  Computationally, 
the shifted position is given by
\begin{equation}
\nu_{0}(P)=\nu_{0}+P\cdot\delta_{\nu}
\end{equation}
where P is the total pressure and $\delta_{\nu}$ is the 
pressure induced frequency shift and is determined either 
theoretically or experimentally.  A more theoretical explanation of line
shifts which are due to distant collisions was first given by
Lenz and more recently by Breene\cite{bre:56}.

\subsubsection{Line Strengths}

The line {\it strength} or line intensity is a direct measure of the 
ability of a molecule to absorb photons corresponding to a given transition.
It depends upon both the properties of the single molecule and the relative
number of molecules in the upper and lower states.  The strength, $S$, is 
defined as
\begin{equation}
S=\int k(\nu_{0},\nu)d\nu
\label{eqn:strength1}
\end{equation}
where the integral is over all $\nu$.  Experimentally, $S$ can be
determined using Equation \ref{eqn:strength1} if $k(\nu_0,\nu)$ is
measured.  Alternatively, if the functional form of $k(\nu_0,\nu)$ is
known,  regression techniques can be used to determine $S$.  

By far the strongest interaction between matter and an incident field of 
electromagnetic radiation involves the molecule's electric dipole moment.  
The intensity of a dipole transition is proportional to the square of the 
matrix element of the dipole moment operator $M$:
\begin{equation}
R_{i,j}=\int \Psi_i^{*}M\Psi_j dV
\end{equation}
where $dV$ is a volume element in configuration space and the integral 
is over all space.  $\Psi_i$ and $\Psi_j$ are the wavefunctions 
of the lower and upper levels of the transition.  The 
wavefunctions are orthogonal and therefore, if $M$ is unchanged during the
transition, $R=0$.  Consequently, for a dipole transition to occur, the 
electric dipole moment must change between the initial and final energy 
levels 
of a transition.  Otherwise, the molecule is not linked to the 
incident radiation and no absorption occurs.  Weaker transitions can occur,
however, for quadrupole transitions even if there is no change in the 
dipole moment,  although these are not considered in this work.

The line strength is also proportional to the relative populations of 
the upper and lower transition levels.  In thermodynamic equilibrium 
the probability of a molecule being in a specific energy level is given by
\begin{equation}
\frac{N_i}{N}=g_i \exp(-\frac{hc}{kT}E_i)/Z(T)
\label{eqn:boltz}
\end{equation}
where $h$ is Planck's constant, $c$ is the speed of light, $k$ is 
Boltzmann's constant, $T$ is the temperature, $N$ is the total number of
molecules, $N_i$ is the number of molecules in energy level $E_i$, $g_i$ is 
the statistical weight of the level, and $Z(T)$ is the partition function 
given by:
\begin{equation}
Z(T)=\sum_i g_i \exp(-\frac{hc}{kT}E_i)
\label{eqn:partfunc}
\end{equation}
In this work, the partition functions are computed using Gamache's
\cite{gam:90} convenient parameterization: 
\begin{equation}
Z(T) = a + b T + c T^2 + d T^3
\label{eqn:gamache}
\end{equation}
where $a$, $b$, $c$, and $d$ have been tabulated for most molecules found
in the lower atmosphere.  Combining Equations \ref{eqn:boltz} and 
\ref{eqn:partfunc}, the relative 
population of the upper and lower energy levels is given by:
\begin{equation}
 \frac{N_j - N_i}{N}=
  \frac{g_j \exp(-\frac{hc}{kT}E_j) - g_i\exp(-\frac{hc}{kT}E_i)}{Z(T)}
\end{equation}
The line strength is then expressed as 
\begin{equation}
S_{i,j}=\sigma_{i,j}\frac{N_j - N_i}{N}
\end{equation}
where $\sigma_{i,j}$ are integrated absorption cross sections and are 
given by
\begin{equation}
\sigma_{i,j}=\frac{8\pi^3}{3h}\nu_{i,j}|R_{i,j}|^2.
\end{equation}
Combining these results yields the full expression for the line strength:
\begin{equation}
S_{i,j}=\frac{8\pi^3}{3h}\nu_{i,j}|R_{i,j}|^2 
 \frac{g_i \exp(-\frac{hc}{kT}E_i)}{Z(T)}[1 -\exp(-\frac{hc}{kT}\nu_{i,j})]
\end{equation}
Using line strengths determined either theoretically or experimentally at
some reference temperature $T_{ref}$, the strength can be converted to other
temperatures using
\begin{equation}
 S_i (T) = S_i (T_{ref})
 \frac{ Z(T_{ref}) }{ Z(T) }
 \frac{ \exp(-hcE_i /kT) }{ \exp(-hcE_i /kT_{ref}) }
 \frac{ [1-\exp(-hc\nu_i /kT)] }{ [1-\exp(-hc\nu_i /kT_{ref})] }
\label{eqn:linestren}
\end{equation}

\subsubsection{Line Widths}

The line {\it width}, or halfwidth, is defined as half the frequency 
interval 
between $\nu_{0}$ and the frequency at which $k(\nu)$ has fallen to one
half of its maximum value.  Values of line widths in the Earth's atmosphere
can range from 0.0002 cm$^{-1}$ for
conditions where the molecules are isolated to 0.5 cm$^{-1}$ for conditions
of extreme pressure broadening.  Under pressure
broadening conditions, the resulting lineshape near line center is
well approximated as a Lorentzian with a nominal line width of
\begin{equation}
\gamma = \frac{2r^2}{m}\frac{P}{R_m T}
         \left(\frac{3kT}{m}\right)^{\frac{1}{2}},
\end{equation}
which has been derived from classical Kinetic theory using the 
Equipartition theorem and the Ideal gas law.  $r$ s the effective radius 
of the molecule,  $m$ is
the molecule's mass, $P$ is the total pressure, $R_m$ is the gas constant,
and $T$ is the temperature.  A typical time between collisions for
an atmospheric gas at room temperature and pressure is $\sim 10^{-10}s$,
which leads to a Lorentz width of $\sim$0.05 cm$^{-1}$.  
If the line width, $\gamma_0$, is determined at a given 
pressure, temperature combination ($P_0, T_0$), the line width at other 
conditions is
\begin{equation}
\gamma = \gamma_0\left(\frac{T_0}{T}\right)^{\frac{1}{2}}
         \left(\frac{P}{P_0}\right)
\end{equation}
Thus, the line width increases linearly with pressure and decreases with
temperature.   Although this kinetic theory does not result in accurate
values of $\gamma_0$, the pressure dependence is observed in most cases.
More commonly, $\gamma_0$ is determined either experimentally or
calculated with more realistic theories when accurate measurements are not
available.  Furthermore, the temperature exponent, $\frac{1}{2}$, is
generally replaced with a parameter $n$, which is also determined
experimentally. The accuracy of $n$ was
investigated by Lui Zheng and Strow \cite{lui:89} for both
CO$_{2}$$\leftrightarrow$ CO$_{2}$ and CO$_{2}$ $\leftrightarrow$ N$_{2}$
collisions: $n \simeq 0.69$ for CO$_{2}$ $\leftrightarrow$ CO$_{2}$
collisions and $n \simeq 0.75$ for CO$_{2}$ $\leftrightarrow$ N$_{2}$
collisions.  In general, $n$ can vary with transition for the same
molecule.  For example, accepted values of $n$ for H$_2$O 
range from 0.5 to 1.  When $n$ is unknown, default values of 0.64 and 0.68
are generally used.

For mixtures of gases, the total line width is the sum of the individual
partial widths:
\begin{equation}
\gamma_{TOT}=\sum_i \gamma_{0,i}P_i
\end{equation}

From quantum Fourier transform theory calculations, the line width for the
$f\leftarrow i$ transition is calculated using \cite{gam:83}
\begin{equation}
\gamma_i = \left(\frac{n v}{2\pi c}\right)\sum_{J_2}\rho(J_2)\sigma_{if,J_2}
\end{equation}
where $n$ is the perturber density, $c$ is the speed of light, $v$ is the
mean relative thermal velocity ($v=\sqrt{8k_B T/\pi \mu}$), $\mu$ is the
reduced mass of the perturber/absorber system, and $\rho(J_2)$ is the
density of the perturber state $J_2$.  $\sigma_{if,J_2}$ are the absorption
cross sections and are dependent upon which type of interactions are
dominant.  For example, for H$_2$O-N$_2$ collisions, the strongest
interaction is dipole-quadrupole, yielding 
\begin{equation}
\sigma_{if, J_2}^{DQ}=\pi b_0^2 \left(1+s_{if,J_2}(b_0)\right)
\end{equation}
where $b_0$ is an impact parameter related to the minimum distance between 
absorber and perturber during the interaction and $s_{if,J_2}$ is related
to the dipole moment of H$_2$O, the quadrupole moment of N$_2$, and the
impact parameter.  For self-broadened H$_2$O, the main interaction is
dipole-dipole and similar calculations can be performed.  This quantum
treatment of line widths represents a large improvement over simple kinetic
theory calculations.  Such calculations, however, are most often scaled to 
agree with experimental results to obtain the highest degree of accuracy 
and are included in spectral line databases whenever accurate measurements 
are not available or possible.

Experimental studies of line widths can become surprisingly complicated for
several reasons.  One common complication is due to the overlapping and
blending of adjacent spectral lines.  Others include excessive experimental
noise, badly-characterized instrument functions, incorrect ``background'' 
absorptions, and lack of characterization of the optical path.  Some of 
these concerns have been reviewed by Gamache {\it et.al.}\cite{gam:94}.
Furthermore, for some gases such as water vapor, experimental results from 
different investigators for the same spectral line lie well outside quoted 
uncertainties.  Larger systematic and analysis errors, not inaccurate
experimental spectra, are most likely responsible for these disagreements.
The case for N$_2$-broadened water vapor line widths is investigated in
detail in section \ref{chpt:h2o}.

\subsection{Lineshape Theories}\label{sec:shapetheory}

The frequency dependence of the absorption coefficient is determined by the 
molecule's physical state and its environment.  Broadening factors can be 
divided into three general classes.  They are (1) natural broadening,
(2) Doppler broadening, and (3) collision broadening.  While natural and 
Doppler broadening can be described with relatively simple theoretical
models, providing an accurate generalized collision broadening theory is a 
very challenging problem.  Each of these are addressed below.

\subsubsection{Natural Broadening}\label{sec:natural}

The {\it natural} lineshape is best described by considering a stationary,
isolated molecule. If such a molecule is allowed to absorb radiation,
undisturbed by any other form, it will eventually make a transition back 
to a lower level of internal energy.  Consequently, the molecule has a 
limited lifetime at any given energy level.  The resulting lineshape is 
given by
\begin{equation}
 k_{nat}(\nu)=\frac{S}{\pi}\left(\frac{\gamma_{nat}}
{(\nu-\nu_{0})^{2}+\gamma_{nat}^{2}}\right)
\end{equation}
where $\gamma_{nat}=1/\tau_{nat}$ is the ``natural'' line width.  Due to
the relatively long lifetimes of these undisturbed molecules, 
$\gamma_{nat}$ is very small, with values on the order of
$10^{-5}$ cm$^{-1}$.  For this reason, natural lineshapes are not 
observable 
under atmospheric conditions or with spectrometers of average resolution.

\subsubsection{Doppler Broadening}\label{sec:doppler}

The inhomogeneous {\it Doppler} lineshape is applicable to conditions 
encountered in 
the upper troposphere and stratosphere.  In these cases, the temperature is 
assumed to be high enough to produce molecular motion, but the pressure is 
low enough so that the molecules experience no collisions; or at least are 
not subject to {\it strong} collisions which terminate the dipole moment 
oscillation.
At pressures of about 5 Torr or less, the Doppler lineshape 
is predominant, with a typical line width of 0.001 cm$^{-1}$ at 296 K. 
The molecular motion produces an apparent shift in the observed frequencies
and such broadening is called Doppler broadening.

The shifted Doppler frequency, $\nu '$, for a molecule moving with a
speed $v_{m}$ along the direction of observation, relative to the observer,
is given by
\begin{equation}
 \nu '=\nu_{0}\frac{\sqrt{1-(v_{m}/c)^{2}}}{1+v_{m}/c}
\end{equation}
where $\nu_{0}$ is the un-shifted frequency. For $v_{m} \ll c$, $\nu '$
can be approximated with a binomial expansion as
\begin{equation}
 \nu '=\nu_{0}\left(1-\frac{v_{m}}{c}\right)
\end{equation}
Therefore, for each $v_{m}$, there exists a corresponding shifted frequency.
Given a Maxwell distribution of velocities within the gas, the density of
molecules with velocity $v_{m}$ is given by
\begin{equation}
 dn=N\left(\frac{m}{2\pi kT}\right)^{\frac{1}{2}}\exp\left(-\frac{m}{2kT}
v_{m}^{2}\right)dv_{m}
\end{equation}
$N$ is the total number of molecules, $m$ is the molecular mass, $T$ is the
temperature, and $k$ is Boltzmann's constant.  The corresponding absorption
coefficient, $k_{D}(\nu)$, like the Boltzmann distribution, has a
Gaussian form:
\begin{equation}
 k_{D}(\nu)=\frac{S}{\gamma_{D}}\sqrt{\frac{\ln 2}{\pi}} \; e^{-\ln 2
  \left(\frac{\nu -\nu_{0}}{\gamma_{D}}\right)^{2}}
\end{equation}
$\gamma_{D}$, the line width of the Doppler lineshape, is given by
$\nu_{0}\sqrt{\frac{2kT \ln 2}{mc^{2}}}$.    Notice how quickly the
Doppler lineshape goes to zero far from the line center due to the negative
exponential.  
%It is interesting to note that in the relativistic limit, a
%transverse Doppler shift also occurs, yielding a shifted frequency of
%$\nu_{0}\sqrt{1-\frac{v_{m}^{2}}{c^{2}}}$.  

\subsubsection{Collision Broadening}\label{sec:collision}

At pressures greater than $\sim$ 5 Torr, the collisions between molecules
must be addressed.  Collisions are the most important phenomenon to
contribute to broadening at these higher pressures.  In 1906 Lorentz showed
that line broadening takes place when absorbing molecules or atoms collide.
If one assumes that a collision takes place during the time in which
radiation is being absorbed, the coherency of the wave train is
interrupted.  This interruption results in a broadening of the spectral
line.  Quantum mechanically, pressure broadening is caused by the
broadening of the molecules' energy levels by fields produced by the
colliding molecules.  This is a complex
subject, and exact solutions for the absorption coefficient are found only
under certain approximations.  The exact treatment of this problem
requires the knowledge of the time-dependent quantum mechanical
wavefunction of an ensemble of colliding molecules.  In general, this has 
not been achieved to date and therefore the problem is often approached by 
developing empirical or semi-empirical models which simulate the system.
In the following sections, a model leading to the standard Lorentz
lineshape is presented, followed by descriptions of more complex techniques
of dealing with collisional broadening.

\subsubsection{Lorentz Lineshape}\label{sec:lorentz}

In the simplest treatment, the collisional lineshape is that of a
Lorentzian.  At high pressures, collisions occur often and it is unlikely
that a molecule is allowed to oscillate undisturbed for its entire
natural lifetime.  Instead, the molecule is usually perturbed by many 
collisions.  This model makes several assumptions which lead to a simple
solution for the absorption coefficient.  The molecule's dipole moment is
assumed to be oscillating with frequency $\nu_{0}$.  When a collision
occurs at time $t$, the oscillation terminates instantaneously.  No
natural damping is included because the time between collisions,
$t$, is much less than the natural lifetime, $\tau_{nat}$.  In other
words, $\exp(-t/\tau_{nat})=1$ for all times considered.

It is important to understand the assumptions which have been made for this
model.  One of the assumptions is called the {\it impact approximation}, 
which assumes that the time between collisions is much greater than the 
duration of a collision, $\tau_{dur}$, and therefore, the behavior of
the dipole moment during the collision is negligible.  In this case,
$\tau_{dur}$ is taken to be zero, corresponding to an instantaneous 
phase shift in the dipole moment.  These types of collisions are also
sometimes called {\it adiabatic} in that the system has no time to react
to the collisions.  The opposite of the impact approximation is called the
{\it quasi-static} approximation, in which the collision durations are
essentially assumed to be much larger than the time between collisions.
This point is addressed when statistical lineshapes are discussed.
Another assumption made here is that
of {\it strong collisions}.  A strong collision is taken to be an 
interaction which terminates the oscillation, leaving no memory regarding 
its orientation or other properties before the collision.
On the other extreme, {\it weak} collisions are those which have little or
no effect in disturbing the molecule.  In this case, collisional effects 
are only felt as a damping effect after a large number of weaker impacts.
Collisions are also assumed to involve only two molecules, and such 
collisions are referred to as {\it binary collisions}.  One final
assumption is that the molecules follow classical straight line
trajectories between collisions.  So in the Lorentz model, which is often 
called the billiard-ball model, colliding molecules can be thought of as 
quickly moving hard spheres which do not interact with one another 
until they actually touch each other.  

Fourier analysis of this model wavetrain leads to a spectral distribution of
the form
\begin{equation}
\mid {F}\{\mu(t)\} \mid^{2}= \frac{\sin^{2}
  [2\pi(\nu-\nu_{0})t/2]}{[2\pi(\nu-\nu_{0})]^{2}}
\end{equation}
This expression must be averaged over all possible values of $t$.
From the kinetic theory of gases, the distance traveled between collisions, 
$l$, by a molecule of average velocity $v_{m}$ has a Poisson distribution:
\begin{equation}
 p(l)dl=\frac{dl}{l_{m}}e^{-l/l_{m}}
\end{equation}
where $l_{m}$ is the mean free path.  Using $dt=\frac{dl}{v_{m}}$, the
distribution for the time between collisions is
\begin{equation}
 p(t)dt=\frac{dt}{\tau_{col}}e^{-t/\tau_{col}}
\end{equation}
where $\tau_{col}$ is the mean time between collisions.  Using this
distribution, the absorption coefficient, $k_{L}(\nu)$ is found to be
\begin{equation}
 k_{L}(\nu)=\frac{S}{\pi}\left(\frac{\gamma_{L}}
{(\nu-\nu_{0})^{2}+\gamma_{L}^{2}}\right)
\end{equation}
where $\gamma_{L}=1/\tau_{col}$ is the Lorentz line width.  Within this
billiard-ball model, $\tau_{col}$ is calculated as $l_{m}/v_{m}$ and has 
an average value of about $1.5 \times 10^{-10}$ seconds\cite{bre:56}.
This corresponds to a Lorentz line width of approximately 0.02 cm$^{-1}$, 
which is much larger than a typical Doppler width.  Thus, whenever 
collisions are present, they provide the primary form of broadening.  

This absorption coefficient is called the {\it Lorentz} 
lineshape.  It has the same form as the natural lineshape; the only
difference being the value of the line widths.  It is useful to compare 
the Doppler and Lorentz lineshapes.  The Doppler model assumes a Boltzmann 
velocity distribution, which goes smoothly to zero at large velocities.  Its
corresponding spectral distribution, therefore, also decays quickly in the 
far-wing (far from line center).  This is not the case for the Lorentz
model, which assumes instantaneous behavior during collisions.  The effect
of this unphysical temporal behavior is the placement of extremely high
frequency components in the lineshape's spectral distribution.
Consequently, $k_{L}(\nu)$ is too large in the far-wing, and the Lorentz
model predicts too much absorption in this region.

Despite the apparent shortcomings of the model used for the Lorentzian line
shape, it is very accurate for many applications.  The Lorentz lineshape is
accurate as long as two conditions are satisfied: (1) the spectral
region of interest is not too far removed from the line center where the 
impact approximation results in the prediction of too much absorption, and 
(2) there exists no significant overlapping of adjacent 
spectral lines.  The latter of these
two conditions arises because the Lorentz theory assumes no transfer of
intensity from one spectral line to another (often called ``line mixing'').
Experimental deviations from the Lorentz lineshape within $\sim$2--4
$\gamma_0$ of line center of isolated lines have not been confirmed for
systems of atmospheric interest.
%Other collisional broadening theories which take these effects into
%account have been developed and will now be reviewed.

\subsubsection{Van Vleck - Weisskopf Lineshape}\label{sec:vvw}

When describing the procedures used to calculate
the Lorentz and natural lineshapes, the assumption that transitions occur
at relatively high frequencies (i.e. infrared) was made.  When
computing the Lorentz lineshape, the Fourier transform of the dipole moment 
actually yields two terms -- one centered about $\nu_{0}$ and
the other about $-\nu_{0}$. The lineshape should be written as 
\cite{van:45}
\begin{equation}
k(\nu)=\frac{S}{\pi}\left(\frac{\gamma}{(\nu-\nu_0)^{2}+\gamma^{2}}
+ \frac{\gamma}{(\nu+\nu_0)^{2}+\gamma^{2}}\right).
\label{eqn:vvw1}
\end{equation}
This shape is more often used in the microwave
region of the spectrum, where the second term of the sum is not negligible.
For molecules active in the infrared region, however,  $\nu_{0}$ is large
enough such that $(\nu-\nu_{0}) \ll (\nu+\nu_{0})$ and the resulting
lineshape can most often be safely approximated as Lorentzian.  An
exception is in ``window'' regions (far from any line centers).
Another modification to
the Lorentz model involves the behavior of the molecule directly after a 
collision.  In the Lorentz model, we essentially assumed the wave-function 
experienced {\it random} phase shifts during collisions and immediately
began oscillating again at its resonant frequency.  The wave-function,
however, does not experience a random reorientation, but should be
distributed according to the Boltzmann distribution of the field when the
collision occurs\cite{van:77}.  Following this approach leads to a slight 
modification of Equation \ref{eqn:vvw1}:
\begin{equation}
k(\nu)=\frac{S}{\pi}\left(\frac{\nu}{\nu_0}\right)^2 
\left(\frac{\gamma}{(\nu-\nu_0)^{2}+\gamma^{2}}
+ \frac{\gamma}{(\nu+\nu_0)^{2}+\gamma^{2}}\right)
\label{eqn:vvw2}
\end{equation}
which is commonly called the {\it Van Vleck-Weisskopf} lineshape 
\cite{van:45}.
Equation \ref{eqn:vvw2} is an improvement over Equation \ref{eqn:vvw1} in
that (\ref{eqn:vvw1}) predicts no absorption in the limit of zero resonant
frequency, which is not observed experimentally.  Equation \ref{eqn:vvw2}
is also more acceptable in that it agrees with 
Debye's\cite{van:77} relaxation theory in the same limit.

\subsubsection{Van Vleck - Huber Lineshape }\label{sec:vvh}

Another similar lineshape which was developed to satisfy the principle of
{\it detailed balance} (discussed below) is\cite{van:77}
\begin{equation}
k(\nu)=\frac{S}{\pi}\left(\frac{\nu}{\nu_0}\right)
\frac{\mbox{tanh}(hc\nu/2kT)}{\mbox{tanh}(hc\nu_0/2kT)}
\left(\frac{\gamma}{(\nu-\nu_0)^{2}+\gamma^{2}}
+ \frac{\gamma}{(\nu+\nu_0)^{2}+\gamma^{2}}\right)
\label{eqn:vvh}
\end{equation}
which is called the {\it Van Vleck-Huber} lineshape.

\subsubsection{Voigt Lineshape }\label{sec:voigt}

Before going on to explain more elaborate models,
the {\it Voigt lineshape} should be introduced.  It does not
introduce any new physical insight into broadening phenomenon, but is very
useful computationally.  The Voigt lineshape is the convolution of the
Doppler and Lorentz lineshapes.  For this reason, it assumes Doppler
characteristics at low pressure and Lorentz characteristics at higher
pressures.  Therefore, one single expression for the lineshape can be used
throughout a wide range of pressures.  The Voigt lineshape, $k_{V}(\nu)$
is given by
\begin{equation}
 k_{V}(\nu)=\frac{k_{0}y}{\pi}\int_{-\infty}^{\infty} \frac{e^{-t^{2}}}
{y^{2} + (x-t)^{2}}dt
\end{equation}
with
\begin{equation}
 k_{0}=\frac{S}{\gamma_{D}}, \; \; \;
y=\frac{\gamma_{L}}{\gamma_{D}}\sqrt{\ln 2}, \; \; \;
x=\left(\frac{\nu-\nu_{0}}{\gamma_{D}}\right)\sqrt{\ln 2}
\end{equation}
The Voigt lineshape does assume there is no correlation between collision
cross sections and the relative speed of the colloids.  Again, for
atmospheric systems this approximation appears to be quite accurate.

The VanVleck-Huber lineshape can be computed using the Voigt lineshape 
instead of the Lorentz lineshape.

\subsubsection{General Techniques for Calculating Collisional
Lineshapes }\label{sec:gentechs}

The lineshape models have been presented informally in order to provide 
general physical insight.  However,
for more advanced approaches, it is useful to understand the more formal
techniques in which absorption coefficients are calculated.  This is needed
to help understand the deviations from Lorentz lineshapes. For more on 
this, one is refered to Dave Tobin's \cite{tob:96} thesis for a 
discussion, as well as for more references. In particular, his 
dissertation describes line mixing for carbon dioxide, as well as 
intermolecular forces and potentials used in lineshape calculations for 
water vapor.

%%%%%%%%%%%%%%%%%%%%%%%%%%%%%%%%%%%%%%%%%%%%%%%%%%%
\section{Water vapor lineshape}\label{sec:local}
When computing water wapor spectral lineshapes, the effects of the lines
far away from the current region can be included in two ways : by directly 
individually adding on the far wings of each line to the current region, or
just using the lines in the current region, plus a lump sum ``continuum'' 
contribution. As the far wings lineshapes might not be lorentz, and also to 
account for the possibility that the near wing lineshapes might also not
be lorentz, a local lineshape definition is preferentially used, along with 
a continuum contribution. 

The local lineshape is defined as the lorentz lineshape out to 
$\pm 25 cm^{-1}$
minus the lorentz value at 25$cm^{-1}$ away from line center. With this 
definition, the possibility that the lineshape near linecenter is itself not
lorentz, can be modelled by including the effect into the continuum that has
to be added on.

Non-Lorentz H$_2$O lineshapes also have a
significant impact {\it within} the strong pure rotational and vibrational
bands.  This {\it in-band} continuum is
particularly important for satellite infrared remote sensing of atmospheric
H$_2$O profiles.  

For well isolated pressure-broadened water vapor lines in the infrared, the
Lorentz lineshape is very accurate near line center.  However, if one uses
a Lorentz lineshape, this generally overestimate the observed 
absorption in the far-wing atmospheric window regions and underestimates the
absorption within the rotational and vibrational bands. This means that 
the actual water vapor lineshape is extremely sub-Lorentzian in the 
far-wing and at least somewhat super-Lorentzian in the intermediate
and near-wing.  Most experimental studies have focussed on the window 
regions and so the far-wing lineshape has been studied more than 
the near-wing (roughly within 5 to 25 cm$^{-1}$ of line center).

Deviations of H$_2$O spectral lineshapes from Lorentz have been studied
extensively for the atmospheric windows at 4 and 10 $\mu$m.  In general,
these deviations are observed to vary slowly with wavenumber and the 
anomalous absorption has become known as the water vapor continuum.  

Several characteristics were found to be common to all window region 
continuum studies.  In general, the continuum absorption 
\cite[for example]{bur:81,gra:90}: (1) does not change rapidly
with wavenumber, (2) decreases rapidly with increasing temperature for
pure water vapor, (3) is greater for self-broadening than for foreign
broadening, (4) is more significant in regions of weak absorption
than in regions of strong absorption, and (5) displays the pressure
dependencies associated with gaseous absorption.

It is accepted that the deviations from the impact theory calculations in 
window regions are due to the non-Lorentz behavior of the 
far-wings of pure rotational and vibration-rotation water vapor 
absorption lines.  

In the following sections, a review of previous studies of this continuum 
absorption is presented.  These can be separated into two generally 
different approaches.  The first, which is most often adopted in 
experimental
studies, is to express the observed deviations from Lorentzian behavior
through the use of {\em continuum coefficients}.  With this method, the
cumulative effects of all lines are characterized in a convenient form.
The second approach, which provides more information about the shape of 
individual spectral lines, is used in most theoretical studies.
$\chi$-functions are often the end result of this approach.

\subsection{A Definition of the Continuum}

The definition proposed by Clough is widely used in atmospheric 
spectroscopy and radiative transfer, particularly in line-by-line codes 
such as FASCODE\cite{clo:81}, GENLN2\cite{edw:87}, LINEPAK\cite{gor:94}, 
and LBLRTM.  The ``local''
absorption for a single transition is defined as a Lorentz lineshape out to
$\pm$25cm$^{-1}$ from the line center, minus the Lorentz value at 25cm
$^{-1}$. For several lines, the local absorption is expressed 
as\cite{clo:89}
\begin{equation}
k_{local}(\nu)=\nu\tanh\left(\beta\nu/2\right) \rho_{ref}
\frac{T_{ref}}{T}P_{H_{2}O}L
\sum_i\frac{S_i}{\pi}\left\{\begin{array}{cl}
\frac{\gamma_i}{\Delta\nu^{2}+\gamma_i^2}-\frac{\gamma_i}{25^2+\gamma_i^2}
                        & \mbox{if $|\Delta\nu| \leq 25 \mbox{cm}^{-1}$} \\
        0               & \mbox{if $|\Delta\nu| > 25 \mbox{cm}^{-1}$}
\end{array}\right.
\label{eqn:local_def}
\end{equation}
where $T_{ref}=273.15K$, $\rho_{ref}$ is the absorber number density per
atmosphere at $T_{ref}$, $\beta=hc/kT$, and $\Delta\nu=\nu-\nu_i$.
All of the continuum measurements presented in this work are consistent
with Equation \ref{eqn:local_def}.  
This is actually a slight modification of Clough's
definition\footnote{Equations 6 through 8 of Reference
\protect\cite{clo:89} do not reflect the local lineshape definition used
in FASCODE.  They actually lead to a $\chi$ dependent local lineshape, 
which is not used in the line-by-line codes.}, which
also includes the negative resonance terms ($\frac{\gamma_i}{(\nu+\nu_i)^2
+\gamma_i^2}$).  In the infrared region (actually for $\nu > 25$
cm$^{-1}$), the two definitions are equal.  {\em The continuum is then 
simply defined to be any observed absorption not attributable to the local
absorption}.  The continuum therefore includes 
far-wing absorption (beyond 25cm$^{-1}$ from line center), absorption 
due to any near-wing (within 25cm$^{-1}$) non-Lorentz behavior, and 
the Lorentzian value at 25cm$^{-1}$ within $\pm$25cm$^{-1}$ of line 
center (this is often called the ``basement'' term).  This is illustrated 
in Figure \ref{fig:localdef} for a single absorption line.
The ``basement'' term is a relatively minor part of the continuum and is 
introduced to ensure a smooth continuum for computational reasons.  

\begin{figure}[h]
 \begin{center}
 \includegraphics[width=3.5in]{Figures/H2O/local_def.eps}
 \end{center}
 \caption[The local lineshape definition.]{The local lineshape definition
 used in this work.  The far-wing (beyond 25 cm$^{-1}$), near-wing (within
 25 cm$^{-1}$), and basement components of the continuum are labeled.}
 \label{fig:localdef}
\end{figure}

With this definition of the local absorption, the continuum is always a 
positive quantity.  The basement and far-wing
components are certainly always positive.  The near-wing component, which
represents the difference between the actual lineshape and Lorentz within
25 cm$^{-1}$, is also positive because water vapor has a super-Lorentzian
lineshape in this region. In fact, with this continuum definition, a 
non-zero continuum exists even for the Lorentz lineshape.
A calculation of the total absorption coefficient and the continuum 
absorption (total minus local) using the Lorentz lineshape for the 
0-4000 cm$^{-1}$ region is shown
in Figure \ref{fig:lor_con}.  All of the high frequency components of the
absorption are contained in $k_{local}$ and the continuum is therefore a 
smoothly varying function, which can be stored in a look-up table for ease 
of computation.

\begin{figure}[h]
\begin{center}
\includegraphics[width=3.55in]{Figures/H2O/lor_con.eps}\end{center}
\caption[Lorentz calculations for 0-4000 cm$^{-1}$.]{Absorption coefficient
	calculations using the Lorentz lineshape for 0-4000 cm$^{-1}$ at
	$\sim$2 cm$^{-1}$ resolution.  The total
	(solid curve) and continuum (dashed curve, as defined by Equation 
	\protect\ref{eqn:local_def}) absorption coefficients are shown.  The
calculations include the effects of all lines between 0 and 5000 cm$^{-1}$.
Conditions are: 1 torr H$_2$O, 760 torr N$_2$, 12 m path length, 296K.}
\label{fig:lor_con}
\end{figure}

In the line-wings, the displacement $\Delta\nu$ is much greater than the
halfwidth $\gamma$ and the Lorentz terms can be approximated as 
$\frac{\gamma_i}{\Delta\nu^2}$.  This leads to a quadratic pressure
dependence in the absorption coefficient on $P_{H_2O}$ for self-broadened 
water vapor and a linear dependence on both $P_{H_2O}$ and the broadening 
pressure, $P_f$, for foreign broadened water vapor.  Since the continuum 
is mainly due to line wings, the total continuum absorption coefficient 
for all lines is formulated as \cite{bur:81,clo:89}
\begin{equation}
k_{con}(\nu)=\nu\tanh\left(\beta\nu/2\right) \rho_{ref}
\frac{T_{ref}}{T}P_{H_{2}O}L
\frac{296}{T}\left(P_{H_{2}O} C_s^0(\nu,T) + P_{f} C_f^0(\nu,T)\right)
\label{eqn:kcon}
\end{equation}
where $C_s^0$ and $C_f^0$ are the self- and foreign-broadened 
continuum coefficients at 296K and 1 atmosphere.  To express experimental
and theoretical results, the quantities $C_f^0$ and $C_s^0$ are often used.

Because absorption in the windows is very weak, all spectra gathering 
techniques require very long path lengths.  The laboratory studies have 
focussed primarily on self- and nitrogen broadening at room temperature or 
above, while most atmospheric measurements have naturally looked at 
air-broadening at room temperature or below.  Most of the results from 
these measurements are in accord with
those of Burch {\it et.al.}, which are discussed below.

\subsection{Laboratory Measurements of Burch {\it et.al.}}

The most notable experimental studies of the water vapor continuum
within the fundamental $\nu_2$ band, are those of Burch and 
co-workers\cite{bur:67,bur:81,bur:82,bur:84,bur:85}.  
The work was carried out a number of years ago at
relatively low spectral resolution ($\sim$0.5 cm$^{-1}$ at 1500 cm$^{-1}$). 
Self and nitrogen broadened spectra were collected for a range of
temperatures and pressures.  Within the band, 
the wavenumber regions were chosen to be at the center of
the so-called microwindows such that the absorption due to lines within 
$\sim$1cm$^{-1}$ could be ignored and so the low instrument resolution
would not distort the spectra.  This led to a relatively small number of 
continuum measurements (13 points
between 1400 and 1850 cm$^{-1}$).  A compilation of these measurements is
shown in Figure \ref{fig:burch_data1} for self-broadened water vapor and in
Figure \ref{fig:burch_data2} for nitrogen-broadened water vapor.  
The continuum coefficients shown in these figures have been modified
\footnote{Thanks to S. A. Clough for providing these data.} from
their published values to be consistent with Equation \ref{eqn:local_def}.
Van Vleck-Weisskopf continuum coefficients are also shown
for comparison.  Obvious conclusions from these measurements is that the 
impact theory predicts too much absorption in the window regions at 4 and 
10$\mu$m and not enough within the rotation-vibration bands. Due to a lack 
of other laboratory measurements and because these in-band regions are 
dominated by near-wing lineshapes, Burch's measurements have led to most 
of our present understanding of near-wing lineshapes and have formed the
benchmark for theoretical comparisons within the $\nu_2$ band. 
Furthermore, they have been used to
produce empirical lineshapes which are used in operational radiative
transfer codes and band models which in turn are used in Global Climate
Models.  The accuracy of these measurements is therefore crucial for several
applications.

\begin{figure}[h]
\begin{center}
\includegraphics[width=5in]{Figures/H2O/burch_cs0.eps}\end{center}
\caption[Burch's measurements of $C_s^0$, the self-broadened continuum
coefficients, and impact theory calculations
from 0 to 3000 cm$^{-1}$.]{Burch's measurements of $C_s^0$ and impact
theory calculations (solid curve) from 0 to 3000 cm$^{-1}$.
Measurements are shown for temperatures of 296K (circles), 308K (+'s), and 
322K (x's), and 430K (asterisks).}
\label{fig:burch_data1}
\end{figure}

\begin{figure}[h]
\begin{center}
\includegraphics[width=3.5in]{Figures/H2O/burch_cf0.eps}
\end{center}
\caption[Burch's measurements of $C_f^0$, the nitrogen broadened continuum
coefficients, and impact theory calculations from 0 to 3000 cm$^{-1}$.] 
{Burch's measurements of $C_f^0$ and impact theory calculations 
(solid curve) from 0 to 3000 cm$^{-1}$.  Measurements are shown for 
temperatures of 296K (circles), 308K (+'s), and 353K (x's).}
\label{fig:burch_data2}
\end{figure}

A good review of progress in the theoretical studies of the water
continuum is is Tobin's thesis, and is not reproduced here. 

\subsection{Clough's CKD Continuum Models}\label{sec:ckd}

Theoretical approaches have been oriented toward modeling 
far-wing non-Lorentz behavior in order to predict radiative transfer in the
atmospheric windows and consequently are least accurate in the
near-wing of the line, which is also more difficult to handle theoretically.
The very near-wing lineshape is determined primarily by the long
range interaction and the number of collisions per second experienced by
the absorber.  The far-wing is determined by the durations of the very
fast, close collisions which are governed by the strong repulsive
potential at small separation distances.  The
near/intermediate-wing lineshape, however, requires an accurate description
of the absorber-perturber interaction for all intermediate time scales, as
well as for all separation distances and angular orientations.  This
becomes a complicated two-body problem involving not only the long range
interaction, but also other short-lived electrostatic interactions which 
arise at shorter separation distances.

Lacking an accurate intermediate-wing lineshape model, 
Clough {\it et.al.}\cite{clo:89,clo:95}
have developed empirical models of the continuum to obtain the best overall
agreement with measurements in  both the window regions and within the
bands.  These models are widely used in a variety of atmospheric 
applications.  Clough's model expresses the absorption coefficient as
\begin{equation}
k(\nu)=\nu\tanh(\beta\nu/2)\sum_i\frac{S_i}{\pi}\left(
\frac{\gamma_i}{(\nu-\nu_i)^2 + \gamma^2}\chi(\nu-\nu_i) +
\frac{\gamma_i}{(\nu+\nu_i)^2 + \gamma^2}\chi(\nu+\nu_i)\right)
\label{eqn:clough}
\end{equation}
$\chi$ is an empirical function used to include non-Lorentz behavior and 
is given by
\begin{equation}
\chi=\left(\begin{array}{cc}
1-(1-\chi^\prime)\frac{(\nu\pm\nu_i)^{2}}{25^2} & 
\mbox{for} \; |\nu\pm\nu_i|\le 25 \mbox{cm}^{-1} \\
\chi^\prime & \mbox{for} \; |\nu\pm\nu_i|\ge 25 \mbox{cm}^{-1} \\
\end{array}\right.
\label{eqn:ckdv0}
\end{equation}
where $\chi^\prime$ is determined empirically.  These models, which are
known as the CKD (Clough, Kneizys, Davies) continuum models, are shown in 
Figures \ref{fig:chi_self} and \ref{fig:chi_frgn}, along with the results
of Rosenkranz and Ma and Tipping.  

Note that there are separate chi functions for the self and foreign 
contributions to the continuum; hence 
\[
\chi = \frac{P(h20) \times \chi_{Pself} + P(for) \times \chi_{for}}
            {P(h20) + P(for)}
\]

The functional form of Equation \ref{eqn:ckdv0} was chosen because of the
known near and far-wing behavior.  The $\chi$-functions were held
fixed at 1 at line center, adjusted to some value in the intermediate wing, 
and then forced to decay exponentially in the far-wing.
Because Burch's measurements included in-band as well as window
region measurements,  Clough's $\chi$-functions were forced to assume the
correct overall behavior in the near, intermediate, and far-wing.  
It should be noted that the form of the $\chi$-functions between 0 and 25
cm$^{-1}$ has no direct physical justification, but was adopted simply to
interpolate between the line center and fitted region.

The original CKD model (CKDv0) was developed by fitting
Equation \ref{eqn:clough} (with the use of line parameters from the 1986
HITRAN database) to match the laboratory continuum measurements of Burch,
which are shown in Figures \ref{fig:burch_data1} and \ref{fig:burch_data2}.

\begin{figure}[h]
\begin{center}\includegraphics[width=3.5in]{Figures/H2O/cs0_1.eps}
\end{center}
   \caption[Self-broadened continuum coefficients of Ma and Tipping, Burch
	and Clough.]{Self-broadened continuum coefficients of Ma and
Tipping (dashed curve), Burch (circles, 296K; pluses, 308K; x's, 322K),
	CKDv0(solid curve), CKDv1(dotted curve), CKDv2.1(dash-dot curve).}
   \label{fig:cs0_1}
\end{figure}

\begin{figure}[h]
\begin{center}\includegraphics[width=3.5in]{Figures/H2O/cf0_1.eps}\end{center}
   \caption[Nitrogen-broadened continuum coefficients of Ma and Tipping, 
  Burch and Clough.]{Nitrogen-broadened continuum coefficients of Ma and 
Tipping (dashed curve), Burch (circles, 296K; pluses, 308K; x's, 322K),
	CKDv0(solid curve), CKDv1(dotted curve), CKDv2.1(dash-dot curve).}
   \label{fig:cf0_1}
\end{figure}

Burch's measurements and continuum coefficients computed with CKDv0 and by
Ma and Tipping are shown in Figures \ref{fig:cs0_1} and
\ref{fig:cf0_1}.  Also shown are two subsequent CKD models:
CKDv1 and CKDv2.1.  These newer models represent modifications to CKDv0 made
to maintain agreement with measurements.  For instance, $C_s^0$ 
of the CKDv1 model is given by
\begin{equation}
C_s^0=C_s^0(\mbox{version} 0) \left( 1 - 0.2333\frac{200^2}{(\nu-1050)^2 + 200^2}\right)
\end{equation}
which results in a 23 percent reduction at 1050 cm$^{-1}$, yielding better
agreement with HIS field measurements, yet remaining compatible with
Burch's laboratory results.  Thus, there are no $\chi$-functions
associated with the later CKD versions.  Measurements which have led to CKDv2.1
are discussed in the next section.  

The data shown in Figures \ref{fig:cs0_1} and \ref{fig:cf0_1} for Ma and
Tipping's calculations were provided by Q. Ma.  For these calculations, a
Lorentz lineshape was used within 25 cm$^{-1}$ of line centers due the
inaccuracy of their $\chi$-functions in the near wing.  Thus, their
coefficients do not include any non-Lorentz behavior within 25 cm$^{-1}$.
%In other words, they feel that their most accurate lineshape is: Lorentz
%within 25 cm$^{-1}$ of line center, and Lorentz $\cdot\chi$ beyond 25
%cm$^{-1}$.  
Their coefficients have also been modified from their original form
by this author to include the basement terms to achieve consistency 
with the continuum definition.

\begin{figure}[h]
\begin{center}
  \includegraphics[width=3.5in]{Figures/H2O/con_part2.eps}\end{center}
  \caption[Far-wing, near-wing, and basement contributions to the total continuum.]
	{The individual contributions of the far-wing 
    (beyond 25 cm$^{-1}$), near wing (within 25 cm$^{-1}$), and 
    ``basement'' components to the total continuum
    absorption based on the CKDv0 $\chi$-function.}
  \label{fig:parts}
\end{figure}

With an accurate description of the $\chi$-function responsible for the
in-band water vapor continuum, its relation to the observed absorption can
be presented.  In window regions, $C_s^0$ and $C_f^0$ vary smoothly with 
wavenumber since there are relatively few strong local absorption lines.  
Within strong absorption 
bands, however, the continuum may be influenced more by high frequency 
near-wing non-Lorentz behavior and these continuum coefficients may not 
vary as smoothly with wavenumber.  Figure \ref{fig:parts} shows the 
individual contributions of the far-wing, near-wing, and basement components 
to the total continuum (as described in Figure \ref{fig:localdef}) in the 
$\nu_2$ region based on the CKDv0 $\chi$-function.  Although the 
continuum is dominated by far-wing effects in the wings of the $\nu_2$ band, 
the near-wing contributions dominate the continuum at the peaks of the P- and 
R-branches.  By studying the continuum in this region, information about the
near and intermediate wing lineshape can be determined.


\subsection{Recent Field Measurements}

\begin{figure}[h]
\begin{center}\includegraphics[width=3.5in]{Figures/H2O/theriault.eps}
\end{center}
  \caption[Nitrogen broadened continuum coefficients reported by
	Theriault.]{Nitrogen broadened continuum coefficients reported by
	Theriault. The CKD models are also shown: solid, CKDv0; dashed,
	CKDv1; dotted, CKDv2.1.}
  \label{fig:theriault1}
\end{figure}

Recent field measurements have led to modifications in the CKD models.
Th\'{e}riault {\it et.al.} \cite{the:94} recently recorded 
atmospheric transmission spectra over a horizontal path of 5.7 km.  These
spectra were utilized to test the accuracy of the existing H$_2$O
continuum.  Their results suggest that, in the wings of the $\nu_2$ band,
the foreign component of the CKDv0 continuum had to be decreased by
approximately a factor of 2 to recover good model-measurement agreement.
The accuracy of these findings, however, is limited by the nature of the
experiment and the difficulties in characterizing the optical path.
These measurements are shown in Figure \ref{fig:theriault1}.

Atmospheric emission spectra from the University of Wisconsin's High 
Resolution Interferometer (HIS)\cite{the:94,smi:90,clo:88} also show 
that large errors remain in CKDv0.  These spectra are 
very sensitive to the continuum and provide an excellent test of 
the models when the atmospheric state is well characterized.  
Differences between observed and calculated high-resolution atmospheric 
radiances can be as large as 4K in brightness temperature in-between lines 
inside the $\nu_{2}$ band when using CKDv0\cite{the:94,rev:90,rev:89}.

Clough recently modified his continuum model to obtain better 
agreement with several types of atmospheric observations
\cite{clo:95,the:94,clo:88}.  In particular, the atmospheric studies
suggested that the nitrogen-broadened continuum near $\sim$1200 cm$^{-1}$
was approximately a factor of 2 smaller 
than predicted by CKDv0.  This improved model (CKDv2.1) reduces the maximum 
errors in HIS calculated radiances to $\sim$2K in-between spectral lines in 
the 1200 cm$^{-1}$ region \cite{str:95}.  However, these errors are highly 
dependent on wavenumber and the newer model also appears to have
produced slightly worse results (as compared to CKDv0) further into the 
band, near the band center.  Errors due to CKDv0 and CKDv2.1 for these 
measurements have been shown in Figure \ref{fig:camex_nu2}.

\section{Carbon dioxide lineshape}
Remote sensing of atmospheric temperature and humidity from satellites is
dependent on the ability to calculate observed radiances as a
function of the atmospheric state.  This so--called ``forward problem'' is 
at the heart of physically--based retrieval algorithms.  Advanced infrared
sounders such as the Atmospheric Infrared Sounder (AIRS)\cite{air:91},
which is scheduled to fly on the Earth Observing System (EOS) PM platform, 
will measure radiances in thousands of spectral channels between 3.7$\mu$m 
and 15$\mu$m.  Physical retrieval algorithms use
some subset of these channel radiances to determine global atmospheric
temperature and humidity, as well as many other atmospheric and surface
parameters.

The 4.3$\mu$m spectral region, which is dominated by the $\nu_3$
vibrational band of CO$_2$, is particularly useful in retrieving atmospheric
temperature profiles.  This region and several Q-branches in the 15$\mu$m
region are useful because the absorption varies strongly with wavenumber, 
thus allowing many levels of the atmosphere to be probed within a very
narrow spectral range.  

Deviations from Lorentzian lineshape behavior in CO$_2$ are observed 
primarily in two
cases: (1) when there is significant overlap of adjacent spectral lines 
and (2) when the spectral region of interest is far from the line center.
The Lorentz model fails in these two cases because (1) it neglects the
interaction between spectral lines due to inelastic collisions while
absorption is occurring and (2) it treats collisions between 
molecules as if they were instantaneous.  These two phenomenon are
frequently referred to as line-mixing and the duration-of-collisions
effect, respectively.  A lineshape which is accurate for the $\nu_3$ band 
of CO$_2$ must include both effects.

The lineshape proposed here takes into account the effects of both 
line-mixing and non-zero collision times.  The line-mixing 
model follows the basic formalism developed by Smith\cite{smi:81} 
and Rosenkranz\cite{ros:88}.  In this work, instead of doing involved 
calculations of the the relaxation rates as in 
\cite{cou:86,boi:89,bon:91}, a simple model to approximate the
rates is used.  Non-zero collision times are modeled with Birnbaum's 
autocorrelation approach\cite{bir:82}. 

The following section reviews the line-mixing formalism, which is (in
principle) not specific to any spectral region or molecule.  The
application to the $\nu_3$ $\Sigma\leftarrow\Sigma$ band of CO$_2$ is made
in Section \ref{sec:apply43}.  

\subsection{Line Mixing}\label{sec:lm_general}
Deviations from the Lorentz lineshape in regions of overlapping spectral
lines have been observed in many cases
\cite{bar:58,ben:66,fan:58*1,kol:58,gor:67} .  In particular, large
deviations are found in infrared Q-branches, where the spectral lines are
very closely spaced.  Most attempts \cite{bar:58,fan:58*1,kol:58}
to account for line-mixing have used the
impact approximation.  This simply means that the theory
treats the collisions between molecules as instantaneous and will therefore
be accurate only in spectral regions close to the line centers.  

Within an ensemble of colliding molecules, a single molecule is not a
conservative system.  Through interactions with the other molecules, its
internal energy can be transformed into other forms of energy throughout
the ensemble.  For this reason, if the system is defined as a single 
molecule, the energy of the system does not have to be conserved.  
Quantum mechanically,
this means that the Hamiltonian of the system does not have to be Hermitian
and the system can have imaginary energies.  The total Hamiltonian 
of the molecule is given by $H(t)=H_{0}(t)+ H_{1}(t)$.  In this
expression, $H_{0}(t)$ is the Hamiltonian of the molecule without any
interaction with its perturbers.  $H_{0}(t)$ therefore has real eigenvalues
and its eigenvectors are the stationary states of the molecule.  
$H_{1}(t)$ is the Hamiltonian representing the interaction of the molecule
with its perturbers. It could, for example, represent the effect of 
inelastic collisions.  $H_{1}(t)$ has energies which are partially 
imaginary.

Collision broadening is determined by the behavior of a molecule when
making a transition between two known states.  This implies that 
transitions should not be described by two independent functions which 
model each state separately, but by a wavefunction which represents the 
evolution from one state to another.  This evolution is described by the
evolution operator, $T(t)$.  According to the Schroedinger equation, $T(t)$ 
is given by $e^{-\imath(H_{0}+H_{1})\frac{t}{\hbar}}$.
Let the initial and final states of a transition be given by $\Phi_{i}$ and
$\Phi_{f}$.  Since both of these are stationary states of
the molecule, they are solutions to $H_{0}\Phi=E_{0}\Phi$ and have totally 
real energies.  For a transition of duration $\tau_{t}$, which begins
at time $t=0$, the total wavefunction of the system at various times is:
\begin{equation} 
  \Psi(t=0) \; = \; T(0)\Psi(0) \; \equiv \; \Phi_{i}
\end{equation}
\begin{equation} 
  \Psi(0\leq t\leq\tau_{t}) \; = \; T(t)\Phi_{i} \; = \Phi_{i}
e^{-\imath(H_{0}+H_{1})\frac{t}{\hbar}}
\end{equation}
\begin{equation} 
  \Psi(t=\tau_{t}) \; = \; T(\tau_{t})\Phi_{i} \; = \; \Phi_{i}
e^{-\imath(H_{0}+H_{1})\frac{\tau_{t}}{\hbar}} \; \equiv \; \Phi_{f}
\end{equation}
Since $H_{1}(t)$ has imaginary eigenvalues, $\Psi(t)$ is subject to a
damping term which drains energy from the system, therefore allowing for
absorption.  To calculate the absorption coefficient from this model, the 
autocorrelation function of the electric dipole moment $\mu(t)$ is used.
The autocorrelation of $\mu(t)$ is given by
$\phi(t)=<\mu(0)\cdot\mu(t)>$, where $< >$ denotes a statistical average 
over the entire system.  Because $\mu(t)$ has contributions from
many different elements of the system, $\phi(t)$ will, in general, involve 
a complicated sum.  The absorption coefficient is then calculated by 
taking the Fourier Transform of $\phi(t)$ \cite{ bir:82}.  An
expression for the correlation of the electric dipole moment of this system
over the duration of a transition is written as \cite{bar:58}
\begin{equation} 
\phi(t)= <\mu(0)\cdot\mu(t)>= \sum_{if} \rho_{i}[<\Phi_{i}(0)|{
\bf
  d}|\Phi_{f}(0)>\cdot <\Phi_{f}(t)|{\bf d}|\Phi_{i}(t)>]
\end{equation}
where the sum is taken over all possible states.  ${\bf d}$ is the dipole
moment operator of the system and $\mathbf{\rho}$ is the density matrix, 
which is related to the relative population of states and is assumed to be 
constant during a transition.  Substituting the evolution operator into 
this expression, the time dependence of the correlation function is 
obtained. It is here that the impact approximation must be made. This 
allows the interaction
Hamiltonian, $H_{1}(t)$, which is generally a fluctuating, time-dependent
interaction, to be represented by an effective constant interaction
\cite{bar:58}.  Since $H_{1}$ is taken to be a constant in time, 
its effect on the correlation function is greatly simplified.  If this 
assumption were not made, the time dependence of $H_{1}$ would 
be required.  Instead, only the ``time-averaged''
interaction is needed and the details of its time dependence are
ignored.  With this approximation, the lineshape is then calculated by
taking the Fourier transform of $\phi(t)$.  Baranger \cite{bar:58}
calculated the lineshape to be
\begin{equation} k_{mix}(\nu)=\frac{\rho}{\pi} {\bf IM} \left(
\sum_{i}<\Phi_{i}|[D(\nu-H_{0}-H_{1})^{-1}]|\Phi_{i}>\right)
\end{equation}
where the sum is over all initial states and 
$D=\sum_{f}{\bf d}|\Phi_{f}><\Phi_{f}|{\bf d}$.  If $H_{1}$ is
diagonal, with diagonal elements of $\imath \gamma_{j}$, where
$\gamma_{j}$ are the Lorentz widths of the spectral lines, the
resulting lineshape is the sum of many Lorentzians:
\begin{equation} k(\nu)=\sum_{j} \frac{d_{j}^{2}\rho_{j} \; \gamma_{j}}
{(\nu-\nu_{0})^{2}+\gamma_{j}^{2}}
\end{equation}
where $d_{j}^{2}\rho_{j}$ is essentially the line strength. This Lorentzian
result shows that the model obeys the impact approximation.  It also
demonstrates the role of the off-diagonal elements of the interaction 
potential -- they are the cause of interaction between spectral 
transitions. If the off-diagonal elements of $H_{1}$ are non-zero, 
intensity can be transferred from one line to another, with the 
``amount'' of line-mixing determined by the magnitude of the 
corresponding off-diagonal element of
$H_{1}$.  This is why the phenomenon is often called line-mixing.  Thus, a
vibration-rotation band cannot be regarded as the simple sum of individual
transitions, but must be treated as a complex interacting system.  The
interaction, which is discussed later, is provided by a bath of inelastic
collisions.  Inelastic collisions between an absorbing molecule and a
perturber can cause the the absorber to gain or lose rotational energy.  If
this process occurs while a transition is occurring, it is possible that
intensity is transferred from one spectral line to another.

The line-mixing absorption coefficient was re-written in doubled-state form 
better suited for computation of infrared spectra by E. W. Smith
\cite{smi:81}.  If the eigenvalues of $H_{0}$
are the line center frequencies (energy and frequency are used
interchangeably here) and the interaction, $H_{1}$, is linearly
proportional to the pressure $P$ (binary collisions), $k_{mix}(\nu)$
can be re-written as\cite{str:88*1,smi:81,ben:75}
\begin{equation} 
k_{mix}(\nu)=\frac{N}{\pi} {\bf IM} \left( \sum_{j,k} d_{j}\ll 
j\mid
[(\nu-\nu_{0})-\imath P{\bf W}]^{-1} \mid k\gg d_{k}\rho_{k}\right)
\label{eqn:kmixz}
\end{equation}
where $N$ is the molecular density of absorbers, $d_{j}$ and $d_{k}$ are the
dipole moment matrix elements corresponding to the radiative transitions
$|j\gg$ and $|k\gg$, ${\bf \nu}$ is a diagonal matrix with 
$\ll j|{\bf \nu}|k\gg =\nu\delta_{jk}$, $\rho_{k}$ is a density matrix
element  that represents the population difference between the upper and
lower levels of the transition $|k\gg$, and P is the gas pressure. {\bf W}
is the relaxation, or interaction, matrix with off-diagonal 
elements, $W_{j,k}(j\neq k)$, representing the rate, or magnitude, at which
collisions transfer intensity from line $k$ to $j$ and its diagonal 
elements, $W_{j,j}$, representing the line widths of line $|j>>$.

Calculating an absorption coefficient for many transitions over a
large spectral range using Equation \ref{eqn:kmixz} involves the inversion
of a large matrix for each desired frequency.  For this reason, 
$k_{mix}(\nu)$ was re-written as \cite{gor:67}
\begin{equation} 
k_{mix}(\nu)=\frac{N}{\pi} {\bf IM} \left({\bf d}\cdot{\bf G}({
\bf
  \nu})^{-1} \cdot{\bf \rho}\cdot{\bf d}\right)
\end{equation}
where ${\bf G}={\bf \nu}-{\bf H}$ and 
${\bf H}={\bf \nu_{0}}+\imath P{\bf W}$.
${\bf H}$ is diagonalized with a complex matrix ${\bf A}$ to get the 
diagonal matrix ${\bf L}={\bf A^{-1}}\cdot{\bf H}\cdot{\bf A}$.
${\bf G}$ is also diagonalized by ${\bf A}$ and $k_{mix}(\nu)$ is written
as
\begin{equation} 
k_{mix}(\nu)=\frac{N}{\pi} {\bf IM} \left(\sum_{i}\frac{({\bf d
\cdot
    A})_{i} ({\bf A^{-1}\cdot\rho\cdot d})_{i}}{\nu - l_{i}}\right)
\label{eqn:kmix2}
\end{equation}
where $l_{i}$ are the diagonal elements of ${\bf L}$.  In this form, the 
calculation of the absorption coefficient requires only one matrix 
inversion.

Using time-independent perturbation theory, the second order energy
correction for an interaction potential, $H_{1}$, is given by
$\sum_{m\neq k}\frac{|(\Phi_{k},H_{1}\Phi_{m})|^{2}}{E_{k}-E_{m}}$.
Following this approach, and assuming that $PW_{jk}/(\nu_{j}-\nu_{k})$ is
small for all lines, Rosenkranz found the first-order approximation for
$k_{mix}(\nu)$ to be \cite{ros:75,ros:88}
\begin{equation}
k_{1st}(\nu)=\frac{N}{\pi}\sum_{j} S_{j}
\left(\frac{P \gamma_{j}+ 
(\nu-\nu_{j})PY_{j}}{(\nu-\nu_{j})^{2}+(P\gamma_{j})^{2}}\right)
\; \; \mbox{with} \; \; 
Y_j=2\sum_{k\neq j}\frac{d_k}{d_j}\frac{W_{kj}}{\nu_j-\nu_k}
\label{eqn:kmix}
\end{equation}
where 
%$N$ is the absorber density, $P$ is the pressure, $S_j$, $\nu_j$, and
%$\gamma_j$ are the line strength, center and half-width, and 
$\mbox{Y}_j$ are first-order mixing coefficients.  For a single transition, 
this lineshape is the sum of a Lorentzian and an asymmetric term.  Far from
the line centers, the asymmetric terms become proportional to $\nu^{-1}$.  
In order for $k_{1st}(\nu)$ to go to zero in these regions, the sum of the
coefficients must vanish.  That is, detailed balance must be obeyed.  In 
this context, Strow and Reuter\cite{str:88*1} showed that detailed balance 
is obeyed if
\begin{equation}
 \sum_{j} S_{j} Y_{j}=0.
\end{equation}
They also used this result to show that, in the far-wing limit, the 
ratio of mixing and Lorentz absorption coefficients is a
constant\cite{str:88*1}:\footnote{The ratio of $k_{mix}$ to Lorentz is also
a constant in the far-wing.}
\begin{equation}
 \frac{k_{1st}}{k_{L}}= 1 + \frac{\sum_{j}S_{j}Y_{j}\nu_{j}}
{\sum_{j}S_{j}\gamma_{j}}
\end{equation}
This is a useful result because it allows the mixing lineshape to be
calculated by simply multiplying the Lorentz lineshape by a constant in
the far wing.
$k_{1st}(\nu)$ is also useful because it provides a compact form for
expressing the effects of line mixing.  The amount of line mixing 
for a single line can be represented by the magnitude of 
the first-order mixing coefficients.
Thus, lines which experience a large amount of mixing will have $Y$'s of
large magnitude and those which are Lorentzian will have $Y$'s which are
zero. The accuracy of the first order mixing absorption coefficient 
decreases with pressure.  The largest errors in first-order mixing occur 
in spectral
regions where lines overlap significantly.  As long as this overlap is not
too great or as long as the spectral region of interest is in the far-wing,
however, the first-order approximation is accurate.

The only point left to be addressed is the determination of the 
off-diagonal,
or mixing, terms of ${\bf W}$.  Direct calculations of these terms are very
complex, involving a detailed knowledge of the intermolecular potentials 
and energy transfer during a collision.  More often, empirical scaling laws 
based on energy changes caused by inelastic collisions are used to model
the interactions.  Several of these laws have been developed
to model collisions for rotational-vibrational transitions, one of which is
the exponential power energy-gap, or PEG scaling law. The PEG law models the
energetically upward state-to-state inelastic collisional rates as a 
function of the rotational energy difference, $\Delta E_{j'j}$.  An upward 
rate going from the state $j$ to state $j'$ is modeled as 
\begin{equation}
 K_{j'j}=a_{1}\left(\frac{\mid\Delta E_{j'j}\mid}{B_{0}}\right)^{-a_{2}}
\exp\left(\frac{-a_{3}\mid\Delta E_{j'j}\mid}{kT}\right)
\label{eqn:krates}
\end{equation}
where $B_{0}$ is the rotational constant and $a_{1}$, $a_{2}$, and $a_{3}$
are adjustable parameters which are discussed below.  Other similar scaling 
laws are
also employed.  They include the modified exponential energy-gap, or MEG
scaling law and the ECSL law \cite{bru:82,ste:86}.  All of these
methods have shown to give similar results as the PEG law \cite{mac:92}.
A more realistic model would not only depend on the energy difference
between two levels, but would include rotational and vibrational factors 
as well.

Detailed balance is obeyed if
\begin{equation}
K_{jj'}(2j'+1)e^{-\frac{E_{j'}}{kT}}=K_{j'j}(2j+1)e^{-\frac{E_{j}}{kT}}
\end{equation}
This relation essentially ensures that energy is conserved and gives the
downward rates, $K_{jj'}$:
\begin{equation} 
K_{jj'}=K_{j'j} \frac{2j+1}{2j'+1} e^{\frac{\Delta E}{kT}}
\end{equation}

$a_{1}$, $a_{2}$, and $a_{3}$ of Equation \ref{eqn:krates} are determined 
by equating the width of a spectral line to the sum of all of the rates
which limit the lifetime of that transition via a least-square fit to the
known linewidths \cite{tob:93}.  This is based simply on the fact that
any rates which shorten the molecule's lifetime in a specific energy 
state broadens the spectral line.  These rates include all of those 
which occur in 
either the lower or upper state of the transition.  Since
vibrational energies are much greater than rotational energies, only
collisions between states within the same vibrational level are considered.
Therefore, using line widths extracted from experimental data, $a_{1}$,
$a_{2}$, and $a_{3}$ are determined by requiring
\begin{equation} \gamma_{j} \equiv {\bf W_{jj}} = \sum_{j'\neq j}
\mbox{All} \; K_{j'j} \; \mbox{which limit the transition lifetime}.
\label{eqn:lwsum}
\end{equation}
This is obviously not the exact expression used in calculations;
its details depend on the type of transition which is occurring.  This is
detailed later when specific symmetries are investigated.  
Equation \ref{eqn:lwsum} is valid only if elastic reorientation collisions 
and vibrational relaxations do not contribute significantly to the 
widths, which appears to be valid for CO$_{2}$-X systems.

The off-diagonal elements of ${\bf W}$ are then taken to be proportional 
to the corresponding collisional rates of ${\bf K}$.  These are the 
mixing terms.
For two rotational levels which are energetically close, the collisional 
rate between them is relatively large and they experience mixing.  On the 
other hand, if two levels are energetically far from each other, the 
corresponding ${\bf K}$ rates are negligible and no mixing occurs.

In order for line-mixing to occur, collisions which connect two spectral 
lines must occur in both the upper and lower vibrational states.  If, for
example, inelastic collisions occurred in only the upper vibrational state,
the only effect would be to limit the transition lifetime, and therefore
increase the line width.  
%Take, for example, a Q(2) transition. This is a
%$\Delta J=0$ transition for a $J$=2 level.  If collisions are present in
%both  upper
%and lower vibrational states, mixing will occur and it is possible that
%Q(2) will exchange intensity with neighboring spectral lines.  On the
%other hand, if collisions are present only in the upper level, the effect
%of a collision is to actually change the identity of the transition.  For
%example, a collision from $J$=2 to $J$=3 in the upper vibrational level
%which occurs during a $Q(2)$ transition
%results in the production of the $\Delta J$ =+1 transition: R(2).
This point is significant for cases in which the collision rates in 
the upper and lower vibrational states are not equal.

Summarizing the calculational procedures, the relaxation rates, $K_{jj'}$, 
are first determined by adjusting $a_{1}$, $a_{2}$,
and $a_{3}$ so that the sum of all relaxation rates which limit the
lifetime of a transition equals the known line width.  This is done using
an equation similar to Equation \ref{eqn:lwsum}.  The off-diagonal elements
of ${\bf W}$ are then taken to be proportional to the corresponding
off-diagonal elements of ${\bf K}$.  The details of this step depend on
the symmetry of the band and is discussed in Section \ref{sec:apply43}.
The diagonal elements of ${\bf W}$ are equated to the line widths.
The absorption coefficients are then calculated using 
full- ($k_{mix}$, Equation \ref{eqn:kmix2}) or first- ($k_{1st}$, Equation
\ref{eqn:kmix}) order mixing.  This procedure is incorporated into a
least-squares fitting algorithm in which an adjustable parameter
that controls the magnitude of the off-diagonal mixing terms of ${\bf W}$ is
adjusted to obtain optimum agreement with measured spectra.  

\subsection{Duration-of-Collision Effects}\label{sec:doceffect}
In addition to ignoring the effects of line mixing, the Lorentz model is
inaccurate because it is based on the impact approximation.  That is, the
duration of the interaction between two colliding molecules is assumed to
be negligible.  Therefore, the Lorentz lineshape is too large at
frequencies far from the line centers.  A more accurate 
model treats the molecules as softer charge distributions with 
collisions therefore having finite durations.  Just how long the 
collisions last is difficult to determine, but depends on how
the molecular potentials interact during a collision.  Assuming that 
the collisions have a finite duration eliminates the need for 
instantaneous derivatives in the temporal wave function and therefore 
decreases the high frequency components of the spectral distribution.
This effect is shown in Figure \ref{fig:doc_effect}.

This problem is approached by determining the behavior of the dipole 
moment during the collision.  Birnbaum developed a theory using a 
versatile empirical correlation function \cite{bir:76,bir:79} with known 
short and long time behavior and derived expressions for the absorption 
coefficient as a function of frequency for spectral transitions whose 
levels are perturbed by collisions.  A flexible model was used:
\begin{equation}
\phi(t)=<\mu(0)\cdot\mu(t)>=\exp\left(\frac{
\tau_{dur}-(\tau_{dur}^{2}+y^{2})^{\frac{1}{2}}}{\tau_{col}}\right)
\label{eqn:birn1}
\end{equation}
with $y=(t^{2}-\imath 2\tau_{0} t)^{1/2}$ and $\tau_{0}=\frac{\hbar}{2kT}$.
$\tau_{dur}$ represents the mean duration of collisions and $\tau_{col}$ is
the mean time between collisions.  The lineshape function is then
determined by Fourier analysis of $\phi(t)$.  At long times, Equation
\ref{eqn:birn1} reduces to 
$\exp\left([\tau_{dur}\pm\imath\tau_{0}-|t|]\tau_{col}^{-1}\right)$ and the
resulting line shape is Lorentzian with a width of $\tau_{col}^{-1}$. 
In terms of the Lorentz shape ($k_{L}(\nu)$), a lineshape representing an
average collision duration of $\tau_{2}$ is given by \cite{edw:91}
\begin{equation}
k_{B}(\nu)=k_{L}(\nu)\chi_{B}(\nu)=k_{L}(\nu) A_{m} z K_{1}(z)\exp\left(
\tau_{2}\gamma_{j}+\tau_{0}\Delta\nu\right)
\label{eqn:birnbaum}
\end{equation}
with
\begin{equation}
z=\sqrt{(\gamma_{j}^{2}+\Delta\nu^{2})(\tau_{0}^{2}+\tau_{2}^{2})}
\; \; \mbox{and} \; \;
\Delta\nu=\nu-\nu_{j}
\end{equation}
where $K_{1}(z)$ is a modified Bessel function of the second kind, 
$\tau_{0}=\frac{0.72}{T}$ \footnote{$\tau_0$ and $\tau_2$ have been
converted to units of $\mbox{cm}$ by multiplication by $2\pi\mbox{c}$.}, 
and A$_{m}$ is a constant representing the effect of line-mixing far 
from band center.   $\chi_{B}$ is independent of vibrational band 
except for the mixing factor $A_{m}$.  Although $\tau_2$ is expected 
to decrease with increasing temperature \cite{bir:79}, no explicit 
temperature dependence is included in this application.

This ``corrective'' factor, $\chi_{B}(\nu)$, for the Lorentz line shape 
removes much of the far-wing absorption due inherently to the impact 
approximation.
Since we are essentially multiplying two line shapes, this can be seen as
taking the convolution of the Lorentzian wave train with the correlation
function.  

\begin{figure}[h]
\begin{center}\includegraphics[width=3in]{Figures/CO2/doc_effect.eps}
\end{center}
  \caption[Illustration of the Lorentz and
        finite duration-of-collision models of the time
        development of the dipole moment autocorrelation function
        and the resulting lineshapes.] 
        {A cartoon (top panel) illustrating the Lorentz (solid curve) and
        finite duration-of-collision (dashed curve) models of the time
        development of the dipole moment autocorrelation function when a 
        collision occurs.  $\tau_1$ is the time between collisions and
        $\tau_2$ is the collision duration.  The bottom panel shows the 
        resulting lineshapes with the durations-of-collisions model 
        computed with $\tau_2$/$\tau_1$=0.015.}
  \label{fig:doc_effect}
\end{figure}

\subsection{Combined Line-Mixing and Duration-of-Collision Lineshape}
\label{sec:combined}
Both line-mixing and the duration-of-collision effect have shown to reduce
the amount of absorption in the far-wing limit.  Since the line-mixing
theory is valid only under the impact approximation, the combined effects
of line-mixing and duration-of-collision are approximated as if each
effect were independent.  In order to include the effects of line-mixing
over the entire frequency range, $k_{L}(\nu) A_{m}$ in 
Equation (\ref{eqn:birnbaum}) is replaced by the first order mixing 
absorption coefficient:
\begin{equation}
k(\nu)=\sum_i k_{1st}(\nu_i,\nu)\chi_{B}(\nu_i,\nu).
\label{eqn:proposed}
\end{equation}
{\em This represents the proposed lineshape model}.  The full line-mixing 
lineshape $k_{mix}(\nu)$ can also be implemented by using 
\[
k(\nu)=\frac{k_{mix}(\nu)}{\sum_i k_{Lor}(\nu_i,\nu)}
\sum_i k_{Lor}(\nu_i,\nu)\chi_B(\nu_i,\nu)
\]
if the first-order approximation is too inaccurate (which is seldom true 
for atmospheric applications).
To show the sub-Lorentzian nature of this lineshape, ratios to the Lorentz 
profile are shown in Figure \ref{fig:ratios}.  Far from the $\nu_3$ band 
center, line-mixing has the overall effect of scaling the Lorentzian 
profile by a constant, approximately 0.10.  Also in the far-wing, $\chi_B$
is well approximated by a decaying exponential.  One obvious result of
these calculations is that line-mixing is responsible for the majority of
the observed sub-Lorentz behavior directly beyond the band-head.

\begin{figure}[h]
\begin{center}\includegraphics[width=4in]{Figures/CO2/ratios.eps}
\end{center}
  \caption[Ratios of the line-mixing, duration-of-collision, and combined
   absorption coefficients to Lorentz for the fundamental $\nu_3$ R-branch] 
    {Ratios of the line-mixing (Equation \ref{eqn:kmix}),  
    duration-of-collision (Equation \ref{eqn:birnbaum}), and combined
    (Equation \ref{eqn:proposed}) absorption coefficients to Lorentz 
    for the fundamental $\nu_3$ R-branch computed using $\zeta$=1 
    and $\tau_2$=0.0275.  In the far wing, the ratios of line-mixing to
    Lorentz is a constant while the finite duration-of-collision effect 
    leads to a decaying exponential.} 
  \label{fig:ratios}
\end{figure}

Summarizing, Equation \ref{eqn:proposed} represents the proposed lineshape
model.  It includes both line-mixing and duration-of-collision effects -- 
the
two phenomenon responsible for nearly all deviations from Lorentz lineshape
behavior in CO$_2$.  Because the variables $\zeta$ and $\tau_2$ are built
into the model, it is capable of producing a wide range of lineshape
behavior.  If, for example, $\zeta$ and
$\tau_2$ are set equal to zero, the resulting lineshapes are Lorentzian, 
while if $\zeta=1$ and $\tau_2 \gg 0$, the resulting lineshape is extremely
sub-Lorentzian in the far-wing. 

\newpage
\appendix{{\bf APPENDIX } : Computing the lineshapes}
These appendices duplicate some of the GENLN2 subroutines that have 
been used in this Matlab code, to compute lineshape parameters.

\subsection{qfcn.m}
This function evaluates the correction to the overall line strength, due 
to the temperature differing from 296 K. The partition function can be 
quickly evaluated as a third order polynomial :
\begin{displaymath}
Q(T)=a + bT + cT^{2} + dT^{3}
\end{displaymath} 
where the coefficients $a,b,c,d$ depend on the gasID, isotope. The output 
from the function call is a vector containing the ratio Q(296)/Q(T) for the
different isotopes.

\begin{verbatim}
function [qfcn]=q(A,B,C,D,E,lines,T); 

% initialize coefficients vectors for qtips coefficients 
a1 = ones(length(lines.ZISO),1);b1 = ones(length(lines.ZISO),1); 
c1 = ones(length(lines.ZISO),1);d1 = ones(length(lines.ZISO),1); 

% Assign coefficients according to isotope 
no_isotopes = max(lines.ZISO); 
for i = 1: no_isotopes 
  ind = find(lines.ZISO == i); 
  a1(ind) = a1(ind)*A(i); 
  b1(ind) = b1(ind)*B(i); 
  c1(ind) = c1(ind)*C(i); 
  d1(ind) = d1(ind)*D(i); 
end 

% Evaluate partition functions at desired temperature and 296K 
Qt   = a1 + b1*T   + c1*T^2   + d1*T^3; 
Q296 = a1 + b1*296.0 + c1*(296.0^2) + d1*(296.0^3); 

qfcn=Q296./Qt;

\end{verbatim}

\subsection{broad.m}
This function computes the line broadening, as a function of total and self
pressures, and layer temperature. Pressures are in atmospheres.

The code starts out differently for CO2, for water and for the rest of the 
gases, as follows, for individual lines : \\
(a) water : if self broadening $ \le $ $ \epsilon $ then\\
              self broadening = 5*air broadening\\
            else\\
              self broadening = self broadening\\
(b) others : if self broadening $ \le $ $ \epsilon $ then\\
              self broadening = air broadening\\
            else\\
              self broadening = self broadening\\
The total broadening is then
\begin{displaymath}
  brd=air broadening*(press-press_self) + self broadening*press_self
\end{displaymath}
\begin{displaymath}
  brd=(296.0/T)^{pwr}*brd;
\end{displaymath}

(c)CO2 : (done more rigorously for individual PQR lines in run6co2)\\
            if self broadening $ \le $ $ \epsilon $ then\\
              self broadening = 0.0\\
            else\\
              self broadening = self broadening\\
The total broadening is then
\begin{displaymath}
  brd_for=air broadening*(press-press_self)*(296.0/T)^{pwr}
\end{displaymath}
\begin{displaymath}
  brd_self=self broadening*(press-press_self)*(296.0/T)^{0.685}
\end{displaymath}
\begin{displaymath}
  brd=brd_self+brd_for
\end{displaymath}

\begin{verbatim}
function [brd]=broad(press,press_self,press_ref,air,self,pwr,T,iGas)
%compute the broadening by combing air and self broadening
%remember units are in cm-1 per atm at 296 K, so we need the pressures
% press      = current AIRS pressure in atm
% press_self = current self pressure in atm
% press_ref  = current reference pressure in atm
% air        = air broadening cm-1/atm at 296 k
% self       = self broadening cm-1/atm at 296 k
% pwr        = power relationship to scale brd wrt temperature
% T          = temperature
%iGas        = GAS ID

%this eqn is from pg 31 of Genln2 manual

%%%brd=air*(press-press_self)/press_ref + self*press_self/press_ref;
%assume press_ref = 1.0 atm

if iGas ~= 2
  %this is the vectorised code
  dummysmall = (self < eps);
  dummybig   = (self >= eps);
  slfb=self.*dummybig;

  if (sum(dummysmall) > 0)
    if (iGas == 1)
      slfs=(5*air).*dummysmall;
    else
      slfs=air.*dummysmall;
      end
  else
    slfs=zeros(size(self));
    end
  slf=slfs+slfb;

  brd=air*(press-press_self) + slf*press_self;
  brd=(296.0/T).^(pwr).*brd;

else
  %this is the vectorised code
  dummysmall = (self < eps);
  dummybig   = (self >= eps);
  slfb=self.*dummybig;

  if (sum(dummysmall) > 0)
    slfs=air.*dummysmall;
  else
    slfs=zeros(size(self));
    end
  slf=slfs+slfb;

  brdf=(press-press_self)*air.*(296.0/T).^(pwr);
  brds=(press_self)*slf.*(296.0/T).^(0.685);
  brd=brdf+brds;
  end
\end{verbatim}

\subsection{findstren.m}

This subroutine finds the line center strength. It first changes the 
linestrength that is in the HITLIN database back to the units that are
in the HITRAN databse by multiplying by Avogadros number * 1000 
\begin{displaymath}
  s00=s0*6.022045e26
\end{displaymath}
The line strength is then found from 
\begin{displaymath}
S(T) = q S(296) \frac{Q(296)}{Q(T)} s_{b} s_{e}
\end{displaymath}
where q= gas amount in GENLN2 units of $kilomoles/cm^{2}$, S(296) is the 
line strength read off the tape, Q(296)/Q(T) is the partition function 
correction, $s_{b}$ is a Boltzmann factor accounting for lower state 
population at 
temperature $T$ (and hence depending on energy of lower state $E_{li}$), and
$s_{e}$ accounts for detailed balance.

\begin{verbatim}
function  [strength]=find_stren(qfcn,v0,T,E_li,s0,amt)
%renormalises the strength based on eqn in pg 31 of Genln2 manual
%qfcn = Q(T)/Q(296)
%v0   = central wavenumber
%T    = temperature
%s0   = strength
%E_li = lower state energy
%amt  = gas amt (kilomolecules/ cm2)

s00=s0*6.022045e26;                     %or could do amt=amt*6.022e26
c2=1.4387863;                           %K/ cm-1  from Genln2 manual
sb=exp(-c2*E_li/T)./exp(-c2*E_li/296.0); %boltzman factor (distr at T)
se=(1-exp(-c2*v0/T))./(1-exp(-c2*v0/296.0)); %adjust for detailed balance

strength=amt*(qfcn'.*s00.*sb.*se);
\end{verbatim}

This is the individual line strength used in the computations of the lines,
whether they are lorentz or voigt or doppler or ......

\bibliographystyle{unsrt}
\bibliography{/salsify/packages/Tex/atmspec}

\end{document}